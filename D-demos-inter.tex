\subsection{Demostración del lema \ref{lem-mcm}}
\label{app-lem-mcm}		

Dado $n \geq 1$, sea $d_n=\MCM(n)$, y para $m \in \{1, \ldots, n\}$, sea  
		\begin{eqnarray*}
		I(m,n) &=& \int_{0}^{1}x^{m-1}(1-x)^{n-m}dx.
		\end{eqnarray*}
Se tiene que
		\begin{eqnarray*}
			\int_{0}^{1}x^{m-1}(1-x)^{n-m}dx  &=&
				  \int_{0}^{1}\sum_{i=0}^{n-m}{n-m\choose i}(-1)^ix^{i+m-1}dx\\
				  &=&
				  \sum_{i=0}^{n-m}\bigg[{n-m\choose i}(-1)^i \int_{0}^{1}x^{i+m-1}dx\bigg]\\
				  &=& \sum_{i=0}^{n-m}\bigg[{n-m\choose i}(-1)^i \bigg(\frac{x^{m+i}}{m+i}\bigg)\bigg\rvert_0 ^1\bigg]\\
				  &=&\sum_{i=0}^{n-m}{n-m\choose i}(-1)^i \frac{1}{m+i}
		\end{eqnarray*}
		Del desarrollo anterior concluimos que $I(m,n)\cdot d_n\in \mathbb{N}$, ya que $1 \leq m+i\leq n$ para cada $i \in \{0, \ldots, n-m\}$ y $\int_{0}^{1}x^{m-1}(1-x)^{n-m}dx \geq 0$, puesto que en el intervalo $[0,1]$ se tiene $x \geq 0$ y $(1-x) \geq 0$.
%		\comentario{que paso aca? no hice new page}
		Por otro lado, si calculamos $I(m,n)$ por partes, tenemos:	
		\begin{eqnarray*}
		I(m,n)&=&\int_{0}^{1}x^{m-1}(1-x)^{n-m}dx\\
		&=&\bigg(x^{m-1}(-1)\frac{(1-x)^{n-m+1}}{n-m+1}\bigg)\bigg\rvert_0 ^1 -\int_0^1(-1)\frac{(1-x)^{n-m+1}}{n-m+1}(m-1)x^{m-2}dx\\
		&=& \frac{m-1}{n-m+1}\int_{0}^{1}x^{m-2}(1-x)^{n-m+1}dx\\
		&=& \frac{m-1}{n-m+1}\bigg[\bigg(x^{m-2}(-1)\frac{(1-x)^{n-m+2}}{n-m+2}\bigg)\bigg\rvert_0 ^1 -\int_0^1(-1)\frac{(1-x)^{n-m+2}}{n-m+2}(m-2)x^{m-3}dx\bigg]\\
		&=& \frac{(m-1)(m-2)}{(n-m+1)(n-m+2)}\int_{0}^{1}x^{m-3}(1-x)^{n-m+2}dx\\
		& \vdots &\\
		&=& \frac{(m-1)(m-2)\cdots(m-(m-1))}{(n-m+1)(n-m+2)\cdots(n-m+(m-1))}\int_{0}^{1}x^{m-m}(1-x)^{n-m+(m-1)}dx\\
		&=& \frac{(m-1)(m-2)\cdots 2\cdot 1}{(n-m+1)(n-m+2)\cdots(n-1)}\int_{0}^{1}(1-x)^{n-1}dx\\
		&=& \frac{(m-1)!}{\frac{(n-1)!}{(n-m)!}}\bigg((-1)\frac{(1-x)^n}{n}\bigg)\bigg\rvert_0^1\\
		&= &\frac{(m-1)!(n-m)!}{n(n-1)!} \ = \ \frac{m(m-1)!(n-m)!}{m \cdot n!} \ = \ \frac{m!(n-m)!}{m\cdot n!} \ = \ \frac{1}{m{n\choose m}}
		\end{eqnarray*}
%		\begin{align}
%		I(m,n)&=\int_{0}^{1}x^{m-1}(1-x)^{n-m}dx\nonumber\\
%		&=x^{m-1}(-1)(1-x)^{n-m+1}\frac{1}{n-m+1}\bigg\rvert_0 ^1 -\int_0^1(-1)\frac{(1-x)^{n-m+1}}{n-m+1}(m-1)x^{m-2}dx\nonumber\\
%		&=\frac{m-1}{n-m+1}\int_{0}^{1}x^{m-2}(1-x)^{n-m+1}dx\nonumber\\
%		&=\frac{m-1}{n-m+1}(x^{m-2}(-1)(1-x)^{n-m+2}\frac{1}{n-m+2}\bigg\rvert_0 ^1 -\int_0^1(-1)\frac{(1-x)^{n-m+2}}{n-m+2}(m-2)x^{m-3}dx)\nonumber\\
%		&=\frac{(m-1)(m-2)}{(n-m+1)(n-m+2)}\int_{0}^{1}x^{m-3}(1-x)^{n-m+2}dx\nonumber\\
%		&\hspace{5cm}\vdots\nonumber\\
%		&=\frac{(m-1)(m-2)\cdots(m-(m-1))}{(n-m+1)(n-m+2)\cdots(n-m+(m-1))}\int_{0}^{1}x^{m-m}(1-x)^{n-m+(m-1)}dx\nonumber\\
%		&=\frac{(m-1)(m-2)\cdots 2\cdot 1}{(n-m+1)(n-m+2)\cdots(n-1)}\int_{0}^{1}(1-x)^{n-1}dx\nonumber\\
%		&=\frac{(m-1)!}{(n-1)!/(n-m)!}(-1)\frac{(1-x)^n}{n}\bigg\rvert_0^1\nonumber\\
%		&=\frac{m(m-1)!(n-m)!}{m\cdot n(n-1)!}=\frac{m!(n-m)!}{m\cdot n!}=\frac{1}{m{n\choose m}}
%		\end{align}
		Dado que $I(m,n)\cdot d_n\in \mathbb{N}$, concluimos entonces que $m{n\choose m}\divi  d_n$, para todo $m \in \{1, \ldots, n\}$.
		En particular, se tiene que $n{2n\choose n}$ divide a $d_{2n}$ y por lo tanto se cumple también que $n{2n\choose n}$ divide a $d_{2n+1}$.
		Por otro lado, se tiene que $(2n+1){2n\choose n}=(n+1){2n+1\choose n+1}$, y de esto se deduce que $(2n+1){2n\choose n}$ divide a $ d_{2n+1}$ (dado que $(n+1){2n+1\choose n+1}$ divide a $d_{2n+1}$). Por lo tanto, sabemos que existen $\alpha, \beta \in \mathbb{N}$ tales que:
		\begin{eqnarray*}
		\alpha \cdot n{2n\choose n} & = & d_{2n+1}\\
		\beta \cdot (2n+1){2n\choose n} & = & d_{2n+1}.
		\end{eqnarray*}
Se tiene entonces que $\alpha \cdot n = \beta \cdot (2n+1)$, de lo cual se deduce que $\beta = \gamma \cdot n$ para $\gamma \in \mathbb{N}$, puesto que $\MCD(2n+1,n)=1$. Concluimos que $n(2n+1){2n\choose n}$ divide a $d_{2n+1}$,
%(esto ya que $n$ divide a $\frac{d_{2n+1}}{{2n\choose n}}$ y $(2n+1)$ divide a $\frac{d_{2n+1}}{{2n\choose n}}$, luego $d_{2n+1}/{2n\choose n} = n\cdot (2n+1)\cdot k$, con $k$ constante) 
de lo cual se deduce que $n(2n+1){2n\choose n} \leq d_{2n+1}$.
		Además, 
		\begin{eqnarray*}
		4^n \ = \ 2^{2n} \ = \ (1+1)^{2n} \ = \ \sum_{i=0}^{2n}{2n\choose i} \ \leq \ (2n+1)\max\bigg\{{2n\choose i} \,\bigg|\, 0\leq i\leq 2n\bigg\}\ =\ (2n+1){2n\choose n}.
		\end{eqnarray*}
		Multiplicando por $n$ ambos lados de la desigualdad y considerando que $n(2n+1){2n\choose n} \leq d_{2n+1}$, nos queda
		\begin{eqnarray*}
			n\cdot 4^n \ \leq \ n(2n+1){2n\choose n} \ \leq \ d_{2n+1}
		\end{eqnarray*}
		Así, si $n\geq 2$ entonces $d_{2n+1}\geq n\cdot 4^{n} = n\cdot 2^{2n}\geq 2\cdot 2^{2n}=2^{2n+1}$, y la propiedad mencionada en el lema se cumple para todos los impares mayores o iguales a 5. Además, si $n\geq 4$ entonces $d_{2n+1}\geq n\cdot 4^{n}\geq 4\cdot 4^{n}=2^{2n+2}$. Así, dado que $d_{2n+2}\geq d_{2n+1}$, se concluye que $d_{2n+2}\geq 2^{2n+2}$ y la propiedad mencionada en el lema se cumple para todos los pares mayores o iguales a 10.
%		(pares mayores o iguales a 9). 
Finalmente, como $d_8=840 \geq 2^8$, podemos afirmar que $\MCM(n) \geq 2^n$ para todo $n \geq 7$, lo cual concluye la demostración del lema. 

Como comentario final, nótese que la propiedad no se cumple para $n = 6$ puesto que $\MCM(6) = 60$ y $2^6 = 64$. Dejamos al lector como ejercicio verificar que la propiedad tampoco se cumple para $n=4$.
%$\forall N\geq 7$, $d_N\geq 2^N$ (se puede verificar que para los $N\leq 4$ y $N = 6$ no se cumple esta propiedad).
%		Con esto, queda demostrado el lema.
		
%\subsection{Demostración del Corolario \ref{mm-mcm}}
%	Sabemos por el Lema \ref{lem-mcm} se cumple para cualquier $m\geq 7$. Sea $p$ un primo tal que $p^{\alpha}$ es la mayor potencia de $p$ que divide a $d_{m}$. Como $p$ es primo, entonces $p^{\alpha}$ divide a algún $j$ en $1\leq j\leq m$\comentario{esto pensarlo}. Esto implica que $p^{\alpha}\leq m=p^{\log_p m}=p^{\frac{\log m}{\log p}}$\comentario{que feo este exponente}. Como esto pasa para todos los primos menores o iguales que $m$, entonces
%	\begin{align}
%		d_{m}\leq \prod_{p\leq m}p^{\frac{\log m}{\log p}}\nonumber
%	\end{align}\comentario{esto explicar un poco mas?}
%	Luego, por el Lema \ref{lem-mcm} tenemos que 
%	\begin{align}
%		&2^{m}\leq \prod_{p\leq m}p^{\frac{\log m}{\log p}},\hspace{0.5cm} \text{y aplicando logaritmo en ambos lados}\nonumber\\
%		\Leftrightarrow &m\log 2\leq \log (m)\cdot \pi(m)\nonumber\\
%		\Leftrightarrow &\pi(m)\geq \frac{m\log 2 }{\log m}\nonumber	
%	\end{align}
%	Esta desigualda se cumple para todo $m\geq 7$. Se puede verificar rápidamente que tambien se cumple cuando $m$ toma los valores de 4, 5 y 6. Con esto terminamos la demostración del corolario.
		
\subsection{Lema \ref{prop-1}: demostración de que $\log^2 n + 2\lfloor \log B \rfloor \leq \log^4 n$, para $n\geq 5$}
\label{app-prop-1}
Sabemos que si $n$ es un natural positivo entonces $\log n \leq n$. Supongamos que $n\geq 10$. Si desarrollamos la primera desigualdad tenemos que 
\begin{eqnarray*}
\log^5 n \leq n^5 &\Rightarrow& \lceil\log^5 n\rceil \leq \lceil n^5\rceil \ = \ n^5\\
	&\Rightarrow& B \leq n^5\\
	&\Rightarrow& \log B  \leq \log {n^5}\\
	&\Rightarrow& \lfloor\log B\rfloor  \leq \log {n^5} \ = \ 5\log n	 
\end{eqnarray*}

De lo anterior podemos afirmar que
\begin{eqnarray*}
	\log^2 n + 2\lfloor \log B \rfloor
	&\leq& \log^2 n + 10\log n\\
	&\leq& \log^2 n + 10\log^2 n \ = \ 11\log^2 n
\end{eqnarray*}
Además, como $11\leq\log^2 10$, entonces
\begin{eqnarray*}
	11\log^2 n &\leq& \log^2 10 \cdot\log^2 n\\
	&\leq& \log^2n\cdot \log^2n\\
	&=& \log^4 n
\end{eqnarray*}
De esta forma concluimos que si $n\geq 10$, entonces $\log^2 n + 2\lfloor \log B \rfloor \leq \log^4 n$. Para $n\in \{5,\ldots, 9\}$ podemos calcular manualmente los valores de ambos lados de la desigualdad, y ver que esta también se cumple.


