

Desde la antiguedad los números primos han sido estudiados en
profundidad, permitiendo el desarrollo y descubrimiento de distintas
propiedades acerca de estos, muchas de las cuales no son evidentes.
%las que no son evidentes, y muchas veces
%son, poco intuitivas para las personas.
De hecho, determinar si un número es primo no es para nada sencillo si
el número es relativamente grande, de hecho para hacer esto es
necesario, en principio, realizar una larga serie de cálculos que
toman mucho tiempo. Lo anterior conduce a preguntarse cuánto tiempo
se necesita para resolver este problema.
%Es por esto que se llega a la
llegando a la siguiente pregunta:
%siguiente interrogante:
¿determinar, dado un número $n$, si $n$ es primo es
un problema que se puede resolver con un algoritmo que demora tiempo
polinomial en el tamaño del número $n$ (vale decir, en la cantidad de
dígitos de $n$)?
    
        Luego de más de 2000 años proponiendo algoritmos para determinar si un número es primo,\footnote{Como un ejemplo de esto, la criba de Eratóstenes es un algoritmo para determinar todos los primos menores o iguales a un número natural $n$  dado. Este algoritmos fue desarrollado por Eratóstenes, quien vivió entre los años  276 y 194 a. de C.} en el año 2004 se propone
    %presenta 
    %un artículo del test de primalidad 
    en 
    %AKS 
    \cite{AKS04}
    %, en donde se propone 
    un algoritmo determinista de primalidad que tiene complejidad polinomial. En 
    %el \textit{paper} 
    %este
    el artículo se demuestra también la correctitud del algoritmo, donde se usan resultados de teorí­a de números y de álgebra abstracta. 
    
    El algoritmo presentando en \cite{AKS04} es conocido hoy en día como test de primalidad AKS. Este algoritmo en sí­ es muy sencillo de programar, sin embargo, la demostración de la correctitud no es para nada trivial, a pesar de que están las demostraciones explicadas a grandes rasgos en el documento en el que se presenta el test. La cantidad de documentos que hay que estudiar para poder leerlo fluidamente no permite una lectura continua para alguien que no 
    %está acostumbrado a trabajar 
    trabaja de forma habitual en estas áreas. Por esto, 
    el presente trabajo consiste, primero, en demostrar detalladamente cada teorema y lema dado en \cite{AKS04},
    %el \textit{paper} oficial del algoritmo AKS, 
    especificando cada herramienta utilizada de las áreas mencionadas anteriormente, mostrando las ideas básicas necesarias para comprender las demostraciones.

