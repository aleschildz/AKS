\documentclass[10pt]{article}
\usepackage[utf8]{inputenc}
%\usepackage[latin1]{inputenc}
\usepackage[spanish]{babel}
\usepackage{amssymb}
\usepackage{amsmath}
\usepackage{amsthm}

\usepackage{mathtools}
\usepackage{listings}
\usepackage{algorithm}
\usepackage[noend]{algorithmic}
\usepackage{setspace}
\usepackage{todonotes}
\usepackage{fullpage}
\usepackage{xspace}
\usepackage{hyperref}
\usepackage{faktor}
\usepackage{xcolor}
%\usepackage{bbold}
%\usepackage[lighttt]{lmodern}


\setlength{\parskip}{4pt}
\setlength{\parindent}{0pt}
\newcommand{\R}{\mathrm{I\!R}}
\newcommand{\N}{\mathrm{I\!N}}
\newcommand{\0}{\mathbf{0}}
\newcommand{\1}{\mathbf{1}}
%(no) divide a
%\newcommand{\nodiv}{\hspace{-4pt}\not|\hspace{1.8pt}}
\newcommand{\nodiv}{\text{$\not|$}\hspace{1.8pt}}
\newcommand{\divi}{\hspace{1.8pt}|\hspace{1.8pt}}

%MODULOS QUE USE
%\newcommand{\modulo}{\hspace{0.4cm}\mod\hspace{0.1cm} (X^r-1,n)}
%\newcommand{\modulox}{\hspace{0.4cm}mod\hspace{0.1cm} (X^r-1)}
%\newcommand{\modulop}{\hspace{0.4cm}mod\hspace{0.1cm} (X^r-1,p)}
%\newcommand{\modulohp}{\hspace{0.4cm}mod\hspace{0.1cm} (h(X),p)}
%
%CAMBIANDOS PARA USAR LATEX COMANDO \mod
\newcommand{\modulo}{\mod(X^r-1,n)}
\newcommand{\modulox}{\mod (X^r-1)}
\newcommand{\modulop}{\mod (X^r-1,p)}
\newcommand{\modulohp}{\mod (h(X),p)}
\newcommand{\modulos}{\!\!\mod(X^r-1,n)}
\newcommand{\moduloxs}{\!\!\\mod (X^r-1)}
\newcommand{\modulops}{\!\!\mod (X^r-1,p)}
\newcommand{\modulohps}{\!\!\mod (h(X),p)}
\newcommand{\moduloy}{\!\!\mod (Y^r-1,p)}
\newcommand{\modulomr}{\!\!\mod (X^{m_1r}-1,p)}

%\newcommand{\modl}{\hspace{0.4cm}\text{mod}\hspace{0.1cm}}
\newcommand{\modl}{\mod}
\newcommand{\mods}{\,\text{\rm mód}\,}
\newcommand{\MCD}{{\rm mcd}}
\newcommand{\MCM}{{\rm mcm}}
\newcommand{\poly}{{\rm poly}}
\newcommand{\ord}[1]{\mathrm{ord}_{#1}}

%ESTO LO OCUPE MUCHO EN EL ULTIMO LEMA
\newcommand{\mia}{\lfloor\sqrt{t}\log  n\rfloor}

%PARA OPERACIONES EN TEORIA DE CUERPOS
\newcommand{\+}{\oplus}
\newcommand{\x}{\otimes}

%COMENTARIOS (AUN NO SE COMO COMENTAR A LA IZQUIERDA)
%\newcommand{\comentario}{\todo[line,color=green!40, linecolor=red]}
%\newcommand{\comentarioin}{\todo[line,color=green!40, linecolor=red,inline]}
%\newcommand{\comentarior}{\todo[line,color=blue!40, linecolor=red,inline]}
%\newcommand{\demostrar}{\todo[line,color=red!30, linecolor=red]}

\newcommand{\comentario}[1]{}
\newcommand{\comentarioin}[1]{}
\newcommand{\comentarior}[1]{}
\newcommand{\demostrar}[1]{}

%PARA PERMITIR QUE LAS ECUACIONES LARGAS PUEDAN SER CORTADAS Y PUESTAS EN PAGINAS DISTINTAS
\allowdisplaybreaks

\newtheorem{theorem}{Teorema}[section]
\newtheorem{corollary}[theorem]{Corolario}
\newtheorem{lemma}[theorem]{Lema}
\newtheorem{proposition}[theorem]{Proposición}
\newtheorem{afirmacion}[theorem]{Afirmación}

\newtheorem*{lemma*}{Lema}
\newtheorem{observacion}[theorem]{Observación}
\newtheorem{prop}[theorem]{Proposición}
\theoremstyle{definition}
\newtheorem{definition}[theorem]{Definición}
\theoremstyle{remark}
\newtheorem{example}[theorem]{Ejemplo}
\theoremstyle{remark}
\newtheorem*{remark}{Observación}

\DeclareMathOperator{\diam}{diam}
\DeclareMathOperator{\dis}{d}
\DeclareMathOperator{\inte}{int}
\DeclarePairedDelimiter\abs{\lvert}{\rvert}
\newcommand{\norm}[1]{\left\lVert#1\right\rVert}
\newenvironment{sciabstract}{
    \begin{quote}}
    {\end{quote}}

\renewcommand\qedsymbol{$\blacksquare$}
\newcommand{\tq}{\mid}
\newcommand{\grado}{{\rm gr}}
\newcommand{\AKS}{{\rm \textbf{AKS}}}

\newcommand{\PRIMO}{\text{\rm PRIMO}}
\newcommand{\COMPUESTO}{\text{\rm COMPUESTO}}
\newcommand{\gen}[1]{\langle #1 \rangle}

\title{{\bf El Test de Primalidad AKS}}
\author{Bernardo Barías \\ 
\texttt{bjbarias@uc.cl}
\and 
Marcelo Arenas\\
\texttt{marenas@ing.puc.cl}
\and
María Alejandra Schild\\
\texttt{aleschildz@mat.uc.cl}}

\date{\today}

\begin{document}
	
	\maketitle
	
	
%%%	%aqui poner de que trata el documento
	\begin{abstract}
\noindent El test de primalidad AKS \cite{AKS04} es el primer algoritmo
determinista capaz de decidir en tiempo polinomial si un entero positivo es primo.
          %Antes de este test no existían algoritmos que resolvieran este problema en tiempo polinomial de forma determinista.
En este documento haremos una revisión detallada y autocontenida de su funcionamiento y correctitud. Anexamos al final todo el material necesario para comprender las herramientas de teoría de números y de álgebra abstracta utilizadas.
\end{abstract}


\section{Introducción}
Una de las preguntas más fundamentales de la matemática, y que ha sido estudiada desde la Antigüedad, es cómo decidir rápidamente si un entero positivo es primo o compuesto. Dicho problema no solo tiene una enorme importancia teórica, sino que también es crucial en criptografía.

Dado un entero positivo $n$, el algoritmo más sencillo posible para decidir primalidad sería revisar si existe algún divisor de $n$ que sea mayor que $1$ y menor que $n$. Esto requerirá del orden de $n$ divisiones. No es difícil convencerse de que, en realidad, solo es necesario hacer esa revisión hasta $\sqrt{n}$, pues cualquier número compuesto debe tener un divisor en ese rango. Esto reduce el número de divisiones del algoritmo al orden de $\sqrt{n}$. Sin embargo, esto sigue siendo demasiado lento. Por ejemplo, decidir si un número de $100$ dígitos es primo usando ese algoritmo podría requerir hasta $\sqrt{10^{100}} = 10^{50}$ divisiones, lo que excede por mucho la capacidad de un computador moderno. Puesto que los protocolos criptográficos actuales trabajan con números de cientos de dígitos, ese algoritmo no es útil. Cuando decimos que nos gustaría contar con un algoritmo rápido, nos referimos a que el número de divisiones debiese crecer de manera polinomial respecto al número de dígitos de la entrada. En otras palabras, necesitamos un algoritmo cuya complejidad en el peor de los casos sea un polinomio respecto a $\log(n)$. Por supuesto, la complejidad asintótica $\sqrt{n}$ que acabamos de discutir no cumple esa condición.


\textcolor{red}{Ale: aquí podríamos agregar una breve discusión sobre tests deterministas bajo cierta condición no probada, y tests aleatorizados. Y terminar diciendo que AKS es el primer algoritmo determinista que corre en tiempo polinomial sin condiciones.}


Finalmente, en el año 2004, Manindra Agrawal, Neeraj Kayal y Nitin Saxena proponen en \cite{AKS04} un algoritmo de primalidad determinista que corre en tiempo polinomial, basado en resultados de teorí­a de números y de álgebra abstracta. Dicho algoritmo es conocido como el test de primalidad AKS. Si bien es muy sencillo de programar, las razones por las que el algoritmo funciona son profundas, y pueden resultar inaccesibles para quienes no están familiarizados con las técnicas utilizadas en la demostración de correctitud presentada en \cite{AKS04}. Por esto, el objetivo del presente trabajo es dar una explicación ampliada y autocontenida del algoritmo, partiendo desde las ideas básicas que están detrás.


\section{Motivación}
\label{sec-notacion}
Recordemos que, dado un entero positivo $n$, el conjunto $\mathbb{Z}_n \coloneq \{0, 1, \dots, n-1\}$ forma un grupo abeliano con la operación de suma en módulo $n$. Usaremos la notación $a \mods n$ para denotar al resto de dividir el entero $a$ en $n$, y también escribiremos $a \equiv b \modl n$ cuando $(a \mods n) = (b \mods n)$.

Para contestar la pregunta de si $(\mathbb{Z}_n, \cdot)$ es un grupo, necesitamos entender cuándo un número es invertible bajo la multiplicación en $\mathbb{Z}_n$. Un entero $a$ se dice \emph{invertible} en módulo $n$ si existe un entero $b$ tal que $a \cdot b \equiv 1 \modl n$. Por ejemplo, el número 2 es invertible en módulo 9
puesto que $2 \cdot 5 \equiv 1 \modl 9$, mientras que el número 3 no es invertible en módulo 9, pues para cualquier entero $b$ tenemos que el resto de dividir $3b$ en $9$ es $0$, $3$ o $6$. De hecho, es sencillo probar que $a$ es
invertible en módulo $n$ si y solo si $a$ y $n$ son coprimos, es decir, si el máximo común divisor de $a$ y $n$ es 1, lo cual es denotado como $\MCD(a,n) = 1$. El argumento utiliza el siguiente resultado bien conocido en teoría de números:

\begin{prop}[Lema de Bézout] \label{bezout}
Para todo par de enteros positivos $a$ y $b$ existen $x, y \in \mathbb{Z}$ tales que $ax+by = \MCD(a, b)$.
\end{prop}

En nuestro caso, si $\MCD(a, n) = 1$, entonces existen $x, y \in \mathbb{Z}$ tales que $ax+ny = 1$. Luego $ax \equiv 1 \modl n$, y entonces $a$ es invertible en módulo $n$.
Así, podemos concluir que, para $n \geq 2$,
$(\mathbb{Z}_n, \cdot)$ no es un grupo porque $0$ no es
invertible. Más aún, tenemos que $(\mathbb{Z}_n - \{0\},
\cdot)$ es un grupo si y solo si $n$ es un número primo, ya que $n$ es primo si y solo si todos los números en el conjunto $\{1, \ldots,
n-1\}$ son coprimos con $n$, lo que equivale a que todos sean invertibles en módulo $n$.

En general, para un entero $n \geq 1$, definimos $\mathbb{Z}_n^*$ como el conjunto de los $a \in \{1, \dots, n-1\}$ tales que $\MCD(a, n) = 1$. Por lo que discutimos en el párrafo anterior, $(\mathbb{Z}_n^*, \cdot)$ es un grupo abeliano. El orden de este grupo, que corresponde a la cantidad de enteros entre $1$ y $n-1$ que son coprimos con $n$ es una función fundamental en teoría de números.

\begin{definition}[Función $\varphi$ de Euler]
Dado un entero $n \geq 1$, definimos $\varphi(n) \coloneqq |\mathbb{Z}_n^*|$. \hfill$\blacksquare$
\end{definition}

Notemos que $\varphi(n) = n - 1$ si y solo si
$n$ es un número primo. Por lo tanto, calcular la función $\varphi$ de un número es al menos igual de difícil que verificar si este es primo.


Cuando estemos trabajando con el grupo $(\mathbb{Z}_n^*, \cdot)$, usaremos la notación $\langle a\rangle_n$ para denotar a $\langle a \rangle$, el subgrupo generador por $a$. Esta precisión adicional se introduce para evitar ambigüedad cuando se trabajan con varios grupos $(\mathbb{Z}_n^*, \cdot)$ (con parámetros $n$ distintos) a la vez.

\begin{definition}[Orden multiplicativo]\label{def_ord_mult}
	 Sean $a, n\in \mathbb{Z}$ tales que $n \geq 1$ y $\MCD(a,n)=1$. Decimos que el \emph{orden multiplicativo} de $a$ en módulo $n$ es
         \begin{eqnarray*}
         \ord{n}(a) \coloneqq \ord{\mathbb{Z}_n^*}(b) = |\langle b \rangle _n|,
         \end{eqnarray*}
	donde $b$ es el resto de dividir $a$ en $n$. \hfill$\blacksquare$
\end{definition}

La definición \ref{def_ord_mult} es correcta pues, si $\MCD(a, n) = 1$, entonces $b \in \mathbb{Z}_n^*$.


\ale{hasta aquí es nuevo}


Dado que $(\mathbb{Z}_n^*, \cdot)$ es un grupo de orden $\phi(n)$ cuyo
neutro es el número 1, tenemos que $a^{\phi(n)} \equiv 1 \modl n$ para
cada $a \in \mathbb{Z}_n^*$. De esta forma, sabemos que para cada $a
\in \mathbb{Z}_n^*$, existe algún valor $r \geq 1$ tal que $a^r \equiv
1 \modl n$, y el siguiente número, conocido como el orden de $a$ en
$n$, está bien definido:
\begin{eqnarray*}
  O_n(a) & = & \min \{ r \in \mathbb{N} \mid 1 \leq r \leq \phi(n)
  \ \text{ y } \ a^r \equiv 1 \modl n \}.
\end{eqnarray*}
Este valor también juega un rol fundamental en test de primalidad AKS.

Otro concepto clave para el test de primalidad AKS es el de
polinomio en $\mathbb{Z}_n$, que definimos a continuación. Dada una
variable $X$ (también llamado indeterminado), recuerde que
$\mathbb{Z}[X]$ es el conjunto de todos los polinomios con
coeficientes en los números enteros. Por ejemplo, $2 X^3 - 3X +7$ y
$-4 X + 8$ son polinomios en $\mathbb{Z}[X]$. Como es usual, un
polinomio $p(X) \in \mathbb{Z}[X]$ es nulo si todos sus coeficientes
son iguales a cero, y el grado de un polinomio no nulo $p(X)$,
denotados como $\grado(p(X))$, es definido como el mayor número $k$
tal que el coeficiente de $X^k$ no es igual a $0$. Finalmente, dos
polinomios $p(X), q(X) \in \mathbb{Z}[X]$ son iguales si ambos son
nulos, o si $\grado(p(X))=\grado(q(X)) = k$ y para cada $i \in \{0,
\ldots, k\}$, se tiene que los coeficientes de $X^i$ en $p(X)$ y
$q(X)$ son iguales.

Dado $n \geq 1$ y una variable $X$, $\mathbb{Z}_n[X]$ es definido como
el conjunto de todos los polinomios con coeficientes en $\mathbb{Z}_n$. Todos
los conceptos definidos para $\mathbb{Z}[X]$ son validos para
$\mathbb{Z}_n[X]$ pero considerando que las operaciones deben
realizarse en módulo $n$. En particular, sea $p(X) = \sum_{i=0}^k a_i
X^i$ y $q(X)= \sum_{i=0}^k b_i X^i$ son polinomios en
$\mathbb{Z}_n[X]$. Entonces el polinomio $r(X) = p(X) + q(X)$ es
calculado considerando que $r(X) = \sum_{i=0}^k c_i X^i$ y cada $c_i$
es igual a $a_i + b_i$ en módulo $n$, vale decir, $c_i = (a_i+b_i)
\mods n$. La multiplicación $s(X) = p(X) \cdot q(X)$ es definida bajo
la misma consideración. Finalmente, utilizamos la notación $p(X)
\equiv q(X) \modl n$ para indicar que $p(X)$ y $q(X)$ son los mismos
polinomios en $\mathbb{Z}_n[X]$, lo cual es definido como para el caso
de $\mathbb{Z}[X]$ pero considerando que los coeficientes de los
polinomios deben ser congruentes en módulo $n$ (vale decir, para cada
$i \in \{0, \ldots, k\}$, se debe tener que $a_i \equiv b_i \modl n$).

%Un último concepto que debemos introducir antes de mostrar el test de
%primalidad AKS es la equivalencia de dos polonimos en módulo otro
%polinomio.
Dados $p(X), q(X), r(X) \in \mathbb{Z}[X]$ con $r(X)$ un
polinomio no nulo, decimos que $p(X)$ es congruente a $q(X)$ en módulo
$r(X)$, denotado como $p(X) \equiv q(X) \modl r(X)$, si $r(X)$ divide
a $p(X) - q(X)$, vale decir, si existe un polinomio $s(X) \in
\mathbb{Z}[X]$ tal que $r(X) \cdot s(X) = p(X) - q(X)$. De manera
análoga, dado $n \geq 1$ y $p(X), q(X), r(X) \in \mathbb{Z}_n[X]$ con
$r(X)$ un polinomio no nulo en $\mathbb{Z}_n[X]$, decimos que $p(X)$
es congruente a $q(X)$ en módulo $r(X)$ en $\mathbb{Z}_n[X]$, denotado
como $p(X) \equiv q(X) \modl (r(X),n)$, si $r(X)$ divide a $p(X) -
q(X)$ en $\mathbb{Z}_n[X]$, vale decir, si existe un polinomio $s(X)
\in \mathbb{Z}_n[X]$ tal que
\begin{align*}
  r(X) \cdot s(X) \equiv p(X) - q(X) \modl n.
\end{align*}
Un último concepto que debemos introducir antes de mostrar el test de
primalidad AKS es el máximo común divisor de dos polinomios. Para esto
vamos a considerar los polinomios en $\mathbb{Z}_p[X]$, donde $p$ es
un número primo. Un polinomio $d(X) \in \mathbb{Z}_p[X]$ se dice
mónico si $d(X) = X^k + c_{k-1} X^{k-1} + \cdots + c_1 X + c_0$. Dados
dos polinomios $p(X), q(X) \in \mathbb{Z}_p[X]$ tales que ambos no son
nulos, $\MCD(p(X),q(X))$ en $\mathbb{Z}_p[X]$ es un polinomio $r(X)$
que satisface las siguientes propiedades:
\begin{itemize}
\item $r(X)$ es mónico,
  
\item $r(X)$ divide a $p(X)$ y divide a $q(X)$ en $\mathbb{Z}_p[X]$, y

\item para cualquier polinomio $s(X)$ tal que $s(X)$ divide a
  $p(X)$ y divide a $q(X)$ en $\mathbb{Z}_p[X]$, se tiene que $s(X)$
  divide a $r(X)$ en $\mathbb{Z}_p[X]$.
\end{itemize}
Nótese que la condición de que $r(X)$ es mónico es utilizada para
asegurar la unicidad en la definición de $\MCD(p(X),q(X))$ en
$\mathbb{Z}_p[X]$ (vale decir, que exista un único polinomio $r(X)$
que satisface las propiedades anteriores). Finalmente, $p(X)$ y $q(X)$
son coprimos en $\mathbb{Z}_p[X]$ si $\MCD(p(X),q(X)) = 1$ en
$\mathbb{Z}_p[X]$.

%\comentarioin{definir el $O_n(r)$ y argumentar por que es primo relativo (inverso), definir funcion de euler, notacion de módulo}
%\comentarioin{Marcelo: tambien definir
%  $\mathbb{Z}_n^*$,
%  $\mathbb{Z}_p[X]$,  $\mathbb{Z}[X]$, la notación $\mod (p(X),n)$, explicar qué significa que $p(X) = q(X)$ en módulo $n$ e introducir la notación $p(X) \equiv q(X) \mods n$ (nótese que esto es un caso particular de la notación $p(X) \equiv q(X) \mods r(X)$ cuando $r(X)$ es un polinomio de grado 0), $\grado(p(X))$}
%\comentarioin{Marcelo: tenemos que decidir dónde vamos a introducir la notación sobre polinomios irreducibles y cuerpos, en particular la notación $\mathbb{Z}_p[X]/h(X)$}
%\comentarioin{Definir la cardinalidad y caracteristica de un cuerpo como el del comentario anterior}
%\comentarioin{definir caracteristica de cuerpo}
%\comentarioin{Me parece algo raro mostrar el resultado extra del lema 3.1 y no mostrar la demostracion del lema aca en el documento. Quizas podriamos mostrar el lema y el corolario, pero ambas demostraciones ponerlas en el apendice.}

\subsection{Un resultado fundamental usado en el test de primalidad AKS}
Concluimos esta sección presentando un resultado que será utilizado en
la demostración de la correctitud del algoritmo.
	\begin{lemma}[\cite{N82}]\label{lem-mcm}
		Sea $\MCM(n)$ el mí­nimo común múltiplo de los primeros $n$ números naturales positivos. Para $n \geq 7$ se cumple:
		\begin{eqnarray*}
			\MCM(n) & \geq & 2^n
		\end{eqnarray*}
	\end{lemma}
	La demostración de este lema es dada en el apéndice \ref{app-lem-mcm}. Como consecuencia de este lema se obtiene una cota inferior para la cantidad de números primos menores o iguales a un número $n$ dado. De hecho, en \cite{N82} se demuestra el lema \ref{lem-mcm} para establecer el corolario \ref{mm-mcm}. Aunque este no será utilizado en el resto del documento, damos su demostración por su simpleza y la importancia del resultado. 
	\begin{corollary}\label{mm-mcm}
	Sea $\pi(n)$ el número de primos menores o iguales a $n$. Para todo $n\geq 3$, se cumple que 
	\begin{eqnarray*}
		\pi(n) & \geq &\frac{n}{\log_2(n)}
	\end{eqnarray*}	
	\end{corollary}
	\begin{proof}
	Sea $n \geq 7$, de manera tal que $\MCM(n) \geq 2^n$ por el lema \ref{lem-mcm}.
	%Sabemos por el Lema \ref{lem-mcm} se cumple para cualquier $m\geq 7$. 
	Dado un primo $p \in \{2, \ldots, n\}$, sea $\alpha_p$ el mayor número natural tal que $p^{\alpha_p} \leq n$. Por definición de $\MCM(n)$, tenemos que
	\begin{eqnarray*}
		\MCM(n) & = & \prod_{p \in \{2, \ldots, n\} \,:\, p \text{ es primo}} p^{\alpha_p}
	\end{eqnarray*}
	Dado que $p^{\alpha_p} \leq n$ para todo primo $p \in \{2, \ldots, n\}$,
        %(ya que $p^{\log_p n}= n$),
        concluimos que
\begin{eqnarray*}
		\MCM(n) & \leq & \prod_{p \in \{2, \ldots, n\} \,:\, p \text{ es primo}} n.
                %p^{\log_p(n)} \ = \ \prod_{p \in \{2, \ldots, n\} \,:\, p \text{ es primo}} p^{\frac{\log_2(n)}{\log_2(p)}}
	\end{eqnarray*}
	Aplicando la función $\log_2(\cdot)$ en ambos lados de la desigualdad obtenemos:
	\begin{eqnarray*}
		\log_2(\MCM(n)) & \leq & \log_2\bigg(\prod_{p \in \{2, \ldots, n\} \,:\, p \text{ es primo}} n\bigg)\\
%		& = & \sum_{p \in \{2, \ldots, n\} \,:\, p \text{ es primo}} \log_2(p^{\frac{\log_2(n)}{\log_2(p)}})\\
%		& = & \sum_{p \in \{2, \ldots, n\} \,:\, p \text{ es primo}} \frac{\log_2(n)}{\log_2(p)}\log_2(p)	\\
		& = & \sum_{p \in \{2, \ldots, n\} \,:\, p \text{ es primo}} \log_2(n)\\
		& = & \log_2(n) \cdot \pi(n)
		\end{eqnarray*}
Dado que 	$\MCM(n) \geq 2^n$, tenemos que $\log_2(\MCM(n)) \geq n$. Así, dado que $\log_2(n) \cdot \pi(n) \geq \log_2(\MCM(n))$, obtenemos que
	\begin{eqnarray*}
		\pi(n) & \geq &\frac{n}{\log_2(n)}
	\end{eqnarray*}	
	Para concluir la demostración del lema basta verificar manualmente que la propiedad también se cumple para los valores 3, 4, 5 y 6. Dejamos esta parte de la demostración al lector.
	\end{proof}
	
    


    
\section{El lema central}
\label{sec-lema-central}
        Como es mostrado en la siguiente sección, el test de
        primalidad AKS está basado en el siguiente lema que
        caracteriza a los números primos.
	\begin{lemma}\label{lem-2.1}
		Sea $a\in \mathbb{Z}$ y $n \in \mathbb{N}$, tales que $n\geq 2$ y $\MCD(a,n)=1$. Entonces
		\begin{eqnarray*}
			n \text{ es primo} & \Leftrightarrow & (X+a)^n \equiv X^n+a \modl n
		\end{eqnarray*}
	\end{lemma}
	\begin{proof}
		Antes de demostrar ambas direcciones, vamos a desarrollar el polinomio $P(X)=(X+a)^n - (X^n+a)$. 
		%desarrollaremos un polinomio que será fundamental para. Sea $P(X)=(X+a)^n - (X^n+a)$. 
Por el teorema del binomio, tenemos que $(X+a)^n=\sum_{i=0}^{n}{n\choose i}X^i a^{n-i}$. Luego,
		\begin{eqnarray}
			P(X) &=&  \bigg(\sum_{i=0}^{n}{n\choose i}X^i a^{n-i}\bigg)-X^n -a \nonumber\\ 
			&=&  \bigg(\sum_{i=1}^{n-1}{n\choose i}X^i a^{n-i}\bigg) + X^n + a^n -X^n -a\nonumber\\
			&=&  \sum_{i=1}^{n-1}{n\choose i}X^i a^{n-i} + (a^n -a) \label{eq:one}
		\end{eqnarray}		 
Con esta propiedad podemos demostrar ambas direcciones del lema. 
\begin{itemize}
\item[($\Rightarrow$)]
%		\subsubsection*{($\Rightarrow$)}
			Supongamos que $n$ es primo. Ya que $\MCD(a,n)=1$, el pequeño teorema de Fermat nos dice que $a^{n-1}\equiv 1  \modl n$. Luego, $a^n\equiv a \modl n$. Por lo tanto, de $\eqref{eq:one}$ nos queda que
			\begin{eqnarray*}
				P(X) &\equiv  & \sum_{i=1}^{n-1}{n\choose i}X^i a^{n-i} \modl n. %\label{eq:one}
			\end{eqnarray*} 
			Además, 
			%como $n$ es primo, 
			para cada $i \in \{1, \ldots, n-1\}$ tenemos que
			\begin{eqnarray*}
				{n\choose i} \ = \ \frac{n!}{(n-i)!i!} \ = \ n\cdot \frac{(n-1)!}{(n-i)!i!} \ = \ n\cdot c_1, 
				%\equiv 0 \modl n
			\end{eqnarray*}
			donde $c_1 \in \mathbb{N}$. 
			Nótese que esto se cumple ya que $c_1=\frac{(n-1)!}{(n-i)!i!}$, $i \in \{1, \ldots, n-1\}$ y $n$ es primo, por lo que para cada divisor $d$ de $(n-i)! i!$ tal que $d \neq 1$, se tiene que $d \nodiv n$ ya que $d < n$. 
			 %los factores de $(n-i)!$ y $i!$ son menores que $n$, entonces $(n-i)!i! \nodiv n $, por lo tanto $c_1 \in \mathbb{N}$. Así­ nos queda
			 Por lo tanto, tenemos que ${n\choose i} \equiv 0 \modl n$ para cada $i \in \{1, \ldots, n-1\}$, y así nos queda
			\begin{eqnarray*}
				P(X) \ \equiv \ \sum_{i=1}^{n-1}{n\choose i}X^i a^{n-i} \ \equiv \ 0 \modl n
			\end{eqnarray*}
			Por lo tanto, se cumple que $(X+a)^n \equiv X^n+a \modl n$.
		
%		\subsubsection*{($\Leftarrow$)}
\item[($\Leftarrow$)]
			Esta dirección se demostrará considerando el contrapositivo. Es decir, mostraremos que si $n$ no es primo, entonces $ (X+a)^n \not\equiv X^n+a$ en módulo $n$.
			
			Como $n$ es compuesto, debe existir un primo $q$ %tal 
			que %$q$ 
			divide a $n$.
			Sea $q^k$ la potencia máxima de $q$ %tal 
			que %$q^k $ 
			divide a $n$, vale decir, existe un entero $ c $ tal que $n=cq^k$ y $q$ no divide a $c$. A continuación, vamos a demostrar que $q^k\nodiv{n \choose q}$. Para %ver que
			esto %se cumple, 
			considere que
			\begin{eqnarray}
				\nonumber {n \choose q} &=& \frac{n!}{(n-q)!q!}\\
				&=& \nonumber \frac{n(n-1)\cdot\cdot\cdot(n-q+1)(n-q+1)}{q!}\\
				&=&  \nonumber \frac{c q^k(n-1)\cdot\cdot\cdot(n-q+1)}{q(q-1)!}\\
				&=&\nonumber  \frac{c q^{k-1}(n-1)\cdot\cdot\cdot(n-q+1)}{(q-1)!}\\
				%\hspace{0.5cm}\text{y dado que $q$ es primo y ${n\choose q}$ es entero,}\\
				&=& q^{k-1}\cdot \frac{c (n-1)\cdot\cdot\cdot (n-q+1)}{(q-1)!} \label{eq:lem-left}
			\end{eqnarray}	
			%Dado lo anterior, como 
			Además, dado que $q$ divide a $n$, el 
			% siguiente número $r<n$ tal que $q$ divide a $r$ 
			mayor múltiplo de $q$ menor a $n$ es $n-q$. 
			%Esto se puede demostrar fácilmente por contradicción. Supongamos que existe un número $ \alpha $ en los naturales con $1\leq \alpha <q$ tal que $q$ divide a $(n-\alpha)$. Como $q$ divide a $n$, entonces necesariamente $q$ divide a la resta $n-(n-\alpha)$. De esto podemos afirmar que $q$ divide a $\alpha$,
			%\comentario{esta bien?}, 
			%lo cual lleva a una contradicción porque $1\leq \alpha <q$. 
			Entonces, $q$ no divide a ningún número $r\in\{n-q+1,n-q+2,...,n-1\}$, y como $q$ es primo, tampoco divide a la multiplicación de todos ellos, es decir, $q\nodiv (n-1)\cdot\cdot\cdot(n-q+1)$ (si $q$ no fuera un número primo, podría pasar que dividiera la multiplicación). Además, $q\nodiv c$ por como definimos $c$. De esto, de la igualdad \eqref{eq:lem-left} y el hecho que $q$ es primo, podemos concluir que $q^k\nodiv {n\choose q}$.
			
			Con lo anterior, y sabiendo que $a^{n-q}$ y $q^k$ son coprimos (ya que $\MCD(a,n) = 1$ y $q^k\divi  n$), sabemos que $q^k\nodiv {n\choose q} a^{n-q}$. Además, tenemos que
			\begin{eqnarray*}
				q^k\nodiv {n\choose q} a^{n-q}
				& \Rightarrow & c\cdot q^k\nodiv {n\choose q} a^{n-q}\\
				&  \Leftrightarrow &n\nodiv {n\choose q} a^{n-q}\\
				&  \Leftrightarrow& {n \choose q}a^{n-q} \not \equiv 0 \modl n
			\end{eqnarray*} 
			Hemos demostrado entonces que el coeficiente de $X^q$ en $P(X)$ no es idénticamente 0 en módulo $n$.
%en el puesto número $q$ de $\sum_{i=1}^{n-1}{n\choose q}X^q a^{n-q}\not \equiv 0 \modl n$. 
Luego, $P(X)$ no es idénticamente 0 en módulo $n$, y llegamos a que $(X+a)^n \not \equiv X^n+a \modl n$. Esto concluye la demostración del lema.		%Con esto terminamos ambas direcciones de la demostración del lema.
\end{itemize}
%\comentario{cambiar los nmid por not |}
	\end{proof}
	

	

	
\section{El algoritmo y su correctitud}
\label{sec-algoritmo-correctitud}
En el algoritmo \ref{alg-aks} presentamos el algoritmo propuesto en \cite{AKS04}. 
A continuación, se demostrará que este algoritmo es correcto y funciona en tiempo polinomial. Nótese que desde ahora en adelante usaremos el término $\poly(x_1, \ldots, x_k)$ para denotar un  polinomio arbitrario en las variables $x_1$, $\ldots$, $x_k$. También utilizaremos esta notación con sub-índices para referirnos a polinomios distintos, por ejemplo $\poly_1(x,y)$ y $\poly_2(z)$. 
%La demostración consiste en una serie de lemas que permitirán, en conjunto, probar que, dado un número natural $n > 1$, el algoritmo retorna $\PRIMO$ cuando $n$ es primo, y $\COMPUESTO$ en el caso contrario.


\begin{algorithm}[h]
\caption{\quad Test de primalidad \AKS}
\label{alg-aks}
\hspace*{\algorithmicindent} \textbf{Input:} un número natural $n>1$\\
\hspace*{\algorithmicindent} \textbf{Output:} $\PRIMO$ si $n$ es primo, $\COMPUESTO$ si $n$ no es primo
\begin{algorithmic}[1]
   \IF {$n=a^b$ para $a,b\in \mathbb{N}$ y $b>1$}\label{alg-1} 
      \RETURN \COMPUESTO \label{alg-r1}
   \ELSE
      \STATE encontrar el número natural $r > 1$ más pequeño tal que $O_r(n)>\log ^2n$ \label{alg-2}
      \IF {$1<\MCD(a,n)<n$ para algún $a \in \{2, \ldots, r\}$} \label{alg-3}
        		\RETURN \COMPUESTO \label{alg-r3}
      \ELSE
         \IF {$n\leq r$} \label{alg-4}
         \RETURN \PRIMO \label{alg-r4}
         \ELSE
            \FOR {$a=1$ \TO $\lfloor \sqrt{\phi(r)} \log n\rfloor$} \label{alg-5-for}
            	\IF {$(X+a)^n \not\equiv X^n+a \modulo$} \label{alg-5}
	           \RETURN \COMPUESTO \label{alg-r5}
	        \ENDIF
            \ENDFOR
            \RETURN \PRIMO \label{alg-6}
         \ENDIF
      \ENDIF      
   \ENDIF
\end{algorithmic}
\end{algorithm}
	
%	\begin{algorithmic}
%		%\setstretch{1.25}
%		%\caption{Test de primalidad} 
%		%Input: Un entero $n>1$.
%		%\begin{enumerate}
%			%\label{alg-1} 
%			\IF {$(n=a^b$ para $a\in \mathbb{N}$ y $b>1)$} 
%				\STATE \RETURN \COMPUESTO
%%			\item \label{alg-2} Encontrar el $r$ más pequeño tal que $O_r(n)>\log ^2n$.
%%			\item \label{alg-3} If $1<\MCD(a,n)<n$ para algún $a\leq r$, return COMPUESTO.
%%			\item \label{alg-4} If $n\leq r$, return PRIMO.
%%			\item \label{alg-5} For $a=1$ to $\lfloor \sqrt{\phi(r)} \log n\rfloor$ hacer:
%%			
%%		\hspace*{1cm}If ($(X+a)^n \neq X^n+a \modulo$), return COMPUESTO.
%%		\item \label{alg-6} return PRIMO.
%%		\end{enumerate}
%	\end{algorithmic}
	%\comentario{hacer nuevo comando PRIMO usando rm primo}
	
	\begin{theorem} \label{teo-4.1}
		Dado un número natural $n > 1$, el algoritmo \ref{alg-aks} con entrada $n$ retorna $\PRIMO$ 
		si, y solo si, $n$ es un número primo. Además, el algoritmo \ref{alg-aks} funciona en tiempo $O(\poly(\log n))$, vale decir, en tiempo polinomial en el largo de la entrada $n$. 
	\end{theorem}
	La demostración del teorema se divide en dos partes. En primer lugar vamos a mostrar que el algoritmo retorna $\PRIMO$ o $\COMPUESTO$ para cualquier entrada en tiempo polinomial, es decir, el algoritmo termina, y en segundo lugar, veremos que retorna $\PRIMO$ cuando la entrada es un número primo, y $\COMPUESTO$ cuando la entrada es un número compuesto. 
	\begin{proposition}\label{prop-lem-1}
		El algoritmo termina y retorna $\PRIMO$ o $\COMPUESTO$ para cualquier entrada entera $n>1$. Además, el algoritmo funciona en tiempo $O(\poly(\log n))$.
				%Además, el algoritmo demora tiempo polinomial en retornar en el peor caso, respecto al largo de la entrada.
	\end{proposition}
	La demostración de la proposición anterior consiste principalmente en mostrar la existencia del número $r$ que buscamos en el paso \ref{alg-2} del algoritmo, para cualquier entrada $n$, ya que en todos los otros pasos se hace una cantidad finita de operaciones. Además de mostrar la existencia del número $r$, mostraremos que podemos encontrarlo en tiempo polinomial (con respecto al largo de la entrada), lo que, junto con un desarrollo adicional, nos permitirá afirmar que el algoritmo termina en tiempo polinomial en cualquier caso. 
	\begin{lemma}\label{prop-1}
		Para cada $n > 1$, existe $r \leq \max\{3,\lceil \log ^5 n\rceil\}$ tal que $O_r(n)>\log ^2 n$.
	\end{lemma}
	\begin{proof}
		Lo primero que podemos notar es que cuando $n=2$, se tiene que $\log^2 2 =1$. Si calculamos los órdenes multiplicativos de 2 cuando $r=1$ y $r=3$:
		%$r$ toma los valores 1 y 3:
		\begin{eqnarray*}
			&&O_1(2) \ = \ 1 \ \not> \ \log ^2 2 \ =\ 1\\
			&&O_3(2) \ = \ 2  \ > \ \log ^2 2 \ =\ 1,
		\end{eqnarray*}
		podemos ver que si escogemos $r=3$ entonces se cumple el lema para $n=2$. De forma análoga, podemos concluir que si $n = 3$, el lema se cumple para $r = 5$, y si $n=4$, entonces el lema se cumple para $r= 11$.
		%\comentario{definicion de orden multiplicativo/puse en minuscula el podemos}
		Ahora supongamos que $n \geq 5$. Sea $B = \lceil \log^5 n \rceil$. Como la función logarítmo es creciente, podemos observar que $B \geq \lceil \log ^5 5\rceil = 68 > 7$. Esto nos permite usar el lema \ref{lem-mcm} con $n = B\geq 7$, lo cual nos dice que $\MCM(B) \geq 2^{B}$.
		Sea
\begin{eqnarray*}
		N &=& n^{\lfloor\log B\rfloor} \cdot\prod_{i=1}^{\lfloor \log ^2n \rfloor} (n^i-1)
\end{eqnarray*}
%\comentarioin{Aca hay que referenciar el resultado de math stackexchange ya que el paper original tenia un error en esta parte}
A continuación la demostración se dividirá en tres partes.\footnote{La
  demostración del lema \ref{prop-1} en \cite{AKS04} contiene un
  error, como es mencionado por los mismos autores en una versión
  posterior de este artículo (ver \url{https://www.cse.iitk.ac.in/users/manindra/algebra/primality_v6.pdf}). La demostración corregida presentada aquí está basada en \url{https://math.stackexchange.com/questions/119573/on-proof-of-aks-primality-test-algorithm}.}
Primero, mostrar la existencia de un número natural
 %\comentario{naturales? en plural?} 
 $r\leq B$ tal que $r$ no divide a $N$,  segundo, probar que podemos escoger un valor $r$ del paso anterior tal que $O_r(n)$ esté bien definido, y por último, mostrar que $O_r(n)>\log ^2 n$. Notemos que si encontramos un $r$ con estas características, habremos demostrado el lema \ref{prop-1}.

Primero, demostraremos que existe un valor $r\leq B$ que no divide a $N$. Supongamos por contradicción que para todo $1\leq r\leq B$, $r$ divide a $N$. De esto podemos decir que $N$ es un múltiplo común de los primeros $B$ naturales positivos, por lo que $\MCM(B)\leq N$ por definición de mínimo común múltiplo. Notemos que para $n\geq 5$,
\begin{eqnarray*}
	N&=& n^{\lfloor \log B\rfloor}\cdot\prod_{i=1}^{ \lfloor\log ^2n\rfloor} (n^i-1)\\
	&<& n^{\lfloor \log B\rfloor}\cdot\prod_{i=1}^{\lfloor\log ^2n\rfloor} n^i\\
	&=& n^{\lfloor \log B\rfloor + 1+2+\cdots + \lfloor\log ^2n\rfloor}\\
	&=& n^{\lfloor \log B\rfloor + \frac{\lfloor\log ^2 n\rfloor(1+\lfloor\log ^2 n\rfloor)}{2}}\\
	&\leq & n^{\lfloor \log B\rfloor +\frac{(\log ^2 n)(1+\log ^2 n)}{2}}\\ 
	& = & n^{\frac{2\lfloor \log B\rfloor +\log ^2 n + \log ^4 n}{2}}\\
	&\leq & n^{\frac{\log^4 n + \log^4 n}{2}} \quad\quad\text{(por Apéndice \ref{app-prop-1})}\\
	&=& n^{\log^4 n}\\
	&=& (2^{\log n})^{\log^4 n}\\
	&=& 2^{\log^5 n}\\
	&\leq& 2^{\lceil\log^5 n\rceil } \ = \ 2^{B}
\end{eqnarray*}
%comentario{por apendice arreglar en eqnarray (ya agregue al apendice la demostracion de la cota)}
De lo anterior, concluimos que $N<2^{B}\leq \MCM(B)$ (por el lema \ref{lem-mcm}), lo cual nos lleva a una contradicción, ya que sabíamos que $\MCM(B)\leq N$. De esta forma, hemos demostrado que existe un valor $r\leq B$ que no divide a $N$.

Sea $r$ el menor natural positivo que cumple con lo anterior, es decir, $r \nodiv N$ y para todo valor $r'\in\{1,...,r-1\}$ se cumple que $r'\divi N$.
% tal que $1\leq r'\leq B$ y $r' \nodiv N$, se cumple que $r \leq r'$.
% que no divide a $N$, se cumple que $r\leq r'$. 
 Antes de comenzar la segunda parte de la demostración, observemos que si $m\geq 2$ y $m^k\leq B$, tenemos que 
\begin{eqnarray*}
m^k \ \leq \ B &\Rightarrow& k\log m \ \leq \ \log B\\
	&\Rightarrow& k\ \leq \ \frac{\log B}{\log m}
	\ \leq \ \frac{\log B}{\log 2} \ = \ \log B
\end{eqnarray*}
De lo anterior podemos ver que el entero $k$ más grande que cumple con estas restricciones es $k = \lfloor \log B\rfloor$ (para cualquier $m\geq 2$ que cumpla con $m^k\leq B$). Esta observación la utilizaremos a continuación.

%Con el objetivo de demostrar que $r$ y $n$ son coprimos, 
Nuestro objetivo ahora es demostrar que $\MCD(r,n) =1 $, ya que esto implicaría que $O_r(n)$ está bien definido. 
Para esto comenzaremos observando que $\MCD(r,n)$ no es divisible por todos los primos que dividen a $r$. Dicho de otra forma, existen factores primos de $r$ que no son factores primos de $n$. Esto puede ser demostrado como sigue. Sea $r=q_1^{e_1}\cdots q_{\ell}^{e_{\ell}}$ la factorización prima de $r$.
Para todo $i\in [1, \ell]$, se cumple que $1<q_i^{e_i}\leq r\leq B$, y por la observación anterior se cumple que $e_i\leq \lfloor\log B\rfloor$ (aquí usamos $k=e_i$). De esto, podemos concluir que $q_i^{e_i}$ divide a $q_i^{\lfloor\log B\rfloor}$.
 Supongamos por contradicción que para todo $i\in [1, \ell]$, se cumple que $q_i$ divide a $\MCD(r,n)$. Como $q_i$ es primo, entonces también debe ser un factor primo de $n$. Luego, podemos afirmar que $n=\alpha\cdot q_1\cdots q_{\ell}$, donde $\alpha$ es un entero positivo. Elevando ambos lados de la igualdad por $\lfloor\log B\rfloor$, tenemos que $n^{\lfloor\log B\rfloor}=\alpha^{\lfloor\log B\rfloor}\cdot q_1^{\lfloor\log B\rfloor}\cdots q_{\ell}^{\lfloor\log B\rfloor}$, y como $q_i^{e_i}$ divide a $q_i^{\lfloor\log B\rfloor}$ para todo $i\in [1, \ell]$, concluimos que $r$ divide a $n^{\lfloor\log B\rfloor}$. Tenemos entonces que $r \divi N$ por definición de $N$, 
 %y por consiguiente a $N$, 
 lo cual es una contradicción ya que sabemos que $r$ no divide a $N$. Es por esto que no todos los factores primos de $r$ dividen a $\MCD(r,n)$. De esta forma, podemos escribir la factorización prima de $r$ como $r = a\cdot b$, donde
 \begin{eqnarray*}
 	a&=& p_1^{\alpha_1}\cdots p_k^{\alpha_k},\hspace{0.5cm}p_i\text{ es primo y $p_i$ divide a }\MCD(r,n)\\
 	b&=& s_1^{\beta_1}\cdots s_k^{\beta_{k'}},\hspace{0.5cm}s_i\text{ es primo y $s_i$ no divide a }\MCD(r,n)
 \end{eqnarray*}
 %y $r = p_1^{\alpha_1}\cdots p_k^{\alpha_k} \cdot  s_1^{\beta_1}\cdots s_k^{\beta_{k'}}$ es la descomposición prima de $r$.
Lo primero que podemos observar es que necesariamente $a$ divide a $n^{\lfloor \log B\rfloor}$, ya que de la misma forma que antes, se debe cumplir que $\alpha_i\leq \lfloor \log B\rfloor$ (ya que $p_i^{\alpha_i}\leq B$) para todo $i \in \{1, \ldots, k\}$, y como $p_i$ es también un factor primo de $n$, entonces $p_i^{\alpha_i}$ debe dividir a $n^{\lfloor \log B\rfloor}$. Por otro lado, tenemos que $\MCD(b,n) = 1$, ya que suponiendo que no fuera así, existiría un primo $s_i$ que divide a $b$ y a $n$, y por ende divide a $\MCD(r,n)$ ya que $r=ab$. Sin embargo, esto no puede ser cierto por como hemos construido $b$. Lo anterior nos dice que $b$ y $n$ no tienen factores primos en común, por lo que también se cumplirá que $\MCD(b,n^{\lfloor \log B\rfloor}) = 1$. Por último, como sabemos que $a$ divide a $n^{\lfloor \log B\rfloor}$ y $\MCD(b,n^{\lfloor \log B\rfloor})=1$, podemos afirmar que $b$ no divide a $\prod_{i=1}^{\lfloor \log ^2n \rfloor} (n^i-1)$, ya que de lo contrario $r = ab$ dividiría a $N$, lo cual contradice nuestra elección del valor $r$. De todo esto podemos concluir que $b$ no divide a $N$, y como $b\leq r$ y $r$ es el menor número que no divide a $N$, entonces se debe cumplir que $r = b$. Luego,  como $\MCD(b,n)=1$, se tiene que $\MCD(r,n) = 1$ y $O_r(n)$ está bien definido (este número existe).

Por último, supongamos por contradicción que para este mismo valor $r$ se tiene que $O_r(n) \leq \log^2 n$. Dado que $O_r(n)$ es un número entero, concluimos que $O_r(n) \leq \lfloor\log^2 n\rfloor$.
Desarrollando entonces a partir de la definición de orden multiplicativo tenemos
\begin{eqnarray*}
	n^{O_r(n)} \ \equiv \ 1 \modl r & \Leftrightarrow & n^{O_r(n)} - 1 \ \equiv \ 0 \modl r \\
	& \Leftrightarrow & r\ \divi \ n^{O_r(n)} - 1\\
	&\Rightarrow & r \ \divi \ \prod_{i=1}^{ \lfloor\log ^2n\rfloor} (n^i-1) \quad\quad \text{puesto que }O_r(n) \leq \lfloor \log^2 n \rfloor\\	
	&\Rightarrow & r\ \divi \ N 
\end{eqnarray*}
Lo anterior es una contradicción ya que escogimos al valor $r$ de tal manera que no dividiera a $N$. Con esto podemos afirmar que necesariamente $O_r(n) > \log^2 n$. Esto concluye la demostración del lema. 
	\end{proof}
	%\comentario{Marcelo: llegué hasta este punto}
	Con la información que obtuvimos a partir del lema anterior, nos gustaría demostrar la proposición~\ref{prop-lem-1}. Para esto, mostraremos que en cada paso, el algoritmo demora tiempo polinomial respecto al largo de la entrada.
	\begin{itemize}
	\item En el paso \ref{alg-1}, el algoritmo debe verificar si $n$ es potencia (no trivial) de otro número natural, lo cual puede ser hecho en tiempo $O(\poly_1(\log n))$, vale decir, en tiempo polinomial respecto al largo de $n$ (ver algoritmo \ref{alg:es_potencia} 
	%(para más detalles del análisis de complejidad ir a
	en el apéndice \ref{app-es_potencia}).
	
	\item De acuerdo al lema \ref{prop-1}, el algoritmo debe calcular en el paso \ref{alg-2} los valores $O_r(n)$ para cada $r \in \{2, \ldots, \max\{3,\lceil \log^5 n\rceil\}$. Notemos que para calcular $O_r(n)$ podemos utilizar un algoritmo  que tiene complejidad $O(\poly_2(\log n, r))$ (ver algoritmo \ref{alg:mult_ord} en el apéndice \ref{app-orden_multiplicativo}).
	%en el cual supusimos que las operaciones aritméticas y de comparación son operaciones de costo 1.\comentario{Sin embargo, en la vida real son $\log n$ o $r$?....}  
	De lo anterior y usando el lema \ref{prop-1}, concluimos que calcular el valor $r$ pedido en el paso \ref{alg-2} del algoritmo toma tiempo $O(\lceil \log^5 n\rceil \cdot \poly_2(\log n, \lceil \log^5 n\rceil))$, vale decir, toma tiempo $O(\poly_3(\log n))$. 
%	$O_r(n)$ toma tiempo $O(\log n +\log ^5 n) = O(\log^5 n)$. 
	Es por esto que esta línea del algoritmo también demora una cantidad polinomial de pasos en relación al largo de la entrada $n$.
	%\comentario{asi esta bien? o falta un poco de explicacion}
	%\comentarioin{Marcelo: fuera del apéndice no es correcto decir que calcular $O_r(n)$ toma tiempo $O(\log(\max\{n,r\}) + r)$ (está fuera del ámbito de los supuestos del apéndice). Sería mejor decir $O((\log(\max\{n,r\}) + r) \cdot p(\log n, r))$, donde $p(\log n, r)$ es un polinomio que depende de $\log n$ y $r$.}
%\comentarioin{Entonces en todos los algoritmos debemos decir que es $O((\textbf{complejidad del apendice})*p(\log n,r))$ para ser mas consistentes?\\Por qué sería $\log n$ y $r$ en vez de $\log n$ y $\log r$? Ahi tambien tenemos que argumentar: todas las operaciones a un elemento de largo $x$ se hacen en $p(\log x)$?}
	
	\item En el paso \ref{alg-3} del algoritmo calculamos el máximo común divisor entre $n$ y $a\leq r$, un total de $r$ veces en el peor de los casos. El máximo común divisor entre $n$ y $a$ puede ser calculado por un algoritmo que funciona en tiempo $O(\max\{\log n, \log a\})$ (ver algoritmo \ref{alg:mcd} en la sección \ref{app-mcd} del apéndice). De esta forma, el paso \ref{alg-3} del algoritmo toma tiempo $O(r \cdot \max\{\log n, \log r\})$. Dado que $r\leq \max\{3,\lceil \log ^5 n\rceil\}$, concluimos que el paso \ref{alg-3} del algoritmo toma tiempo $O(\lceil \log ^5 n\rceil \cdot \max\{\log n, \log \lceil \log ^5 n\rceil\})$, vale decir, tiempo $O(\poly_4(\log n))$ como era esperado. 
%Esto lo podemos hacer con el algoritmo \ref{alg:mcd} realizando un total de $O(r\cdot \log n)$ pasos. Como $r\leq Max\{3,\lceil \log ^5 n\rceil\}$, entonces esta línea termina en a lo más $O(\log^6 n)$ pasos. 
	
	%\item En el paso \ref{alg-4} hacemos verificamos si $n \leq r$,  lo cual claramente toma tiempo polinomial ya que $r\leq \max\{3,\lceil \log ^5 n\rceil\}$. 
	
	\item En los pasos \ref{alg-5-for} y \ref{alg-5} del algoritmo debemos verificar $\lfloor \sqrt{\phi(r)} \log n\rfloor$ veces si dos polinomios son distintos en módulo $(X^r-1,n)$. A primera vista, no sabemos si la cantidad de veces que se debe repetir la verificación en el paso \ref{alg-5}
	%si esta cantidad de operaciones 
	es polinomial con respecto al largo de $n$. Sin embargo, sabemos que $\alpha = O_r(n)$ es el menor valor tal que $n^{\alpha}\equiv 1$ en módulo $r$, y como $n^{\phi(r)}\equiv 1$ en módulo $r$, entonces necesariamente $O_r(n)\leq \phi(r)$. Además, sabemos que $\phi(r)\leq r$, por lo que tenemos que $O_r(n)\leq r$. Así, dado que $\log^2 n < O_r(n)$ por el paso \ref{alg-2} del algoritmo, concluimos que:
	% podemos concluir que, dado que $\log^2 n < O_r(n)$, entonces 
	\begin{eqnarray}\label{cota-ell}
		\lfloor \sqrt{\phi(r)} \log n\rfloor \ \leq \ \sqrt{\phi(r)} \log n
		\ \leq \ \sqrt{r}\sqrt{O_r(n)} \ \leq \ \sqrt{r}\sqrt{r}
		\ = \ r
	\end{eqnarray}
	%\comentario{(5)esto lo vamos a usar mas adelante y puedo referenciar la ecuacion}
	Luego, se tiene que $\lfloor \sqrt{\phi(r)} \log n\rfloor\leq r \leq  \max\{3,\lceil \log ^5 n\rceil\}$, de lo cual concluimos que el número de veces que debemos hacer la verificación de la equivalencia de polinomios en el paso \ref{alg-5} es $O(\poly_5(\log n))$. Además, para verificar si $(X+a)^n \not\equiv X^n+a \modulo$, podemos utilizar un algoritmo que calcula $(X+a)^n \!\! \modulo$ y luego comparar el resultado con el polinomio $X^n+a$. En el apéndice \ref{app-fast_exp_mod} presentamos un algoritmo que calcula $(X+a)^n \!\! \modulo$ en tiempo $O(\poly_6(\log n, \grado(X+a), r, \log n))$, donde $\grado(X+a) = 1$.
        %$\gradop(X)) = 1$ es el grado del polynomio $p(X)$ ($\grado(X+a)=1$).
        Dado que $r \leq  \max\{3,\lceil \log ^5 n\rceil\}$, concluimos que la condición $(X+a)^n \not\equiv X^n+a \modulo$ puede ser verificada en tiempo $O(\poly_6(\log n, 1, \lceil \log ^5 n\rceil, \log n))$, vale decir, en tiempo $O(\poly_7(\log n))$. Finalmente, combinando los resultados anteriores, concluimos que los pasos \ref{alg-5-for} y \ref{alg-5} del algoritmo pueden ser realizados en tiempo $O(\poly_5(\log n) \cdot \poly_7(\log n))$, vale decir, en tiempo polinomial en el largo de la entrada $n$.
%	 Además, debemos ser capaces de calcular $(X+a)^n$ en módulo $(X^r-1,n)$, lo cual lo podemos realizar mediante el algoritmo \ref{alg:fast_exp_mod} presentado en el apéndice \ref{app-fast_exp_mod} en $O(\log n \cdot r^2 \cdot deg(q))= O(\log n \cdot r^2)$. Luego, en esta etapa realizamos a los más $O(\log^5 n\cdot\log n \cdot r^2)\leq O(\log^6 n \cdot (\lceil \log^5n\rceil)^2)$, lo que es una cantidad polinomial de pasos respecto al largo de $n$. 
	\end{itemize}
		
%	En el paso \ref{alg-4} hacemos una comparación, para lo cual claramente nos demoramos tiempo polinomial, y en el paso \ref{alg-5} debemos decidir $\lfloor \sqrt{\phi(r)} \log n\rfloor$ veces si dos polinomios son distintos en módulo $(X^r-1,n)$. A primera vista, no sabemos si esta cantidad de operaciones es polinomial con respecto al largo de $n$. Sin embargo, sabemos que $\alpha = O_r(n)$ es el menor valor tal que $n^{\alpha}\equiv 1$ en módulo $r$, y como $n^{\phi(r)}\equiv 1$ en módulo $r$, entonces necesariamente $O_r(n)\leq \phi(r)$. Además, sabemos que $\phi(r)\leq r$, entonces $O_r(n)\leq r$. De todo esto podemos decir que, dado que $\log^2 n < O_r(n)$, entonces 
%	\begin{eqnarray}
%		\lfloor \sqrt{\phi(r)} \log n\rfloor &<&\sqrt{\phi(r)} \log n\nonumber\\
%		&\leq & \sqrt{r}\sqrt{r}\nonumber\\
%		&=& r\label{cota-ell}
%	\end{eqnarray}\comentario{(5)esto lo vamos a usar mas adelante y puedo referenciar la ecuacion}Luego $\lfloor \sqrt{\phi(r)} \log n\rfloor\leq \log^5 n$. Además, debemos ser capaces de calcular $(X+a)^n$ en módulo $(X^r-1,n)$, lo cual lo podemos realizar mediante el algoritmo \ref{alg:fast_exp_mod} presentado en el apéndice \ref{app-fast_exp_mod} en $O(\log n \cdot r^2 \cdot deg(q))= O(\log n \cdot r^2)$. Luego, en esta etapa realizamos a los más $O(\log^5 n\cdot\log n \cdot r^2)\leq O(\log^6 n \cdot (\lceil \log^5n\rceil)^2)$, lo que es una cantidad polinomial de pasos respecto al largo de $n$.  
	
%	Por último, en el paso \ref{alg-6} solamente retornamos, por lo que también es de orden polinomial.
	
	Si observamos el algoritmo, el peor caso es cuando retorna en el paso \ref{alg-6}, ya que pasamos por todos los otros pasos. Como vimos que el tiempo para ejecutar cada paso es polinomial con respecto al largo de la entrada $n$, entonces necesariamente la suma de los tiempos de cada paso también será polinomial. Con esto concluimos la demostración de la
        proposición~\ref{prop-lem-1},
        %lema \ref{lem-1},
        ya que 
	%afirmando que 
	el algoritmo siempre termina y funciona en tiempo~$O(\poly(\log n))$.
	%un tiempo polinomial con respecto al largo de la entrada.  
	
	Las siguientes proposiciones nos hablan de la correctitud del algoritmo. Para la demostración, nos fijaremos en cada uno de sus pasos, y veremos que lo que retorna es correcto.
	%  paso, entonces retorna el resultado correcto.
	% correctamente.\comentario{decir que lo mas dificil es el paso 6? Para que no lleguen al paso 5 preguntandose por que les falta la mitad del documento} 
	Supondremos para cada paso que la entrada es un entero $n>1$.
	
	\begin{proposition}
		Si el algoritmo \AKS\ retorna en el paso \ref{alg-r1}, entonces $n$ es compuesto. 
	\end{proposition}
	\begin{proof}
	Si el algoritmo retorna en este paso, entonces el algoritmo encontró números naturales $a$ y $b>1$ tales que $n=a^b$. Luego, $n$ es compuesto.
	%, y el algoritmo correctamente retorna 
	% Es por esto que el algoritmo no se equivocó al retornar 
	%COMPUESTO.
	\end{proof}
	
	\begin{proposition}
		Si el algoritmo \AKS\ retorna en el paso \ref{alg-r3}, entonces $n$ es compuesto. 
	\end{proposition}
	\begin{proof}
	Si el algoritmo retorna en el paso \ref{alg-r3}, entonces existe $a \leq r$ tal que $1<\MCD(a,n)<n$. A partir de esto se concluye que $n$ es un número compuesto, puesto que $\MCD(a,n) \divi n$. 
	%De esta forma, sabemos que el algoritmo correctamente retorna 
	%COMPUESTO en este paso.
	%Supongamos por contradicción que el algoritmo retornó en este paso, pero $n$ resulta ser primo. De esta forma, vamos a tener que para cualquier $a$ (en particular para $a\leq r$), se cumple que $\MCD(a,n)=1$ si $n$ no divide a $a$, y $\MCD(a,n)=n$ si $n$ divide a $a$. Así, el algoritmo no pudo haber retornado en esta lí­nea del código, ya que no existe un $a\leq r$ tal que $1<\MCD(a,n)<n$. Por lo tanto, nuestra suposición de que $n$ era primo no es válida, ya que nos lleva a una contradicción. De esto concluimos que $n$ debe ser compuesto.
	\end{proof}
	\begin{proposition}
	Si el algoritmo \AKS\ retorna en el paso \ref{alg-r4}, entonces $n$ es primo. 
	\end{proposition}
	\begin{proof}
	  Si el algoritmo retorna en este paso, entonces sabemos por su definición
          %construcción del algoritmo,
          que $n\leq r$. Si suponemos por contradicción que $n$ es compuesto, entonces existe un divisor $p$ de $n$ tal que $1 < p < n$. De esta forma, se tiene que $1 < \MCD(p,n) < n$ y $p < n \leq r$, por lo que el algoritmo habría retornado en el paso \ref{alg-r3}.
	%, ya que hubiera encontrado un divisor $p\leq n\leq r$ no trivial de $n$. Al no haber retornado en \ref{alg-r3}, tenemos una contradicción, y $n$ necesariamente debe ser primo. 
	Como esto contradice el supuesto de que el algoritmo retorna en el paso \ref{alg-r4}, no existe tal divisor $p$ y el número $n$ es primo. 
	%Concluimos 
	\end{proof}
	\begin{proposition}
		Si el algoritmo \AKS\ retorna en el paso \ref{alg-r5}, entonces $n$ es compuesto.
	\end{proposition}
	\begin{proof}
	En el paso \ref{alg-5}, el algoritmo revisa para $a\in \{1, \ldots, \lfloor \sqrt{\phi(r)} \log n\rfloor\}$ si $(X+a)^n \not\equiv X^n+a$ en módulo $(X^r-1,n)$. Si suponemos por contradicción que $n$ es primo, por el lema \ref{lem-2.1} sabemos que $(X+a)^n \equiv X^n+a$ en módulo $n$ para todo $a \in \mathbb{Z}$ (nótese que esta equivalencia se cumple trivialmente si $a\mods n = 0$), por lo que $(X+a)^n \equiv X^n+a$ en módulo $(X^r-1,n)$ para todo $a \in \mathbb{Z}$. De esta forma, si $n$ es primo, la condición en el paso \ref{alg-5} sería falsa para todo  $a\in \{1, \ldots, \lfloor \sqrt{\phi(r)} \log n\rfloor\}$, y el algoritmo no retorna COMPUESTO en el paso \ref{alg-r5}. Como esto contradice nuestro supuesto inicial, concluimos que $n$ es un número compuesto.
%	esta igualdad se cumple para cualquiera de estos $a$, y al hacer módulo ($X^r-1,n$) nos restringimos a todos los $a$ tales que $1\leq a< n$. Luego, no podría haber retornado en \ref{alg-r5}, ya que para hacerlo, algún $a$ debería no cumplir la igualdad. Por lo que $n$ debe ser compuesto.
	\end{proof}
	
	\begin{proposition}\label{prop:paso_final}
		Si el algoritmo \AKS\ retorna en el paso \ref{alg-6}, entonces $n$ es primo.
	\end{proposition}
	Solo nos falta ver la correctitud de este paso para ver la correctitud total del algoritmo.
	Sin embargo, para demostrar esta proposición necesitaremos de más trabajo y de un mayor número de herramientas más complejas que las ya consideradas.
	%con mayor alcance en comparación con 
	%las anteriores. 
	Es por esto que usaremos una serie de lemas
        %\comentarioin{Lemas, proposiciones ¿y definiciones?(si existe alguna proposicion)}
        para lograr demostrar la proposición \ref{prop:paso_final}.
	Con el fin de obtener una contradicción, en el resto de esta sección vamos a suponer 	
	%Para el resto de esta sección, vamos a suponer para obtener una contradicción 
	%(por contradicción), por el resto de esta sección, 
	que el algoritmo retorna primo en el paso \ref{alg-6}, pero $n$ es compuesto.	
%\comentario{Marcelo: hasta aquí revisé.}
%%%%%%%%%%%%%%%%%%%%%%%%%%%%%%%%%%%%%%% hasta aquí cambia el nuevo orden
	
	
	Siguiendo esta línea, lo primero que podemos notar es que, ya que $n$ es compuesto, debe existir un primo $p$ que divide a $n$. Podemos observar que se debe cumplir que $r<p<n$, ya que si fuera cierto que $p\leq r$, el algoritmo hubiera retornado en el paso \ref{alg-r3}, 
		%o en el paso \ref{alg-r4}, 
		pero no lo hizo (retornó en el paso \ref{alg-6}). Por otro lado, también sabemos que 
		%$\MCD(n,r)=1$, ya que si $1<\MCD(n,r)<n$, el algoritmo hubiera retornado en el paso \ref{alg-r3}. De lo anterior tenemos que 
		$n\in\mathbb{Z}_r^*$ ya que $O_r(n)$ está definido. Como $n=p\cdot m$ para algún número natural $m > 1$, sabemos que si $\MCD(p,r)\neq 1$, entonces $\MCD(n,r)\neq 1$. De esto concluimos que $\MCD(p,r)=1$, y $p\in \mathbb{Z}_r^*$.
	Además de esto, podemos afirmar que se cumple que $O_r(p)>1$, lo que demostraremos por contradicción a continuación (nótese que $O_r(p)$ está definido ya que $p \in \mathbb{Z}_r^*$). Supongamos que para todo $p$ primo divisor de $n$ se tiene que $O_r(p)=1$.	Sea la descomposición en primos de $n=p_1^{\alpha_1}\cdots p_t^{\alpha_t}=\prod_{i = 1}^t p_i^{\alpha_i}$. Luego, 
	%para todo $i$ tal que $1\leq i\leq t$ 
	se tiene que
		\begin{eqnarray*}
			O_r(p_i)=1 \ \text{ para todo } i \in \{1, \ldots, t\} &\Rightarrow &p_i^1 \equiv 1 \mods r \ \text{ para todo } i \in \{1, \ldots, t\}\\
			&\Rightarrow &n^1= \prod_{i=1}^t p_i^{\alpha_i} =\prod_{i = 1}^t (p_i^1)^{\alpha_i}\equiv \prod_{i = 1}^t 1^{\alpha_i}\equiv 1 \mod r\\
			&\Rightarrow &O_r(n)=1
		\end{eqnarray*}		 
		Esto es una contradicción, ya que $O_r(n)>\log^2n\geq 1$ cuando $n\geq 2$. Es por esto que debe existir al menos un primo $p$ divisor de $n$ tal que $O_r(p)>1$. 
%		Podemos observar que el primo $p$ que cumple con lo anterior también cumple que $r<p<n$, ya que si fuera cierto que $p\leq r$, el algoritmo hubiera retornado en el paso \ref{alg-r3}, 
%		%o en el paso \ref{alg-r4}, 
%		pero no lo hizo ya que retornó en el paso \ref{alg-6}. Por otro lado, también sabemos que 
%		%$\MCD(n,r)=1$, ya que si $1<\MCD(n,r)<n$, el algoritmo hubiera retornado en el paso \ref{alg-r3}. De lo anterior tenemos que 
%		$n\in\mathbb{Z}_r^*$ ya que $O_r(n)$ está definido. Como $n=p\cdot m$ para algún número natural $m > 1$, sabemos que si $\MCD(p,r)\neq 1$ entonces $\MCD(n,r)\neq 1$. De esto concluimos que $\MCD(p,r)=1$, y $p\in \mathbb{Z}_r^*$.
%		%\comentario{definir}.
		
	        %\comentarioin{Vamos a usar mod(algo) o 'en modulo algo' ?}
                De ahora en adelante, vamos a fijar $p$ y $r$ con las propiedades descritas anteriormente (en particular, sabemos que $r < p < n$). Además, definimos $\ell=\lfloor \sqrt{\phi(r)}\log  n\rfloor$, de forma que podamos analizar el paso \ref{alg-5} del algoritmo donde el valor de $a$ varía entre 1 y $\ell$. Este paso 
	%El paso \ref{alg-5} del algoritmo 
	verifica $\ell$ ecuaciones de la forma $(X+a)^n \not\equiv X^n+a \modulo$.
%	en el bucle\comentario{loop del for? o bucle?}. 
	Podemos observar de $\eqref{cota-ell}$ que $\ell \leq r < p$, y, como no se retornó inmediatamente después de este paso, tenemos que $(X+a)^n\equiv X^n+a$ en módulo $(X^r -1,n)$, para  cada $a \in \{0, \ldots, \ell\}$ (si $a=0$, la ecuación se cumple trivialmente). Luego, para cada $a \in \{0, \ldots, \ell\}$ se tiene que:
	\begin{align}
		&\hspace{-5pt}(X+a)^n\equiv X^n+a \modulos \nonumber\\
&\hspace{15pt}\Rightarrow\ (X+a)^n - (X^n+a) \equiv q(X)(X^r-1) \mod n \quad \text{ para } q(X)\in \mathbb{Z}[X]  \nonumber\\ 
&\hspace{15pt}\Rightarrow\ (X+a)^n - (X^n+a) -q(X)(X^r-1)= n\cdot k \quad \text{ para } q(X)\in \mathbb{Z}[X]  \text{ y } k \in \mathbb{Z} \nonumber\\
&\hspace{15pt}\Rightarrow\ (X+a)^n - (X^n+a) -q(X)(X^r-1)= p\cdot k_1 \quad \text{ para } q(X)\in \mathbb{Z}[X]  \text{ y } k_1 \in \mathbb{Z}, \text{ dado que } p \divi n \nonumber\\
&\hspace{15pt}\Rightarrow\ (X+a)^n - (X^n+a) \equiv q(X)(X^r-1) \mod p \quad \text{ para } q(X)\in \mathbb{Z}[X]  \nonumber\\ 
&\hspace{15pt}\Rightarrow\ (X+a)^n\equiv X^n+a \modulops \label{eq:7}
\end{align}
	Además, como $p$ es primo, 
	%$p\geq 2$ y $p\in \mathbb{Z}$, 
	por el lema \ref{lem-2.1} tenemos que para todo $a \in \mathbb{Z}$ tal que $\MCD(p,a)=1$, se tiene que $(X+a)^p \equiv X^p+a \mod p$. Así, dado que $\ell < p$ y esta congruencia se cumple trivialmente para $a = 0$, concluimos que para cada $a \in \{0, \ldots, \ell\}$:
	\begin{eqnarray}
		 %(X+a)^p &\equiv & X^p+a \modl p \hspace{0.5cm} \forall a \in \mathbb{Z} \text{ tal que } \MCD(p,a)=1  \nonumber\\
		(X+a)^p&\equiv & X^p+a \modulop \label{eq:8}		
	\end{eqnarray}
%\comentario{aca esta medio descontinuado pero no se como conectar}
		Las dos últimas ecuaciones, \eqref{eq:7} y \eqref{eq:8}, nos servirán para para demostrar la siguiente afirmación, la cual será útil para la demostración de la proposición \ref{prop:paso_final}:
	\begin{eqnarray}
		(X+a)^{\frac{n}{p}}&\equiv &X^{\frac{n}{p}}+a\modulop \quad\quad \text{para cada } a \in \{0, \ldots, \ell\}  \label{eq:9}
	\end{eqnarray}
	Para la demostración de \eqref{eq:9}, necesitaremos usar los lemas \ref{lema-a}, \ref{lema-b} y \ref{lema-polinomio_separable}, demostrados a continuación. %siguientes tres lemas:
        %\comentarioin{Marcelo: aquí pondría la definición de polinomio irreducible y la idea de que un polinomio se puede escribir como una multiplicación de polinomios irreducibles.}
Desde ahora en adelante vamos a trabajar con una estructura algebraica con buenas propiedades: los cuerpos. En caso de que el lector no esté familiarizado con esta materia puede dirigirse a la sección \ref{app-cuerpos} del apéndice, donde definimos y describimos las propiedades esenciales para leer este documento. Sea $(F,+,\cdot)$ un cuerpo, y desde ahora en adelante suponga que $\0$ y $\1$ son utilizados para denotar el nuetro de las operaciones $+$ y $\cdot$ en un cuerpo aarbitrario, respectivamente. Lo primero que haremos será introducir el concepto de polinomios irreducibles. Decimos que un polinomio $h(X)\in F[X]$ es irreducible si no puede ser factorizado por dos polinomios de menor grado en $F[X]$. Por ejemplo, en $\mathbb{R}[X]$ sabemos que el polinomio $X^2+1$ es irreducible, mientras que en $\mathbb{C}[X]$ no lo es, ya que $X^2+1 = (X+i)(X-i)$. Además, en $\mathbb{Z}_2[X]$ podemos ver que el mismo polinomio se puede escribir como \mbox{$X^2+1 \equiv (X+1)(X+1) \modl 2$}, y por lo tanto tampoco es irreducible. Nótese que irreducibilidad en $F[X]$ no es equivalente a que no hayan raíces en $F$: $(X^2+1)(X^2+1)$ no tiene raíces en $\mathbb{R}$ y sin embargo no es irreducible en $\mathbb{R}[X]$. Habiéndo introducido la noción de irreducibilidad, podemos hablar sobre la descomposición de polinomios en polinomios irreducibles. Dado un polinomio $q(X)\in F[X]$ podemos descomponerlo de la forma 
\begin{eqnarray}
	q(X) & = & \prod\limits_{i=1}^kr_i(X), \label{descomp pol}
\end{eqnarray}
donde cada $r_i(X)$ es un polinomio irreducible en $F[X]$.
%\comentarioin{Asumo que no es trivial la existencia de esta descomposición (en el sentido que sea finita), justamente porque en otras estructuras no se cumple, es por eso que puse a continuación que podemos asumir esta propiedad} Podemos asumir que esta propiedad se cumple cuando $F$ es un cuerpo, ya que la demostración se escapa del objetivo del documento.
%\comentarioin{igual esto es chanta (?)}
Dada esta propiedad,
%dado que $F$ es un cuerpo,
podemos afirmar que siempre existe una descomposición de la
forma
\begin{eqnarray}
  q(X) & = & c\cdot \prod\limits_{i=1}^k h_i(X), \label{descomp pol mon}
\end{eqnarray}
donde $c\in F$ y cada $h_i(X)$ es irreducible y mónico. Esto se debe a
que podemos tomar la descomposición en polinomios irreducibles
descrita en \eqref{descomp pol}, y multiplicar cada $r_i$ por el
inverso del coeficiente del término de mayor grado. Por ejemplo, en
$\mathbb{Z}_{5}[X]$ tenemos la siguiente descomposición del polinomio
$2X^2 + 2$:
\begin{eqnarray*}
2X^2 + 2 & \equiv & (3X+1)(4X+2) \modl 5.
\end{eqnarray*}
Sabemos que 2 es el inverso de 3 en módulo 5, y 4 es el inverso de sí
mismo en modulo 5. Usamos estos inversos para obtener una
descomposición del polinomio $2X^2 + 2$ de la forma \eqref{descomp pol
  mon}:
\begin{eqnarray*}
  2X^2 + 2 & \equiv & (3X+1)(4X+2) \modl 5\\
  & \equiv & (2\cdot 3) \cdot (4 \cdot 4) \cdot (3X+1)(4X+2) \modl 5\\
  & \equiv & (3 \cdot 4) \cdot (2 \cdot (3X+1))\cdot(4 \cdot (4X+2)) \modl 5\\
  & \equiv & (3 \cdot 4) \cdot (X+2)(X+3) \modl 5\\
    & \equiv & 2\cdot (X+2)(X+3) \modl 5.
\end{eqnarray*}
%el polinomio $3X+1$ es irreducible pero no mónico,
%pero si lo múltiplicamos por $3^{-1}$ obtenemos $3^{-1}\cdot (3X+1) =
%X+3^{-1}$, el cual, además de ser irreducible, es mónico, y sabemos
%que pertenece a $\mathbb{R}[X]$.
A continuación enunciamos los tres lemas que nos ayudarán a demostrar
la ecuación \eqref{eq:9}.
     
	\begin{lemma}\label{lema-a}
		Sean $a(X),b(X),r(X),s(X),t(X)\in \mathbb{Z}_p[X]$ tales que $r(X)$ y $s(X)$ no son nulos. Si
		\begin{itemize}
			\item[(i)] $r(X)$ y $s(X)$ son coprimos en $\mathbb{Z}_p[X]$ $($vale decir, $\MCD(r(X),s(X)) = 1$ en $\mathbb{Z}_p[X]$, como es definido en la sección~\ref{sec-notacion}$)$,
			%\comentarioin{Preliminares: $\MCD$ de 2 polinomios}
			
			\item[(ii)] $t(X)= r(X)\cdot s(X)$,
			\item[(iii)] $a(X)\equiv b(X)\modl (r(X),p)$, y
			\item[(iv)] $a(X)\equiv b(X)\modl (s(X),p)$,
		\end{itemize}
		entonces
		\begin{eqnarray}
			a(X)&\equiv &b(X)\modl (t(X),p) \nonumber 
		\end{eqnarray}
	\end{lemma}
        \begin{proof}
De la condición (iii) concluimos que $r(X)$ divide a $a(X) - b(X)$ en
$\mathbb{Z}_p[X]$, mientras que de la condición (iv) concluimos que
$s(X)$ divide a $a(X) - b(X)$ en $\mathbb{Z}_p[X]$. De esta forma,
concluimos que existen polinomios $\alpha(X), \beta(X) \in
\mathbb{Z}_p[X]$ tales que:
          \begin{eqnarray}\notag
            \alpha(X) \cdot r(X) & \equiv & a(X) - b(X) \modl p\\            
            \beta(X) \cdot s(X) & \equiv & a(X) - b(X) \modl p \label{eq-lema-a-1}
          \end{eqnarray}
          Así, tenemos que
          \begin{eqnarray}\label{eq-lema-a}
            \alpha(X) \cdot r(X) & \equiv & \beta(X) \cdot s(X) \modl p 
            \end{eqnarray}
Dado que $r(X)$ es un polinomio no nulo en $\mathbb{Z}_p[X]$ y $p$ es
un número primo, podemos obtener la siguiente descomposición del polinomio:
            \begin{eqnarray*}
            r(X) & \equiv & c \cdot \prod_{i=1}^k h_i(X) \mod p,
            \end{eqnarray*}
donde $k \geq 0$, $c \in \{1, \ldots, p-1\}$ y cada $h_i(X)$ es un
polinomio mónico de grado mayor a 0 e irreducible en $\mathbb{Z}_p[X]$. Dado
la condición \eqref{eq-lema-a}, podemos concluir existe un polinomio
$\gamma(X) \in \mathbb{Z}_p[X]$ tal que
\begin{eqnarray}\label{eq-lem-prod-irr}
            \beta(X) \cdot s(X) & \equiv & \gamma(X) \cdot \prod_{i=1}^k h_i(X)\mod p
            \end{eqnarray}
Por la condición (i) sabemos que $\MCD(r(X),s(X)) = 1$, por lo que
para cada $i \in \{1, \ldots, k\}$, el polinomio $h_i(X)$ no puede ser
parte de una descomposición de $s(X)$ como producto de irreducibles en
$\mathbb{Z}_p[X]$ (puesto que $h_i(X)$ es un polinomio mónico de grado
mayor a 0 y $h_i(X)$ divide a $r(X)$ en $\mathbb{Z}_p[X]$). Así, desde
\eqref{eq-lem-prod-irr} concluimos que existe un polinomio $\delta(X)
\in \mathbb{Z}_p[X]$ tal que
%\comentarioin{no entiendo por que dadas  esas codiciones se cumple lo siguiente}
\begin{eqnarray*}
            \beta(X) & \equiv & \delta(X) \cdot \prod_{i=1}^k h_i(X)\mod p,
\end{eqnarray*}
Además, sabemos que $c$ es invertible en módulo $p$ (ya que $c \in \{1,
\ldots, p-1\}$ y $p$ es un número primo), por lo que existe $d \in
\{1, \ldots, p-1\}$ tal que $c \cdot d \equiv 1 \modl p$. Por lo tanto,
tenemos que:
\begin{eqnarray*}
  \beta(X) = \delta(X) \cdot \prod_{i=1}^k h_i(X) \mod p& \Rightarrow &
  \beta(X) \equiv \delta(X) \cdot \prod_{i=1}^k h_i(X) \modl p\\
  & \Rightarrow & \beta(X) \equiv \delta(X) \cdot d \cdot c \cdot \prod_{i=1}^k h_i(X) \modl p\\
  & \Rightarrow & \beta(X) \equiv \delta(X) \cdot d \cdot r(X) \modl p\\
  & \Rightarrow & \beta(X) \cdot s(X) \equiv \delta(X) \cdot d \cdot r(X) \cdot s(X) \modl p\\
  & \Rightarrow & \beta(X) \cdot s(X) \equiv \delta(X) \cdot d \cdot t(X) \modl p \quad\quad\quad \text{por la condición (ii)}\\ 
  & \Rightarrow & a(X) - b(X) \equiv \delta(X) \cdot d \cdot t(X) \modl p \quad\quad\quad \text{por \eqref{eq-lema-a-1}}
\end{eqnarray*}
Por lo tanto, concluimos que $a(X) \equiv b(X) \modl (t(X),p)$, que es lo que teníamos que demostrar.
        \end{proof}
        
%	La demostración de este lema queda propuesta para el lector.
        %(\textbf{Hint:} demuestre primero que la identidad de Bezout funciona en $\mathbb{Z}_p[X]$ y a partir de eso demuestre el lema).
	%\comentario{deje propuesta la demostracion del lema}	
	\begin{lemma}\label{lema-b}
		Si se cumplen las ecuaciones $\eqref{eq:7}$ y $\eqref{eq:8}$, entonces para cada $i\in\mathbb{N}$ y $a \in \{0, \ldots, \ell\}$, se tiene que:
		\begin{eqnarray*}
			(X^{p^i}+a)^n&\equiv &(X^{p^i})^n + a \modulop
		\end{eqnarray*}
	\end{lemma} 
%	\comentarioin{Lo siguiente puede estar mejor argumentado, mas que `` recuerde que en la demostracion del lema 2.1 si reemplazamos...'' o bien hacer un corolario justo despues del lema 2.1}
        Para demostrar este lema, notemos que por la demostración del lema \ref{lem-2.1} sabemos que dado $q(X)\in \mathbb{Z}[X]$, como $p$ es primo, se tiene que 
		\begin{eqnarray*}
			(q(X)+a)^p&\equiv &q(X)^p + a \modl p
		\end{eqnarray*}
		Luego, si continuamos desarrollando esta expresión tenemos que
		\begin{eqnarray}
			(q(X)+a)^p\equiv q(X)^p + a \modl p\nonumber & \Leftrightarrow & (q(X)+a)^p - (q(X)^p + a)\equiv 0 \modl p\nonumber\\
			%& \Rightarrow & (X^r-1)\divi  (q(X)+a)^p - (q(X)^p + a) \modl p\nonumber\\
			& \Rightarrow & (q(X)+a)^p - (q(X)^p + a)\equiv 0 \modulop\nonumber\\
			& \Leftrightarrow & (q(X)+a)^p\equiv q(X)^p+a \modulop\label{eq:corolario2.1}
		\end{eqnarray}
	A continuación demostraremos el lema \ref{lema-b} mediante inducción fuerte.
	%\comentario{A lo mejor esta demostración va mejor en el apendice o demostraciones intermedias, y ver tema de continuidad de la oracion con las ecuaciones anteriores}
	\begin{proof}[Demostración del lema \ref{lema-b}]
		El caso base $i = 0$ se cumple por la ecuación \eqref{eq:7}. Para el paso inductivo, supongamos que $(X^{p^i}+a)^n\equiv (X^{p^i})^n + a \modulop$, a partir de lo cual obtenemos: 
%Queremos demostrar que $P(i+1)$ se cumple.
		\begin{align*}
			&(X^{p^i}+a)^n\equiv (X^{p^i})^n + a \modulops\\
			&\hspace{60pt}\Rightarrow \ ((X^{p^i}+a)^n)^p\equiv ((X^{p^i})^n + a)^p \modulops\\
			&\hspace{60pt}\Rightarrow \ ((X^{p^i}+a)^p)^n\equiv ((X^{p^i})^n + a)^p \modulops\\
			&\hspace{60pt}\Rightarrow \ (X^{p^{i+1}}+a)^n\equiv ((X^{p^i})^n)^p + a \modulop \quad\quad \text{por }\eqref{eq:corolario2.1}\\
			&\hspace{60pt}\Rightarrow \ (X^{p^{i+1}}+a)^n\equiv (X^{p^{i+1}})^n + a \modulop
		\end{align*}
		Vale decir, mostramos que la propiedad se cumple para $i+1$, lo cual concluye la demostración del lema. 
	\end{proof}

%\comentarioin{Marcelo: llegué hasta aquí con la revisión (revisé el apéndice \ref{sec-app-alg-int} completo).}
        
	\begin{lemma}\label{lema-polinomio_separable}
		Sea $K$ un cuerpo de característica $q$ y sea $m>1$
                un entero no divisible por $q$.
                %\comentarioin{aca por qué importa la característica?}
		%\comentarioin{Respuesta: porque si $m$ es divisible por la característica, entonces al derivar $X^m-1$ nos queda $mX^{m-1} =X^{m-1}\cdot (1 + \cdots + 1)=X^{m-1}\cdot 0= 0$, pero no queremos eso}
                Entonces podemos escribir:
	\begin{eqnarray*}
		X^m-\1 & = & \prod_{i=1}^{t}h_i(X), 
		\end{eqnarray*}
		donde cada $h_i(X)$ es un polinomio irreducible en $K[X]$,
$h_i(X)\neq h_j(X)$ para $i\neq j$ y $\grado(h_i(X)) \geq 1$.		 
	\end{lemma}
Antes de hacer la demostración del lema, necesitamos introducir las
nociones de raíces de la unidad, cuerpos de descomposición y cuerpo
ciclotómico.        
%\comentarioin{La siguiente definicion la puse en ambiente de definicion porque lo usamos despues y es necesario referenciarla. Me entra la duda si es muy horrible poner esa descripcion de la definicion: raices de la unidad, cuerpo....}
\begin{definition}[Raíces de la unidad, cuerpo de descomposición y cuerpo ciclotómico]\label{c. de descomposicion, ciclotomico def}
Sea $K$ un cuerpo. Decimos que $\eta\in K$ es una $m$-ésima raiz de la
unidad en $K$ si $\eta^m=1$ para $m\geq 1$. El conjunto que contiene
todas las $m$-raíces de la unidad es denotado por $E^{(m)}$.  Además,
si $p(X)\in K[X]$ es un polinomio no constante, entonces decimos que
el cuerpo de descomposición de $p(X)$ es la menor extensión de $K$ que
contiene todas las raíces de $p(X)$. En particular, para un polinomio
de la forma $p(X)=X^m-1$, el cuerpo de descomposición de $p(X)$ es
llamado
%lo anterior es válido y lo denominamos
el $m$-ésimo cuerpo ciclotómico, el cual denotaremos como $K^{(m)}$.
\end{definition}

\begin{example}\label{exa-cuerpo-desc}
El polinomio $X^2-2$ es irreducible en el cuerpo
$(\mathbb{Q},+,\cdot)$, puesto que $\sqrt{2}$ es una de sus raíces y
este no es un número racional. El cuerpo de descomposición de $X^2-2$
puede ser definido de la siguiente forma (si no está familiarizado con
la siguiente notación por favor vea el apéndice \ref{app-cuerpos}):
\begin{eqnarray*}
  \mathbb{Q}[X]/(X^2-2) & = & \{p(X) \mods (X^2-2)  \ \mid \  p(X)\in \mathbb{Q}[X]\}.
\end{eqnarray*}
Nótese que $\mathbb{Q}[X]/(X^2-2)$ extiende a $(\mathbb{Q},+,\cdot)$, puesto que:
\begin{eqnarray*}
  \mathbb{Q}[X]/(X^2-2) & = & \{a+bX  \mid  a,b \in \mathbb{Q}\}.
\end{eqnarray*}
Además, $X$ y $-X$ son raíces del polinomio $Y^2-2$ en $\mathbb{Q}[X]/(X^2-2)$ puesto que:
\begin{eqnarray*}
  X^2 - 2 & \equiv & 0 \modl X^2 - 2\\
  (-X)^2 - 2 & \equiv & 0 \modl X^2 - 2
\end{eqnarray*}
De hecho el cuerpo $\mathbb{Q}[X]/(X^2-2)$ suele denotarse como
$\mathbb{Q}(\sqrt{2})$, donde los elementos son de la forma $a +
b\sqrt{2}$ para $a,b \in \mathbb{Q}$. Es decir, $\mathbb{Q}(\sqrt{2})$
correspode al cuerpo $\mathbb{Q}[X]/(X^2-2)$ pero con la variable $X$
reemplazada por~$\sqrt{2}$.

Considere ahora el polinomio $(X^2-2)(X^2-3)$, el cual también es
irreducible en el cuerpo $(\mathbb{Q},+,\cdot)$, puesto que ni
$\sqrt{2}$ ni $\sqrt{3}$ son números racionales. Para construir el
cuerpo de descomposición de $(X^2-2)(X^2-3)$, primero construimos el
cuerpo $(\mathbb{Q}[X]/(X^2-2), +, \cdot)$. Luego notando que el
polynomio $Y^2 - 3$ es irreducible en este cuerpo, construimos el
siguiente cuerpo:
\begin{eqnarray*}
  (\mathbb{Q}[X]/(X^2-2))[Y]/(Y^3-2) & = & \{p(Y) \mods (Y^2-3)  \ \mid \  p(Y)\in (\mathbb{Q}[X]/(X^2-2))[Y]\}
\end{eqnarray*}
Nótese que $((\mathbb{Q}[X]/(X^2-2))[Y]/(Y^3-2),+,\cdot)$ extiende a
$(\mathbb{Q}[X]/(X^2-2),+,\cdot)$ y  $(\mathbb{Q},+,\cdot)$, puesto
que:
\begin{eqnarray*}
  (\mathbb{Q}[X]/(X^2-2))[Y]/(Y^3-2) & = & \{(a+bX) + (c+dX)Y  \mid  a+bX, c+dX \in \mathbb{Q}[X]/(X^2-2)\}\\
  & = & \{ a+ bX + cY + dXY  \mid  a,b,c,d \in \mathbb{Q}\}.
\end{eqnarray*}
Además, $Y$ y $-Y$ son raíces del polinomio $Y^2-3$ en
$((\mathbb{Q}[X]/(X^2-2))[Y]/(Y^3-2)$ puesto que:
\begin{eqnarray*}
  Y^2 - 3 & \equiv & 0 \modl Y^2 - 3\\
  (-Y)^2 - 3 & \equiv & 0 \modl Y^2 - 3
\end{eqnarray*}
De hecho el cuerpo $((\mathbb{Q}[X]/(X^2-2))[Y]/(Y^3-2)$ suele
denotarse como $\mathbb{Q}(\sqrt{2},\sqrt{3})$, donde los
elementos son de la forma $a + b\sqrt{2} + c\sqrt{3} + d \sqrt{6}$
para $a,b,c,d \in \mathbb{Q}$. Es decir,
$\mathbb{Q}(\sqrt{2},\sqrt{3})$ correspode al cuerpo
$((\mathbb{Q}[X]/(X^2-2))[Y]/(Y^3-2)$ pero con las variables $X$ e $Y$
reemplazadas por $\sqrt{2}$ y $\sqrt{3}$, respectivamente, de manera
tal que $XY = \sqrt{2} \cdot \sqrt{3} = \sqrt{6}$.

Finalmente considere el polinomio $X^3-2$, el cual también es
irreducible en el cuerpo $(\mathbb{Q},+,\cdot)$, puesto que
$\sqrt[3]{2}$ no es un número racional. Para construir el cuerpo de
descomposición de $X^3-2$, primero construimos el cuerpo
$(\mathbb{Q}[X]/(X^3-2), +, \cdot)$, donde:
\begin{eqnarray*}
  \mathbb{Q}[X]/(X^3-2) & = & \{p(X) \mods (X^3-2)  \ \mid \  p(X)\in \mathbb{Q}[X]\}.
\end{eqnarray*}
Nótese que $\mathbb{Q}[X]/(X^3-2)$ extiende a $(\mathbb{Q},+,\cdot)$ puesto que:
\begin{eqnarray*}
  \mathbb{Q}[X]/(X^3-2) & = & \{a+bX+cX^2  \mid  a,b,c \in \mathbb{Q}\}.
\end{eqnarray*}
Además, $X$ es una raíz del polinomio $Y^3-2$ en $\mathbb{Q}[X]/(X^3-2)$ puesto que:
\begin{eqnarray*}
  X^3 - 2 & \equiv & 0 \modl X^3 - 2
\end{eqnarray*}
De hecho, al igual que en los casos anteriores, el cuerpo
$\mathbb{Q}[X]/(X^3-2)$ suele denotarse como
$\mathbb{Q}(\sqrt[3]{2})$, donde los elementos son de la forma $a +
b\sqrt[3]{2} + c(\sqrt[3]{2})^2$ para $a,b,c \in \mathbb{Q}$. Es
decir, $\mathbb{Q}(\sqrt[3]{2})$ correspode al cuerpo
$\mathbb{Q}[X]/(X^3-2)$ pero con la variable $X$ reemplazada
por~$\sqrt{2}$, de manera tal que $X^2 = (\sqrt[3]{2})^2$.  Para saber
si este es el cuerpo de descomposición de $X^3 - 2$, tenemos que
averiguar si todas las raíces del polinomio $Y^3 - 2$ están en
$\mathbb{Q}[X]/(X^3-2)$. Para hacer esto, observamos que en este cuerpo:
\begin{eqnarray*}
  Y^3 - 2 & \equiv (Y-X)(Y^2 + XY + X^2).
  \end{eqnarray*}
    Mirado desde el punto de vista de $\mathbb{Q}(\sqrt[3]{2})$, esto corresponde a decir que:
\begin{eqnarray*}
  Y^3 - 2 & \equiv (Y-\sqrt[3]{2})(Y^2 + \sqrt[3]{2}Y + (\sqrt[3]{2})^2).
  \end{eqnarray*}    
Se puede verificar que el polinomio $Y^2 + XY + X^2$ es irreducible en
$(\mathbb{Q}[X]/(X^3-2))[Y]$, de la misma forma que el polinomio $Y^2 +
\sqrt[3]{2}Y + (\sqrt[3]{2})^2$ es irreducible en
$\mathbb{Q}(\sqrt[3]{2})[Y]$ (puesto que las raíces de este polinomio
tienen una componente imaginaria). Así, para construir la menor
extensión de $(\mathbb{Q}, +, \cdot)$ que incluye las raíces del
polinomio $X^3 -2$, tenemos que considerar el cuerpo
$((\mathbb{Q}[X]/(X^3-2))[Y]/(Y^2 + XY + X^2),+,\cdot)$, lo cual es
equivalente a consider el cuerpo $(\mathbb{Q}(\sqrt[3]{2})[Y]/(Y^2 +
\sqrt[3]{2}Y + (\sqrt[3]{2})^2),+,\cdot)$. Nótese que
$(\mathbb{Q}[X]/(X^3-2))[Y]/(Y^2 + XY + X^2)$ extiende a
$(\mathbb{Q}[X]/(X^3-2),+,\cdot)$ y $(\mathbb{Q},+,\cdot)$ puesto que:
\begin{eqnarray*}
  (\mathbb{Q}[X]/(X^3-2))[Y]/(Y^2 + XY + X^2) & = & \{(a+bX+cX^2) +
  (a'+b'X+c'X^2)Y \mid a+bX+cX^2,\\ && \hspace{111pt} a'+b'X+c'X^2 \in
  \mathbb{Q}[X]/(X^3-2)\}\\ & = & \{ a+ bX + cX^2 + a'Y + b'XY + c'X^2Y
  \mid a,b,c,a',b',c' \in \mathbb{Q}\}.
\end{eqnarray*}
Como comentario final, cabe destacar que el cuerpo
$((\mathbb{Q}[X]/(X^3-2))[Y]/(Y^2 + XY + X^2),+,\cdot)$ suele
denotarse como
\begin{align*}
  \mathbb{Q}\bigg(\sqrt[3]{2}, \frac{-1+i\sqrt{3}}{(\sqrt[3]{2})^2}\bigg),
\end{align*}
donde las variables $X$ e $Y$ han sido reemplazadas por $\sqrt[3]{2}$
y $\frac{-1+i\sqrt{3}}{(\sqrt[3]{2})^2}$, respectivamente, puesto que
este último valor es una raiz del polinomio $Y^2 + \sqrt[3]{2}Y +
(\sqrt[3]{2})^2$.  \qed
\end{example}


\begin{proof}[Demostración del lema \ref{lema-polinomio_separable}]
  Considere el siguiente polinomio en $K[X]$:
  \begin{eqnarray*}
    f(X) & = & \sum_{i=0}^{k} a_iX^i.
  \end{eqnarray*}
  Defina entonces la función
  %$D(f)(X):K[X] \to K[X]$
  $D:K[X] \to K[X]$ de la siguiente manera:
\begin{eqnarray*}
	D(f(X)) &=& \sum_{i=1}^{k}i\cdot a_i X^{i-1}
\end{eqnarray*}
%Podemos ver que para dos polinomios $p(X)$ y $q(X)$ equivalentes en
%$K[X]$ necesariamente $p'(X)\equiv q'(X)$, lo cual es fundamental para
%que el mapeo esté bien definido.  También se puede verificar
Dada la definición de $D$, se puede verificar rápidamente que si $f(X),g(X)\in K[X]$, $c\in K$ y $\ell$ es un número natural mayor a 0, se tiene que:
%se cumplen las siguientes propiedades:
\begin{itemize}
%	\item $(f+g)'=f'+g'$
%	\item $(f\cdot g)'=f'\cdot g + f\cdot g'$
%	\item $(c\cdot f)'=c\cdot f'$
%	\item $((X+c)^{\ell})'=\ell\cdot(X+c)^{\ell -1}$
	\item $D(f(X)+g(X)) = D(f(X)) + D(g(X))$
	\item $D(f(X) \cdot g(X)) = D(f(X)) \cdot g(X) + f(X) \cdot D(g(X))$
	\item $D(c\cdot f(X)) = c\cdot D(f(X))$
	\item $D((X+c)^{\ell}) = \ell\cdot(X+c)^{\ell -1}$
\end{itemize}
Las demostraciones de estas propiedades quedan propuestas como
ejercicio para el lector (puede ser conveniente usar el teorema del
binomio para la última propiedad).

%Sea $m>1$ un entero no divisible por $q$.\comentarioin{aca nuevamente no se por que $m$ no puede ser multiplo de $q$, y creo que ya es redundante} 
Sabemos que 
\begin{eqnarray*}
		X^m-1 & = & \prod_{i=1}^{t}h_i(X)^{\alpha_i}, 
		\end{eqnarray*}
donde cada $h_i(X)$ es un polinomio irreducible en $K[X]$, $h_i(X)\neq h_j(X)$ para $i\neq j$ y $\grado(h_i(X))=m_i$ con $m_i \geq 1$. Además cada $\alpha_i$ es un entero positivo.
%\comentarioin{en vez de poner todo lo anterior, quizas decir que es la descomposicion en polinomios irreducibles en $K[X]$}
Primero supongamos que todas las $m$-ésimas raíces de la unidad
$\eta_i$ están en $K$. Esto quiere decir que podemos escribir
$h_i(X)=(X-\eta_i)$ para cada $i$. Con el objetivo de demostrar que
$\alpha_j = 1$ para todo $j\in\{1,...,t\}$, supongamos por
contradicción que existe un $j$ tal que $\alpha_j>1$. Luego, en $K[X]$
podemos escribir
		\begin{eqnarray*}
			 X^{m}-1  &=& (X-\eta_j)^{\alpha_j}\cdot g(X),
	\end{eqnarray*}
	        donde
                \begin{eqnarray*}
			 g(X) &=& \prod_{\substack{i=1\\ i\neq j}}^{t}h_i(X)^{\alpha_i}.
	\end{eqnarray*}
                Aplicando en ambos lados de la equivalencia la función $D$ introducido al inicio de la demostración, obtenemos:
                %, nos queda
	\begin{eqnarray}
		m X^{m-1} & = & \alpha_j (X-\eta_j)^{\alpha_j -1}\cdot g(X)\ + \ (X-\eta_j)^{\alpha_j}\cdot D(g(X))\label{ecuacion derivada} 
		\end{eqnarray}
	Podemos observar en \eqref{ecuacion derivada} que $\eta_j$ no es una solución del polinomio de la izquierda, ya que $m$ no es un múltiplo de la característica de $K$
        %\comentarioin{aca tenemos que explicar con mas detalle?}
        y $\eta_j\neq \0$ (puesto que $\0^m\neq \0$). Además, notamos que $\eta_j$ sí es solución del polinomio de la derecha en \eqref{ecuacion derivada}, lo cual es una contradicción.   
		%es una de las raíces del polinomio del lado derecho, pero no del izquierdo (notar que $\eta_j\neq 0$ ya que $0$ no es $m$-ésima raiz de la unidad), lo cual es una contradicción. 
	De esto concluimos que si todas las $m$-ésimas raíces de la unidad son elementos de $K$, entonces la multiplicidad de cada una de estas es 1 y por ende $\alpha_i = 1$ para todo $i \in \{1, \ldots, t\}$, lo cual era lo que necesitabamos demostrar.
        
	Ahora supongamos que no todas las $m$-ésimas raíces de la unidad están en $K$ y que existe un $j$ tal que $\alpha_j>1$. Sabemos que $K$ está contenido en $K^{(m)}$, y además que $K^{(m)}$ es un cuerpo de característica $q$ (ver construcción de $K^{(m)}$ en ejemplo \ref{exa-cuerpo-desc}, y para más detalles revisar \cite{AlgebraLang}).
%Se puede demostrar que cualquier cuerpo $K$ está contenido en $K^{(r)}$ (para más detalles revisar \cite{AlgebraLang}).
        %\comentarioin{Cite el libro Algebra de Lang, pero no se si es necesario: ¿es facil ver que K tiene el cuerpo de extension finito con la raíces de la unidad? Creo que no}
        Sabemos que $h_j(X)$ es irreducible en $K[X]$, pero en $K^{(m)}[X]$ podemos factorizarlo de la siguiente manera:
		\begin{eqnarray*}
			h_j(X) & = &\prod_{i=1}^{s} (X-\eta_{j,i})
		\end{eqnarray*}
Tomando cualquier $k\in [1,s]$, podemos escribir $X^m - 1$ en
$K^{(m)}[X]$ de la siguiente forma:
\begin{eqnarray*}
	X^m-1 &=& (X-\eta_{j,k})^{\alpha_j}\cdot g(X).
\end{eqnarray*}
Así, suponiendo que $\alpha_j >1$ llegamos a la misma contradicción
que en el primer caso de la demostración, y podemos afirmar que
$h_j(X)$ no tiene raíces de multiplicidad mayor que 1. De lo anterior
concluimos que en este caso también se cumple que $\alpha_j=1$ para
cualquier $j\in \{1,...,t\}$, con lo cual queda demostrado el~lema.
\end{proof}  

%\comentarioin{Marcelo: revisé hasta aquí}

	Ahora, usando los lemas \ref{lema-a}, \ref{lema-b} y \ref{lema-polinomio_separable} podemos demostrar la ecuación (\ref{eq:9}).
	%\comentario{esta demostracion la queria poner de forma continua (sin el ambiente proof), pero es muy larga. quizas toda la demostracion de (\eqref{eq:9}) debe ir en el apendice?}
	\begin{proof}[Demostración de la ecuación $(\ref{eq:9})$]
	%\comentario{a lo mejor esto va en el apendice}
		Defina $K = \mathbb{Z}_p$. Como $p$ y $r$ son coprimos, entonces $p$ no divide a $r$. Luego, por el lema \ref{lema-polinomio_separable} sabemos que 
		\begin{eqnarray*}
		X^r-1 & = & \prod_{i=1}^{t}h_i(X),
		\end{eqnarray*}	
		donde cada $h_i(X)$ es un polinomio irreducible en $\mathbb{Z}_p[X]$, $h_i(X)\neq h_j(X)$ para $i\neq j$, y $\grado(h_i(X))=m_i$ con $m_i\geq 1$. Sea $h(X) \in \{h_1(X),...,h_t(X)\}$, y suponga que $\grado(h(X))=m$ (en particular, $m \geq 1$).
                %\comentarioin{ya sabemos que $h(X)$ tiene grado mayor
                %igual a 1. Esto para que no sean una constante. Ojo
                %que quizas haya que imponer un polinomio constante
                %(solo 1)} ACA SABEMOS QUE TODOS LOS POLINOMIOS SON
                %DISTINTOS.
                Del lema \ref{lema-b}, si tomamos $i=m-1$ tenemos que $$(X^{p^{m-1}}+a)^n\equiv (X^{p^{m-1}})^n +a \modulops,$$ para cada $a \in \{0, \ldots, \ell\}$ (recuerde que $\ell=\lfloor \sqrt{\phi(r)}\log  n\rfloor$ en la la ecuación \eqref{eq:9}). Así, dado que $h(X)$ divide a $X^r-1$, concluimos que para cada $a \in \{0, \ldots, \ell\}$ se cumple lo siguiente:
		\begin{eqnarray}
%		   & & (X^{p^{m-1}}+a)^n\equiv (X^{p^{m-1}})^n +a \modulop,\hspace{0.5cm}\text{y como }h(X)\text{ divide a }  X^r-1 \nonumber\\
		    (X^{p^{m-1}}+a)^n & \equiv & (X^{p^{m-1}})^n +a \modulohps \label{eq:ast} 
		\end{eqnarray}
		Por otro lado, si fijamos $q(X)=X^{p^{m-1}}$ en $\eqref{eq:corolario2.1}$, podemos afirmar que $$(X^{p^{m-1}}+a)^p\equiv X^{p^{m}} +a \modulop,$$ para cada $a \in \{0, \ldots, \ell\}$.  Al igual que antes, como $h(X)$ divide a $X^r-1$, concluimos que para cada $a \in \{0, \ldots, \ell\}$:
		\begin{eqnarray}
		    %& & (X^{p^{m-1}}+a)^p\equiv X^{p^{m}} +a \modulop,\hspace{0.5cm}\text{y como }h(X)\text{ divide a }  X^r-1 \nonumber\\
		    (X^{p^{m-1}}+a)^p & \equiv & X^{p^{m}} +a \modulohps \label{eq:ast2} 
		\end{eqnarray}
		%\comentarioin{el párrafo siguiente contiene mucha información que va en preliminares: ¿Hay que refernciar mucho o puedo asumir que en este punto ya saben o leyeron el apéndice?}
		%Para continuar con la demostración de la ecuación $\eqref{eq:9}$, utilizaremos algunas propiedades de cuerpos. 
		Para continuar con la demostración definimos $F = \mathbb{Z}_p[X]/h(X)$. Nótese que nos interesa la estructura $F$ ya que estamos trabajando dentro de esta al operar en módulo $(h(X),p)$. Dado que $h(X)$ es un polinomio irreducible de grado $m\geq 1$, se tiene que $F$ es un cuerpo finito de cardinalidad $p^m$. Esto implica que $(F\setminus\{0\},\cdot)$ 
		es un grupo de cardinalidad $p^m-1$.
		Como $X^r\equiv 1 \modulops$, se tiene que $X^r\equiv 1 \modulohps$ puesto que $h(X)$ divide a $X^r-1$.
		Sea $s \in \{1, \ldots, r\}$ el orden de $X$ en $(F\setminus\{0\},\cdot)$ (tenemos que $r > 1$).
                %el menor número tal que $X^s\equiv 1 \modulohps$.
		%+++++
		% en módulo ($h(X),p$). 
		 Como sabemos que el grupo  generado por $X$ en $(F\setminus\{0\},\cdot)$ es subgrupo de $(F\setminus\{0\},\cdot)$, entonces por el teorema de Lagrange obtenemos que $|\langle X\rangle |$ divide a $p^m-1$, vale decir, $s$ divide a $p^m-1$. Esto es equivalente a decir que $s \alpha = p^m -1$ para algún número $\alpha \in \mathbb{N}$. Luego, considerando que  %sumando 1 a ambos lados obtenemos que 
		$p^m=s\alpha +1$, obtenemos para cada $a \in \{0, \ldots, \ell\}$ lo siguiente:
		\begin{eqnarray}
		    (X^{p^{m-1}}+a)^n & \equiv &((X^{p^{m-1}}+a)^p)^{\frac{n}{p}}\modulohps\nonumber\\
		    & \equiv &(X^{p^{m}}+a)^{\frac{n}{p}}\modulohps \quad\quad \text{por }\eqref{eq:ast2} \nonumber\\
		    & \equiv &(X^{s\alpha +1}+a)^{\frac{n}{p}}\modulohps\nonumber\\
		    & \equiv &((X^{s})^{\alpha} \cdot X+a)^{\frac{n}{p}}\modulohps\nonumber\\
		    & \equiv &(1^{\alpha} \cdot X+a)^{\frac{n}{p}}\modulohps\nonumber\\
		    & \equiv &(X+a)^{\frac{n}{p}}\modulohps\label{eq:+}
		\end{eqnarray}
		Además, para cada $a \in \{0, \ldots, \ell\}$ tenemos que:
		\begin{eqnarray}
		    (X^{p^{m-1}}+a)^n & \equiv &(X^{p^{m-1}})^n + a\modulohps \quad\quad \text{por }\eqref{eq:ast} \nonumber\\
		    & \equiv &((X^{p^{m-1}})^p)^{\frac{n}{p}} + a\modulohps\nonumber\\
		    & \equiv &(X^{p^{m}})^{\frac{n}{p}} + a\modulohps\nonumber\\
		    & \equiv &(X^{s\alpha+1})^{\frac{n}{p}} + a\modulohps\nonumber\\
		    & \equiv &((X^{s})^{\alpha}\cdot X)^{\frac{n}{p}} + a\modulohps\nonumber\\
		    & \equiv &(1^{\alpha}\cdot X)^{\frac{n}{p}} + a\modulohps\nonumber\\
		    & \equiv & X^{\frac{n}{p}} + a\modulohps\label{eq:++}
		\end{eqnarray}
		%\comentario{será necesario poner todos los modulo(h,p)?}
		De las ecuaciones $\eqref{eq:+}$ y $\eqref{eq:++}$, podemos concluir que para cada $a \in \{0, \ldots, \ell\}$:
		\begin{eqnarray*}
		    (X+a)^{\frac{n}{p}}&\equiv &X^{\frac{n}{p}} + a\modulohps
		\end{eqnarray*}
		
		Finalmente, dado que $h(X)$ es un elemento arbitrario en $\{h_1(X),...,h_t(X)\}$ (y todos los polinomios son coprimos entre sí), concluimos por el lema \ref{lema-a} que para cada $a \in \{0, \ldots, \ell\}$:
		\begin{eqnarray*}
		    (X+a)^{\frac{n}{p}}&\equiv &X^{\frac{n}{p}} + a\modulops
		\end{eqnarray*}
		Esto concluye la demostración de la ecuación $(\ref{eq:9})$.
	\end{proof}		
%	\comentarioin{Marcelo: hasta aquí revisé} 
	A continuación, definiremos una propiedad sobre polinomios que será fundamental para la demostración de la correctitud del algoritmo.
	\begin{definition}
	Para un polinomio $f(X)$ y un número $m\in \mathbb{N}$, decimos que $m$ es introspectivo para $f(X)$~si
	\begin{eqnarray*}
	[f(X)]^m & \equiv & f(X^m) \modulop
	\end{eqnarray*}
	\end{definition}
Podemos ver de las ecuaciones $\eqref{eq:8}$ y $\eqref{eq:9}$ que $p$
y $\frac{n}{p}$ son introspectivos para el polinomio $g(X) = X+a$
cuando $a \in \{0, \ldots, \ell\}$.  El siguiente lema muestra que el
conjunto de valores que son introspectivos para un polinomio $f(X)$ es
cerrado bajo multiplicación.
	\begin{lemma}\label{lema-4.7}
		Si $m_1$ y $m_2$ son introspectivos para $f(X)$, entonces $m_1\cdot m_2$ es introspectivo para $f(X)$.
	\end{lemma}
	\begin{proof}
Como $m_1$ es introspectivo para $f(X)$, entonces tenemos lo siguiente:
			\begin{eqnarray}
			[f(X)]^{m_1m_2}& \equiv &[f(X^{m_1})]^{m_2}\modulop\label{eq:izq_instrospectivo}
			\end{eqnarray}
Por otro lado, como $m_2$ también es introspectivo para el polinomio,
entonces se debe cumplir lo siguiente:
			\begin{eqnarray*}
				[f(Y)]^{m_2} & \equiv & [f(Y^{m_2})] \moduloy
		        \end{eqnarray*}
Así, reemplazando $Y$ por $X^{m_1}$ obtenemos:                        
			\begin{eqnarray}
		          [f(X^{m_1})]^{m_2}\ \equiv \ f(X^{m_1m_2})\modulomr\label{eq:modulo m1r}
		\end{eqnarray}
		Como $X^{r} \equiv  1\modulop$, se tiene que $X^{m_1r}\equiv 1\modulop$. En otras palabras, esto significa que $X^r -1$ divide a $X^{m_1r}-1$ en $\mathbb{Z}_p[X]$. 
			%&\Rightarrow &X^{m_1r}-1 \equiv (X^{r})^{m_1}-1 \equiv 1^{m_1}-1 \equiv 0  \hspace{0.4cm}(mod\hspace{0.1cm} X^{r}-1)\nonumber\\
			%&\Leftrightarrow &X^r-1\divi  X^{m_1r}-1\nonumber
De lo anterior y  de la ecuación \eqref{eq:modulo m1r} podemos afirmar lo siguiente:
\begin{eqnarray*}
  [f(X^{m_1})]^{m_2}&\equiv & f(X^{m_1m_2})\modulop
  %\label{eq:der_introspectivo}
		\end{eqnarray*}
Desde esta ecuación y \eqref{eq:izq_instrospectivo} concluimos que
\begin{eqnarray*}
  [f(X)]^{m_1m_2} & \equiv & f(X^{m_1m_2})\modulop,
\end{eqnarray*}
vale decir, $m_1\cdot m_2$ es introspectivo para $f(X)$.	
	\end{proof}
	El siguiente lema muestra que, para un número natural $m$, el conjunto de polinomios para los cuales $m$ es introspectivo también es cerrado bajo multiplicación.
	\begin{lemma}\label{lem-introspectivo_bajo mult}
		Si $m$ es introspectivo para $f(X)$ y $g(X)$, entonces también lo es para $p(X) = f(X)\cdot g(X)$.
	\end{lemma}
	\begin{proof}Tenemos que:
		\begin{eqnarray*}
			[p(X)]^m &\equiv & [f(X)g(X)]^m \modulop\\
			&\equiv &[f(X)]^m [g(X)]^m \modulop\\
			&\equiv &f(X^m) g(X^m) \modulop\\
			&\equiv &p(X^m) \modulop.
		\end{eqnarray*}
	\end{proof}
	\comentarior{Marcelo: hasta aquí revisé} 
	A continuación, definimos dos conjuntos:
	\begin{eqnarray}
		I&:=&\{(\frac{n}{p})^i\cdot p^j \tq i,j\in \mathbb{N}\}\nonumber\\
		P&:=&\{\prod_{a=0}^{\ell }(X+a)^{e_a}\tq e_a\in \mathbb{N}\}\nonumber	
	\end{eqnarray}
	Observemos que las ecuaciones $\eqref{eq:8}$ y $\eqref{eq:9}$ y los lemas \ref{lema-4.7} y \ref{lem-introspectivo_bajo mult} nos permiten afirmar que cualquier elemento de $I$ es introspectivo para cualquier polinomio de $P$, ya que, en particular, se cumple para cada término de las multiplicaciones (y los conjuntos son cerrados bajo multiplicación).
	
	A continuación vamos a definir dos grupos, $G$ y $\mathcal{G}$, que están basados en $I$ y $P$ que van a ser muy importantes para el desarrollo de la demostración. Comenzamos definiendo $G$.
		\begin{eqnarray}
			G&:=& \{e\modl r\tq e\in I\}\nonumber
		\end{eqnarray}
	Observemos que cada elemento $e\in I$ es de la forma $(\frac{n}{p})^i\cdot p^j$ con $i,j\in \mathbb{N}$. Como sabemos que $n$ y $p$ son ambos coprimos de $r$, entonces $e$ también debe ser coprimo de $r$.
	%Notemos que en la definición anterior, como $e\in I$, entonces $e$ es de la forma $(\frac{n}{p})^i\cdot p^j$ con $i,j\in \mathbb{N}$. Además, como $n$ y $p$ son ambos coprimos de $r$, entonces $e$ también es coprimo de $r$ para cualquier elemento $e\in I$. 
	Esto se puede mostrar por contradicción: si existiera un $e\in I$ tal que $\MCD(e,r)\neq 1$, tendrí­amos que existe un primo $c$ divisor de $(\frac{n}{p})^i\cdot p^j$ y también divisor de $r$. Al ser $c$ primo, este dividiría a $\frac{n}{p}$ o a $p$. Como $c$ divide a $r$ tendríamos que $\MCD(n,r)\geq c>1$ o $\MCD(p,r)\geq c>1$, ambas contradicciones.
	 Por último, como $G\subseteq I$, y cada elemento de $I$ es coprimo con $r$, entonces cada elemento de $G$ es coprimo con $r$, y por ende $G\subseteq \mathbb{Z}_r^*$. Lo anterior nos permite concluir que el grupo $(G,\cdot)$ es un subgrupo de $(\mathbb{Z}_r^*,\cdot)$, en donde la operación binaria $\cdot$ representa la multiplicación usual. La demostración de lo anterior queda propuesta para el lector (Hint: como ya sabemos que $G\subseteq \mathbb{Z}_r^*$, solo hace falta demostrar que $(G,\cdot)$ es un grupo).	
%	\begin{proof}
%	Demostraremos las tres propiedades:
%		\begin{enumerate}
%			\item PD: Tiene elemento identidad.
%			Podemos notar que para cualquier $m$ en $G$, $e\cdot 1 =1\cdot e = e$\comentario{o mejor decir el elemento neutro $n=1$?}, por lo que hay que ver si $1$ es un elemento de $G$. Como $1 = (\frac{n}{p})^0\cdot p^0$, entonces es inmediato que $1$ pertenece a $G$.
%			\item PD: Si $a,b$ son elementos de $G$, entonces $a\cdot b$ también está en $G$.
%			Tomamos $a,b$ en $G$. Estos elementos tienen un representante en el conjunto $I$, es decir, existen $a_I$ y $b_I$ en $I$ tales que 
%			\begin{align}
%				\left \{ \begin{matrix} &a_I\modl r = a
%\\ &b_I\modl r=b\end{matrix}\right. \nonumber\\
%				\Leftrightarrow \left \{ \begin{matrix} &a_I\equiv a\modl r
%\\ &b_I\equiv b\modl r\end{matrix}\right. \label{eq:corchete} 				
%			\end{align}
%			Sabemos que $a_I = (\frac{n}{p})^{i_1}\cdot p^{j_1}$ y que $b_I = (\frac{n}{p})^{i_2}\cdot p^{j_2}$. Al multiplicarlos obtenemos que $a_I\cdot b_I = (\frac{n}{p})^{i_3}\cdot p^{j_3}\in I$, donde $i_3 = i_1+i_2$ y $j_3 = j_1+j_2$. Luego, $a_I\cdot b_I$ están en $I$ y definimos $d=a_I\cdot b_I\modl r$. Aplicando módulo y usando $\eqref{eq:corchete}$, tenemos que 
%			\begin{eqnarray}
%				a\cdot b &\equiv & a_I\cdot b_I \nonumber\\
%				&\equiv & d\modl r\nonumber 
%			\end{eqnarray}
%			En donde $d$ pertenece a $G$. Luego, la operación es cerrada en $G$.
%			\item PD: Si $a$ pertenece a $G$, entonces existe un elemento inverso $a^{-1}$ en $G$ tal que $a\cdot a^{-1} = 1$.
%			Sea $a$ en $G$, y sea $a_I$ su representante en $I$ como se definió en $\ref{eq:corchete}$ . Sabemos que el inverso multiplicativo de $a_I\modl r$ existe si, y solo si $\MCD(a_I,r)=1$. Como ya probamos que cualquier elemento de $I$ es coprimo de $r$, entonces podemos afirmar que existe un  $b_I$ en $I$ tal que $a_I\cdot b_I \equiv 1$ en módulo $r$. Sea $b=b_I\modl r$. Luego,
%			\begin{eqnarray}
%				a\cdot b &\equiv &a_I\cdot b_I  \modl r\nonumber\\
%				&\equiv &1\modl r\nonumber
%			\end{eqnarray}
%		\end{enumerate}
%		Con esto concluimo que $(G,\cdot)$ es un grupo, y además, como $G\subseteq \mathbb{Z}_r^*$, entonces $(G,\cdot)$ es subgrupo de $(\mathbb{Z}_r^*,\cdot)$.	
%	\end{proof}
	
	Habiendo definido $G$, nos gustaría analizar su cardinalidad. Sea $|G| = t$. Sabemos que el grupo multiplicativo generado por $n$ en módulo $r$ se define de la siguiente forma \comentario{citar apendice?}:
	\begin{eqnarray}
		\langle n\rangle _r &:=& \{n^i\modl r \tq i\in \mathbb{N}\}\nonumber\\
		&=&\{n^1\modl r,n^2\modl r,...,n^{O_r(n)}\modl r \}\nonumber\\
		&=&\{(\frac{n}{p})^1\cdot p^1\modl r,(\frac{n}{p})^2\cdot p^2\modl r,...,(\frac{n}{p})^{O_r(n)}\cdot p^{O_r(n)}\modl r \}\nonumber
	\end{eqnarray}
Sabemos que como $n\in G$, entonces $(\langle n\rangle_r,\cdot)$ es subgrupo de $(G,\cdot)$	(véase \ref{prop-generado}).
	%Como todos los elementos de $\langle n\rangle _r$ son de la forma $(\frac{n}{p})^i\cdot p^i\modl r$, entonces $\langle n\rangle _r\subseteq G$, y es más, se puede demostrar que $(\langle n\rangle _r,\cdot)$ es un subgrupo de $(G,\cdot)$. Esta demostración queda propuesta como ejercicio para el lector.
%	\begin{enumerate}
%		\item el elemento identidad es el $1$
%		\item $\langle n\rangle _r$ es cerrado bajo la operacion $\cdot$
%		\item para cualquier $a$ en $\langle n\rangle _r$, tenemos que $a$ es coprimo de $r$. Es más, si $a = n^i\modl r$, y tomamos $n^{O_r(n)-(i \modl O_r(n))}\modl r$, nos queda
%		\begin{eqnarray}
%			& &(n^i\modl r)\cdot (n^{O_r(n)-(i \modl O_r(n))})\modl r\nonumber\\
%			&\equiv &n^i\cdot n^{O_r(n)-(i \modl O_r(n))}\modl r,\hspace{0.5cm}\text{y como } i = k\cdot O_r(n) + (i\modl O_r(n)),\text{ con }k\in \mathbb{Z}\nonumber\\
%			&\equiv &n^{k\cdot O_r(n) + i\modl O_r(n) + O_r(n)-(i \modl O_r(n))}\modl r\nonumber\\
%			&\equiv &n^{(k+1)\cdot O_r(n)}\modl r\nonumber\\
%			&\equiv &(n^{O_r(n)})^{(k+1)}\modl r\nonumber\\
%			&\equiv &1 \modl r \nonumber			  
%		\end{eqnarray}
%	\end{enumerate}\comentario{el ..y como i=... de la segunda linea del eqnarray no me convence}
	Aplicando el teorema de Lagrange para grupos finitos, podemos afirmar que $|\langle n\rangle _r| = O_r(n)$ divide a $|G|=t$, y por lo tanto, $O_r(n)\leq t$. Como habí­amos escogido $r$ tal que $\log ^2n <O_r(n)$, entonces concluimos que $\log ^2 n <t=|G|$.

	Para definir el segundo grupo, vamos a utilizar los conceptos de raiz de la unidad, cuerpo de descomposición y cuerpo ciclotómico (ver definición \ref{c. de descomposicion, ciclotomico def}).
	%definiremos de forma secuencial algunos conceptos básicos de polinomios ciclotómicos sobre cuerpos finitos. %Esto lo haremos definiendo secuencialmente hasta llegar a lo que ocuparemos.	
%		Comenzaremos hablando sobre cuerpos y sus extensiones. Habíamos visto que dado un cuerpo $K$ y un polinomio no constante $p(X)$ sobre $K$, el cuerpo de descomposición es la menor extensión del cuerpo $K$ que contiene todas las raí­ces de $p(X)$. En particular, para un cuerpo $K$ y un polinomio de la forma $p(X)=X^r-1$ lo anterior es válido, y se denomina el $r$-ésimo cuerpo ciclotómico, el cual lo denotaremos como $K^{(r)}$.
%		A las raí­ces de $p(X)=X^r-1$ en $K^{(r)}$ se les llama las $r$-ésimas raí­ces de la unidad sobre $K$. Este conjunto se puede escribir como $E^{(r)}$. Por ejemplo, en los números complejos $E^{(r)}=\{e^{\frac{2i k\pi }{r}}\tq k\in \mathbb{N}\}$.\comentario{esta bien si escribo esto asi? Esto lo uso a continuación para demostrar que es ciclico} %y podemos ver que efectivamente $E^{(r)}$ contiene únicamente estas raí­ces.		
%\comentarioin{VER QUE SACAR DE ESTE PARRAFO. TODO ESTE PARRAFO REVISAR: no usar primitivas en los complejos, sino que usar primitivas en general y teoremas introducidos en los apendices A y B. Partir del hecho que $w^1$ es una primitiva, ya que genera todas las demás}
Supongamos que estamos trabajando en un cuerpo finito $K$ de caracterí­stica $p$. Por la definición \ref{c. de descomposicion, ciclotomico def} sabemos que $K^{(r)}$ es un cuerpo y contiene a todas las raíces de la unidad del polinomio $p(X)=X^r-1$ sobre $K[X]$, es decir $E^{(r)}\subseteq K^{(r)}$ (recordar que $E^{(r)}=\{\eta_1,...,\eta_r\}$, donde cada $\eta_i$ es una raiz de la unidad).\comentarioin{este conjunto tiene $r$ elementos? contando cada raiz de multiplicidad mayor que uno como elementos distintos}
Es fácil ver que la estructura $(E^{(r)},\cdot)$ es un subgrupo de $(F\setminus \{\0\},\cdot)$ (la demostración queda propuesta para el lector). Por lo tanto, el teorema \ref{subgrupo de cuerpo es ciclico} nos dice que $(E^{(r)},\cdot)$ es cíclico, es decir, siempre existe al menos una $r$-ésima raiz de la unidad $\eta\in E^{(r)}$ tal que $\langle\eta\rangle = E^{(r)}$. A aquellas $r$-ésimas raíces de la unidad que cumplan con la propiedad anterior se les dice primitivas $r$-ésimas raíces de la unidad. Estas generan todas las $r$-ésimas raíces de la unidad.
%, $r$ un entero positivo, ambos distintos de 1, y $\MCD(r,p) = 1$. Si $p$ no divide a $r$, entonces $E^{(r)}$ es un grupo cí­clico respecto a la multiplicación en $K^{(r)}$. Para ver esto, basta que una raí­z pueda generar todas las demás, y que $(E^{(r)},\cdot)$ sea un grupo. Sea $w = e^{\frac{2i\pi }{n}}$. Podemos ver que $w$ pertenece a $E^{(r)}$, y que $E^{(r)}=\{w^0,w^1,...,w^{n-1}\}$. Con esto podemos ver que es cí­clico. El hecho de que $E^{(r)}$ es un grupo, se deja propuesto como ejercicio para el lector. El generador de este grupo se llama primitiva $r$-ésima raí­z de la unidad sobre $K$ (esta raí­z está en $K^{(r)}$ pero no necesariamente en $K$). Las primitivas $r$-ésimas raí­ces de la unidad del $r$-ésimo cuerpo ciclotómico son de la forma $\eta = e^{\frac{2\pi i k}{r}}$, en donde $\MCD(r,k)=1$.\comentarioin{aca creo que estamos abusando de los isomorfismos entre cuerpos FINITOS} 
\begin{afirmacion}\label{afirmacion relacion primitivas}
Sea $\eta\in E^{(r)}$. Entonces $\eta$ es primitiva $r$-ésima raiz de la unidad si y solo si $\eta^s$ también es primitiva $r$-ésima raí­z de la unidad, para todo $s$ con $\MCD(s,r)=1$\comentarioin{Tengo que argumentar que si parto de una raiz primitiva, entonces llego a todas las raices primitivas}
\comentarioin{Si tengo una raiz primitiva, entonces esa raiz  se va a ver como $\eta$ a la algo}
\end{afirmacion}
\begin{proof}
Para el caso en que $r=1$ esto se cumple trivialmente, es por eso que supondremos que $r>1$. Sea $s$ tal que  $\MCD(s,r) = 1$ y $\eta$ una $r$-ésima raiz de la unidad. Para el sentido de izquierda a derecha supongamos que $\eta$ es una primitiva $r$-ésima raí­z de la unidad, es decir el orden multiplicativo de $\eta$ es $r$, y supongamos por contradicción que $\eta^s$ no lo es. Esto quiere decir que el orden de $\eta^s$ es igual a $k<r$. Observemos que 
\begin{eqnarray*}
	1 =  (\eta^{s})^{k}=\eta^{sk} &=& \eta^{\alpha r + \beta},\quad\quad\text{con $sk = \alpha r + \beta$ y $0\leq \beta<r$}\\
	&=&  \eta^{\alpha r}\cdot \eta^{\beta}\\
	&=& (\eta^{r})^{\alpha}\cdot \eta^{\beta}\\
	&=& 1^{\alpha}\cdot \eta^{\beta}\\
	&=& \eta^{\beta}
\end{eqnarray*} 
Lo anterior es una contradicción, ya que $0<\beta<s$ (como $r\neq 1$ entonces $\eta\neq 1$) y $s$ era el orden multiplicativo de $\eta$. De esta forma concluimos el primer sentido de la demostración. Para el sentido contrario supongamos que $\eta^s$ es primitiva $r$-ésima raiz de la unidad pero $\eta$ no lo es. Luego, $\eta$ tiene orden multiplicativo $k< r$. De esta forma tenemos que $$(\eta^{s})^k \ = \ (\eta^{k})^s \ = \ 1^s \ = \ 1$$
Lo anterior es una contradicción ya que habíamos dicho que el orden de $\eta^{s}$ era $r$. Con esto concluimos la demostración.
\end{proof} 
\comentarioin{Bernardo 8/11/20}
\comentarioin{Aca debo decir que $E^{(r)}$ es cíclico y por ende existe una primitiva} 
Como $E^{(r)}$ es cíclico entonces siempre va a existir una primitiva $r$-ésima raiz de la unidad $\eta$. 
Así, como $\eta$ genera $E^{(r)}$, entonces cualquier otra primitiva $r$-ésima raiz de la unidad es una potencia de $\eta$. Es más, por la afirmación \ref{afirmacion relacion primitivas} sabemos que el conjunto de todas las raíces de la unidad es $\{\eta^{s}\mid \MCD(s,r)=1\}$.
De esto podemos concluir que existen $\phi (r)$ primitivas $r$-ésimas raí­ces de la unidad.
	El siguiente lema será útil para la demostración:
	
	\begin{lemma} 
	  %\comentarioin{STAND BY: estoy casi seguro que no necesitamos la dirección de izquierda a derecha, que es la que necesita isomorfismos}
	  \comentarioin{Necesitamos la dirección $\Rightarrow$}
          
		Sea $\eta$ una primitiva $r$-ésima raí­z de la unidad arbitraria. Luego, $\eta$ es un elemento de $ F_{p^k} $ si, y solo si $ p^k\equiv 1$ en módulo $r$.
	\end{lemma}
	La demostración se hará en 2 pasos:\comentario{esto no se si se ve bien (enumerate)}
	\begin{enumerate}
		\item $\eta$ pertenece a $F_{p^k}$ si, y solo si $\eta^{p^k}=\eta$
		\begin{proof}
			\text{ }\\
			"$\Rightarrow$"\\
			Sabemos que 
			\begin{eqnarray}
				& &\eta \in F_{p^k}\nonumber\\
				&\Rightarrow &\eta^{p^k-1} = 1, \hspace{0.5cm} \text{por el orden del conjunto}\nonumber\\
				&\Rightarrow &\eta^{p^k}=\eta
			\end{eqnarray}
			"$\Leftarrow$"\\
			Como los cuerpos finitos del mismo orden son isomorfos\comentario{ojo con esta propiedad}, entonces $\eta$ puede pertenecer a un cuerpo de orden múltiplo de $r$ (ya que el orden de $\eta$ es $r$). Si $\eta^{p^k}=\eta$, esto quiere decir que el orden de $\eta$ divide a $p^k-1$, entonces $\eta \in F_{p^k}$. 
		\end{proof}
		
		\item $\eta^{p^k}=\eta\Leftrightarrow p^k\equiv 1 \modl r$
		\begin{proof}
			\text{ }\\
			"$\Rightarrow$"\\
			Sabemos que $\eta$ es primitiva $r$-ésima raí­z de la unidad. Esto implica que $\eta$ genera todas las otras raí­ces de $X^r-1$ en un ciclo de largo $r$, en donde $\eta^1=\eta^{cr+i}$, para $c\in \mathbb{N}$ y $i=1$, ya que para cualquier $c_1\in \mathbb{N}$, si existiera un  $i_1\neq 1$ con $0\leq i_1\leq r-1$ tal que $\eta=\eta^{c_1r+i_1}$, tendrí­amos que $\eta = \eta^{c_1r+i_1} =(\eta^{r})^{c_1}\cdot\eta^{i_1}=1^{c_1}\cdot\eta^{i_1}=\eta^{i_1}$, y luego, el orden de $\eta$ serí­a $i_1-1<r$. Además por la hipótesis, $\eta^1=\eta^{p^k}$, luego para un $\hat c\in \mathbb{N}$, tenemos que 
			\begin{eqnarray}
				& &p^k=\hat c r+1\nonumber\\
				&\Rightarrow &p^k-1 = \hat c r\nonumber\\
				&\Rightarrow &p^k-1\equiv 0\modl r\nonumber\\
				&\Rightarrow &p^k\equiv 1\modl r\nonumber
			\end{eqnarray}	
			"$\Leftarrow$"\\
			Como $\eta$ es $r$-ésima raí­z de la unidad, entonces $\eta^r = 1$. Sabemos que $p^k\equiv 1 \modl r \Leftrightarrow p^k-1 = cr$, para algún $c\in\mathbb{N}$. Luego,
			\begin{eqnarray}
				\eta^{p^k}&=& \eta^{p^k-1}\cdot\eta\nonumber\\
				&=&\eta^{cr}\cdot\eta\nonumber\\
				&=&(\eta^r)^c\cdot \eta \nonumber\\
				&=& 1^c\cdot\eta \nonumber\\
				&=& \eta\nonumber
			\end{eqnarray}			 					
		\end{proof}
	\end{enumerate}
	Con estas 2 partes, ya que la doble implicancia es transitiva, concluimos que 
	\begin{eqnarray}
		\eta \in F_{p^k} &\Leftrightarrow& p^k\equiv 1 \modl r\label{eq:propiedad}
	\end{eqnarray}	 
	
	\begin{definition}%\comentario{mismo comentario que la definicion anterior(apendice) o a lo mejor cambiarle el nombre de definicion}
		Sea $F_p$ un cuerpo finito de caracterí­stica $p$. Se define el $r$-ésimo polinomio ciclotómico $Q_r(X)$ sobre $F_p$ de la siguiente manera:
		\comentarioin{Es posible que este polinomio no se pueda representar en este cuerpo?}
		\begin{eqnarray}
			Q_r(X) &=& \prod_{s=1,\MCD(r,s)=1}^{r}(X-\eta^s)\nonumber
		\end{eqnarray}
		En donde $\eta$ es primitiva $r$-ésima raí­z de la unidad. 
		Como $\eta$ es primitiva $r$-ésima raí­z de la unidad, entonces $\eta^s$, con $\MCD(s,r)=1$ también lo es, entonces también podemos escribir el polinomio ciclotómico de la siguiente forma: 
		\begin{eqnarray}
			Q_r(X) &=& \prod_{i=1}^{\phi (r)}(X-\eta_i)\nonumber
		\end{eqnarray}
		En donde $\eta_i$ es primitiva $r$-ésima raí­z de la unidad.
	\end{definition} 	
		
	Claramente, $Q_r(X)$ divide a $X^r -1$, ya que el $Q_r(X)$ es mónico y todas sus raí­ces son raí­ces de $X^r-1$. Es más, todas sus raí­ces son primitivas $r$-ésima raí­ces de la unidad.\\
	A continuación demostraremos que el polinomio $Q_r(X)$ se factoriza en factores irreducibles $Q_r(X)=h_1(X)\cdot h_2(X)\cdots h_m(X)$ sobre $F_p$ con $deg(h_i(X))>1$, para $1\leq i \leq m$. 
	\begin{proof}
		Supongamos por contradicción que para algún $j\in \{1,...,m\}$, se tiene que $deg(h_j(X))=1$ sobre $F_p$. Esto implica que pudimos reducir un factor de $Q_r(X)$ a $h(X) = (X-\eta)$, en donde $\eta$ es primitiva $r$-ésima raí­z de la unidad. Por otro lado, sabemos por $\eqref{eq:propiedad}$ que $\eta \in F_{p^k} \Leftrightarrow p^k\equiv 1 \modl r$.\comentarioin{efectivamente aca solo necesitamos la implicancia de izquierda a derecha y nos ahorramos usar los isomorfismos entre cuerpos finitos} El $k$ más pequeño que satisface el lado derecho de la ecuación anterior es $k = O_r(p)$, por definición de orden multiplicativo. También podemos afirmar que $h(X)=(X-\eta)$ es reducible en $F_p$ si y solo si $\eta\in F_{p}=F_{p^{1}}$, ya que de otra forma no lo podrí­amos haber reducido. Esto implica que $\eqref{eq:propiedad}$ se cumple con $k=1$, lo que implica que $O_r(p)=1$, por definición de orden multiplicativo.($\rightarrow\leftarrow$).
	\end{proof}
	Hemos demostrado que $Q(X)$ se puede factorizar en factores irreducibles de grado mayor a 1 sobre $F_p$.\\
	Otra propiedad que demostraremos sobre los polinomios ciclotómicos es la siguiente:
	\begin{eqnarray}
		\text{Si }h(X)\text{ es un factor de }Q_r(X)&\Rightarrow &h(X)\nodiv X^s-1 \text{ para }s<r.\label{eq:final4.7}
	\end{eqnarray} 
	Esta propiedad nos va a ser útil un poco más adelante.\comentario{en verdad no nos basta esta propiedad para el inicio del lema que viene, pero es bueno saberlo por mientras}	 

	\begin{proof}
		Supongamos por contradicción que para algún factor de $Q_r(X)$ se tiene que $h(X)\divi  X^s-1$ para algún $s<r$. Sea $h(X)$ este factor. Luego, $h(X)= (X-\eta_1)\cdot(X-\eta_2)\cdots(X-\eta_j)$ con $j\in\mathbb{N}$ y $\eta_i$ primitiva $r$-ésima raí­z de la unidad para $1\leq i\leq j$. Como $h(\eta_1) = 0$, entonces  $\eta_1^s-1 = 0$, ya que $h(X)\divi  X^s-1$. Esto implica que $\eta_1^s=1$, es decir, el orden multiplicativo de $\eta_1$ es $s<r$($\rightarrow\leftarrow$).
	\end{proof}
	A continuación, se define el segundo conjunto, para el cual se introdujeron las definiciones y propiedades anteriores sobre polinomios ciclotómicos:
	\begin{definition}
		Sea $h(X)$ uno de los factores irreducibles de $Q_r(X)$ sobre $F_p$.
		\begin{eqnarray}
			\mathcal{G} &=& \{x\modulohp\tq x\in P\}\nonumber 
		\end{eqnarray}
		
	\end{definition}
	Este conjunto dotado de la multiplicación es un grupo generado por los elementos $X,X+1,...,X+\ell $ en el cuerpo $F=F[X]/(h(X))$, y es un subgrupo de $F$. La demostración se deja como ejercicio para el lector.\comentario{demostracion de subgrupo propuesta para el lector}\\
	El siguiente lema muestra una cota inferior del tamaño de $\mathcal{G}$.
	\begin{lemma}
		(Hendrik Lenstra Jr.) $|\mathcal{G}|\geq {{t+\ell }\choose{t-1}}$
	\end{lemma}
	
	\begin{proof}
		Lo primero que hay que notar, es que como $h(X)$ es un factor del polinomio ciclotómico $Q_r(X)$, entonces $X$ es una primitiva $r$-ésima raí­z de la unidad en $F$. Esto se puede demostrar en 2 pasos:
		\begin{enumerate}
			\item \underline{$X$ es $r$-ésima raí­z de la unidad en $F$}\\
			Mostrar esto es equivalente a mostrar que $X^r\equiv 1\modulohp$.\\
			Sabemos por la construcción del polinomio ciclotómico que $Q_r(X)\divi  X^r-1$. Además, ya que $h(X)$ es un factor de $Q_r(X)$, entonces $h(X)\divi  Q_r(X)$. De esto podemos concluir que 
			\begin{eqnarray}
				& &h(X)\divi  Q_r(X)\nonumber\\
				&\Leftrightarrow &\exists q(X) \text{ tal que }X^r-1 = h(X)q(X)\modl p\nonumber\\
				&\Leftrightarrow &X^r -1 \equiv 0 \modulohp\nonumber\\
				&\Leftrightarrow &X^r  \equiv 1 \modulohp\nonumber\\
				&\Leftrightarrow &X \text{ es }r\text{-ésima raí­z de la unidad}\nonumber
			\end{eqnarray}	
			\item \underline{El orden multiplicativo de $X$ en $F$ es $r$}\\
			Supongamos que $X^s\equiv 1 \modulohp$ para algún $s<r$.
			Por $\eqref{eq:final4.7}$ podemos afirmar que cada factor $h(X)$ de $Q_r(X)$ no puede ser un factor del polinomio ciclotómico $Q_s(X)$, con $s<r$.
			
			\demostrar{Aca falta demostrar. Con $\eqref{eq:final4.7}$ no nos basta para demostrarlo. Ya que $\modl p$ no implica sin modulo.}	
		\end{enumerate}
		Ahora demostraremos que 2 polinomios distintos de grado menor a $t$ en $P$ se mapean a distintos elementos de $\mathcal{G}$.\\
		Sean $f(X),g(X)\in \mathcal{G}$ tales que $deg(f(X))<t$ y $deg(g(X))<t$. Por contradicción, supongamos que se mapean al mismo elemento en $F$. Esto es equivalente a decir que 
		\begin{eqnarray}
			f(X)&\equiv & g(X)\modulohp\nonumber
		\end{eqnarray} 
		Sea $m\in I$. En $F$ elevamos la equivalencia anterior a $m$ y tenemos que
		\begin{eqnarray}
			[f(X)]^m&\equiv &[g(X)]^m\modulohp\label{eq:2}
		\end{eqnarray}			    
		Sabemos que $m$ es introspectivo para $f(X)$:
		\begin{eqnarray}
			&\Leftrightarrow &[f(X)]^m \equiv f(X^m) \modulop\nonumber\\
			&\Leftrightarrow &[f(X)]^m - f(X^m) \equiv (X^r-1)q(X) \modl p \equiv h(X)q_1(X)q(X) \modl p\equiv h(X)q_2(X) \modl p \nonumber\\
			&\Leftrightarrow &[f(X)]^m \equiv f(X^m) \modulohp\nonumber
		\end{eqnarray}
		Análogamente, esto se puede hacer para $g(X)$.\\
		Luego, de $\eqref{eq:2}$ obtenemos que $f(X^m) \equiv g(X^m) \modulohp$. De esto, podemos decir que $X^m$ es raí­z de $Q(Y)=f(Y)-g(Y)$ en $F$, para cualquier $m\in G$ (ya que los $m$ los habí­amos tomado de $I$). Como $\MCD(m,r)=1$, ya que $G\subseteq \mathbb{Z}_r^*$, entonces $X^m$ es primitiva $r$-ésima raí­z de la unidad $\forall m \in G$ (ya que $X$ lo era y lo estamos elevando a un coprimo de $r$.\\
		Como $|G| = t$, entonces existen $t$ soluciones distintas para $Q(Y)$ (para $m_1\neq m_2 \in G$ se tiene que $X^{m_1}\neq X^{m_2}$, ya que el orden de $X$ es $r>m_1,m_2$) en $F$. Esto contradice que $deg(Q(Y)<t$ ya que estamos en un cuerpo \comentario{Aca usamos que sobre cuerpos "todo funciona", no se si hay que poner esto en alguna parte)}($\rightarrow\leftarrow$).\\
		Luego, si $f(X)\neq g(X)$ en $P$, ambos de grado menor a $t$, entonces $f(X)\neq g(X)$ en $F$.\\\\
		También notemos que como $\langle n\rangle _r\subseteq G\subseteq \mathbb{Z}_r^*$, entonces $O_r(n)=|\langle n\rangle _r|\leq |\mathbb{Z}_r^*|=\phi(r)$. Luego, $r\geq\phi(r)\geq O_r(n)>\log ^2 n$, y por ende $\sqrt{r}>\log  n$. De todo esto, podemos decir que $\ell =\lfloor \sqrt{\phi(r)}\log  n\rfloor < \sqrt{r}\log  n<\sqrt{r}\sqrt{r}=r$. Como $r<p$, luego $\ell <p$\comentario{Esto hay que ponerlo antes del lema A}. Así­ podemos afirmar que si $i\neq j$ y $1\leq i,j\leq l$, entonces $i\neq j$ en $F_p$ y por lo tanto $i\neq j$ en $F$.\\
		También sabemos que $deg(h(X))>1$, luego $X,X+1,...,X+\ell $ son todos distintos en $F$ y no trivialmente $0$\comentario{Que no sea trivialmente 0 no se si nos sirve, ya que son de grado $1<t$ ($O_r(p)\leq t$), entonces como son distintos en $P$ seran distintos en $\mathcal{G}$}. Esto nos sirve para demostrar que existen al menos ${t+\ell }\choose{t-1}$ polinomios distintos de grado $<t$ en $P$. Podemos demostrarlo por inducción:
		\begin{itemize}
			\item \underline{Caso base:}\\
			Para $t=2$ \comentario{También funciona para $t=1$}, tengo que ${{2+\ell }\choose{2-1}}={{2+\ell }\choose{1}}=\ell +2$. Sabemos que tenemos al menos $\ell +1$ polinomios de grado $1$ y $\ell +1$ polinomios de grado 0. Entonces tenemos al menos $2\ell +2$ polinomios de grado $<2$. Como $\ell +2< 2\ell +2$ para nuestro $\ell $, entonces se cumple la propiedad para el caso base.
			\item \underline{Hipótesis de inducción:}\\
			Se cumple la propiedad para $t-1$. Es decir, hay ${{t+\ell -1}\choose{t-2}}$ polinomios distintos de grado $<t-1$ en $P$.
			\item \underline{Tesis de inducción:}\\
			Lo primero que queremos ver es cuántos polinomios distintos de grado $t-1$ podemos formar con los $\ell +1$ polinomios de grado 1. Para esto, usaremos el problema de las $N$ pelotas sin etiquetar y $M$ sacos etiquetados: de cuántas formas distintas se pueden asignar las pelotas a los sacos. La equivalencia con este problema es que cualquier polinomio de grado $t-1$ va a ser de la forma $X^{e_0}(X+1)^{e_1}\cdots(X+\ell )^{e_\ell }$, con $e_0+e_1+...+e_\ell =t-1$. De esta forma, podemos decir que al saco $X$ se le asignan $e_0$ pelotas, al saco $X+1$ se le asignan $e_1$ pelotas, y así­ hasta llegar al saco $X+\ell $.\\
			Para esta problema, la cantidad de asignaciones distintas es ${{N+M-1}\choose{N}}$, y para nuestro caso $N=t-1$ y $M=\ell +1$. Usando la fórmula descrita anteriormente concluimos que podemos formar un total de ${{t+\ell -1}\choose{t-1}}$ polinomios de grado $t-1$ a partir de los $\ell +1$ polinomios de grado $1$.\\
			Por hipótesis de inducción, tengo al menos ${{t+\ell -2}\choose{t-2}}$ polinomios distintos de grado $<t-1$, y si sumo los polinomios de grado $t-1$ tengo:
			\begin{eqnarray}
			& &{{t+\ell -1}\choose{t-2}}+{{t+\ell -1}\choose{t-1}}\nonumber\\
			&= &\frac{(t+\ell -1)!}{(\ell +1)!(t-2)!}+\frac{(t+\ell -1)!}{\ell !(t-1)!}\nonumber\\
			&= &\frac{(t+\ell -1)!}{(\ell +1)!(t-2)!}+\frac{(t+\ell -1)!}{(\ell +1)!(t-2)!}\frac{\ell +1}{t-1}\nonumber\\
			&= &\frac{(t+\ell -1)!}{(\ell +1)!(t-2)!}(1+\frac{\ell +1}{t-1})\nonumber\\
			&= &\frac{(t+\ell -1)!}{(\ell +1)!(t-2)!}(\frac{t-1+\ell +1}{t-1})\nonumber\\
			&= &\frac{(t+\ell -1)!}{(\ell +1)!(t-2)!}\frac{t+\ell }{t-1}\nonumber\\
			&= &\frac{(t+\ell )!}{(\ell +1)!(t-1)!}\nonumber\\
			&=& {{t+\ell }\choose{t-1}}\nonumber
			\end{eqnarray}
			Luego, la propiedad se cumple para $t$.
		\end{itemize}
		Con esto podemos ver que existen al menos ${{t+\ell }\choose{t-1}}$ polinomios distintos de grado $<t$ en $P$, y usando que polinomios distintos de grado $<t$ en $P$ se mapean a polinomios distintos en $\mathcal{G}$, podemos decir que existen al menos ${{t+\ell }\choose{t-1}}$ polinomios distintos en $\mathcal{G}$. Luego, $|\mathcal{G}|\geq {{t+\ell }\choose{t-1}}$.
		
	\end{proof}
	Como $n$ no es una potencia de $p$ (ya que si lo fuera habrí­a retornado en el paso $1$), también podemos acotar por arriba al conjunto $\mathcal{G}$.
	\begin{lemma}
		$|\mathcal{G}|\leq n^{\sqrt{t}}$.
	\end{lemma}
	\begin{proof}
		Consideremos el siguiente subconjunto de $I$:
		\begin{eqnarray}
			\hat{I} &=&\{(\frac{n}{p})^{i}\cdot p^{j}\tq 0\leq i,j\leq \lfloor \sqrt{t}\rfloor\}\nonumber 
		\end{eqnarray}
		Primero vamos a demostrar que si $0\leq i,j,a,b\leq \lfloor \sqrt{t}\rfloor$, con $i\neq a$ o $j\neq b$, entonces $(\frac{n}{p})^{i}\cdot p^{j} \neq (\frac{n}{p})^{a}\cdot p^{b}$. Supongamos por contradicción que $(\frac{n}{p})^{i}\cdot p^{j} = (\frac{n}{p})^{a}\cdot p^{b}$ para $i,j,a,b$ como los de antes. Podemos notar que si $i=a$ entonces es inmediato que $j=b$, luego necesariamente se cumple que $i\neq a$ y $j\neq b$. Supongamos sin pérdida de generalidad que $a>i$. Para que la igualdad se cumpla, se debe cumplir que $b<j$. Como $n$ no es potencia de $p$, y $p\divi  n$, $n$ tiene que ser de la forma $n=cp^{k+1}$, con $k\geq 0$ y $1<c<p$. Luego,
		\begin{eqnarray}
			& &(\frac{n}{p})^{i}\cdot p^{j} = (\frac{n}{p})^{a}\cdot p^{b}\nonumber\\
			&\Leftrightarrow &(cp^k)^{i}\cdot p^{j} = (cp^k)^{a}\cdot p^{b}\nonumber\\
			&\Leftrightarrow &(cp^k)^{a-i} = p^{j-b}\nonumber\\
			&\Leftrightarrow &(\frac{n}{p})^{a-i} = p^{j-b}\nonumber\\
			&\Leftrightarrow &n^{a-i} = p^{j-b+a-i}\nonumber\\
			&\Leftrightarrow &n^{k_1} = p^{k_2}\hspace{0.5cm}k_1,k_2>0\nonumber
		\end{eqnarray}
		Ya que la descomposición en factores primos es única tendrí­amos que $n$ es potencia de $p$ ($\rightarrow\leftarrow$).\\\\
		Con esto podemos afirmar que $|\hat{I}|=(\lfloor \sqrt{t}\rfloor+1)^2>t$. Como $\hat{I}\subseteq I$ y $|G|=t$, entonces deben existir al menos un par de números distintos $m_1,m_2\in \hat{I}$ tales que son iguales en $G$. Esto es equivalente a decir que $m_1\equiv m_2 \modl r$. Por otro lado, si $m=c\cdot r + m\hspace{0.1cm}mod\hspace{0.1cm} r$, entonces 
		\begin{eqnarray}
			X^m &\equiv &	X^{c\cdot r + m\hspace{0.1cm}mod\hspace{0.1cm} r} \modulox\nonumber\\
			&\equiv & X^{c\cdot r}\cdot X^{m\hspace{0.1cm}mod\hspace{0.1cm} r}  \modulox \nonumber\\
			&\equiv & X^{m\hspace{0.1cm}mod\hspace{0.1cm} r}\modulox\nonumber
		\end{eqnarray}
		Con esto podemos afirmar que 
		\begin{eqnarray}
			& & X^{m_1}\equiv X^{m_2} \modulox\nonumber\\
			&\Rightarrow & X^{m_1}\equiv X^{m_2} \modulop\label{eq:4.8}
		\end{eqnarray}
		Sea $f(X)\in P$. Como $m_1,m_2\in \hat{I}\subseteq I$, entonces son introspectivos para $f(X)$. Luego,
		\begin{eqnarray}
			[f(X)]^{m_1} &\equiv & f(X^{m_1}) \modulop,\hspace{0.5cm}\text{y por }\eqref{eq:4.8}\nonumber\\
			&\equiv & f(X^{m_2}) \modulop\nonumber\\
			&\equiv &[f(X)]^{m_2} \modulop\nonumber		
		\end{eqnarray}
		Esto implica que $[f(X)]^{m_1}\equiv [f(X)]^{m_2}\modulohp$ (es decir en $F$). Luego, $f(X)\modulohp = f^*(X)\in \mathcal{G}$ es solución de $Q^*(Y)= Y^{m_1}-Y^{m_2}$ en $F$\comentario{OJO ACA DECIA DIRECTAMENTE QUE F(X) ESTABA EN $\mathcal{G}$ PERO CREO QUE NO ES ASI}. Como $f(X)$ es un elemento arbitrario de $P$, entonces se cumple para todos los polinomios de $P$. De esta forma, cada elemento de $\mathcal{G}$ es solución de $Q^*(Y)$, luego, este polinomio tiene al menos $|\mathcal{G}|$ distintas raí­ces en $F$, por lo tanto, $|\mathcal{G}|\leq deg(Q^*(Y))$ . Por último, 
		\begin{eqnarray}
			deg(Q^*(Y)) &=& Max\{m_1,m_2\}\nonumber\\
			&\leq& (\frac{n}{p})^{\lfloor \sqrt{t}\rfloor}\cdot p^{\lfloor \sqrt{t}\rfloor}\nonumber\\
			&=&n^{\lfloor \sqrt{t}\rfloor}\nonumber
		\end{eqnarray}
		De esto tenemos que $|\mathcal{G}|\leq n^{\lfloor \sqrt{t}\rfloor}$, y concluimos la demostración.
		
	\end{proof}
	Finalmente, usando los lemas demostrados anteriormente, podemos demostrar la correctitud.
	\begin{lemma}\ref{te-final6}
		Si el algoritmo con input $n$ retorna PRIMO en el paso 6 del algoritmo, entonces $n$ es primo.
	\end{lemma}
	\begin{proof}
	Supongamos que el algoritmo retornó PRIMO en el paso 6. Usando lo que hemos definido anteriormente, sabemos que $t=|G|$ y $\ell  = \lfloor \sqrt{\phi(r)}\log  n\rfloor$. También, por el Lema 4.7 \comentario{referenciar lema} tenemos que:
	\begin{eqnarray}
		\mathcal{G}&\geq &{{t+\ell }\choose{t-1}}\label{eq:10}
	\end{eqnarray}
	Además, como $t>\log ^2 n$ entonces $t>\sqrt{t}\log  n$, y por ende $t>\mia$. Luego,
	\begin{eqnarray}
		\frac{(t+\ell )!}{(t-1)!} &=& \frac{(\ell +1+t-1)!}{(t-1)!}\nonumber\\
		&=&\frac{(\ell +1+(t-1))(\ell +1+(t-2))\cdots(\ell +1+(\mia+1))(\ell +1+\mia)!}{(t-1)(t-2)\cdots (\mia+1)(\mia)!}\nonumber\\
		&=& \frac{(\ell +1+(t-1))}{(t-1)}\cdot\frac{(\ell +1+(t-2))}{(t-2)}\cdots\frac{(\ell +1+(\mia+1))}{(\mia+1)}\cdot\frac{(\ell +1+\mia)!}{(\mia)!}\nonumber\\
		&\geq & 1\cdot1\cdots1\cdot1\cdot\frac{(\ell +1+\mia)!}{(\mia)!}\nonumber\\
		\frac{(t+\ell )!}{(t-1)!} &\geq & \frac{(\ell +1+\mia)!}{(\mia)!}\nonumber
	\end{eqnarray}
Esto implica que 
\begin{eqnarray}
	& &\frac{(t+\ell )!}{(\ell +1)!(t-1)!} \geq \frac{(\ell +1+\mia)!}{(\ell +1)!(\mia)!}\nonumber\\
	&\Leftrightarrow &{{t+\ell }\choose{t-1}}\geq {{\ell +1+\mia}\choose{\mia}}\label{eq:11}
\end{eqnarray}	
Por otro lado, como $\ell =\lfloor \sqrt{\phi(r)}\log  n\rfloor\geq \mia$, entonces
\begin{eqnarray}
	& &\frac{(\ell +1+\mia)!}{(\ell +1)!}\nonumber\\
	&= &\frac{(\ell +1+\mia)(\ell +\mia)\cdots (\mia+2+\mia)(\mia+1+\mia)!}{(\ell +1)(\ell )\cdots (\mia+2)(\mia+1)!}\nonumber\\
	&= &\frac{\ell +1+\mia}{\ell +1}\cdot\frac{\ell +\mia}{\ell }\cdots\frac{\mia+2+\mia}{\mia+2}\cdot\frac{(\mia+1+\mia)!}{(\mia+1)!}\nonumber\\
	&\geq &1\cdot 1\cdots 1 \cdot\frac{(\mia+1+\mia)!}{(\mia+1)!}\nonumber
\end{eqnarray}
Luego, 
\begin{eqnarray}
	& &\frac{(\ell +1+\mia)!}{(\ell +1)!}\geq \frac{(2\mia+1)!}{(\mia+1)!}\nonumber\\
	&\Rightarrow &\frac{(\ell +1+\mia)!}{(\ell +1)!(\mia)!}\geq \frac{(2\mia+1)!}{(\mia+1)!(\mia)!}\nonumber\\
	&\Leftrightarrow &{{\ell +1+\mia}\choose{\mia}}\geq {{2\mia +1}\choose{\mia}}\label{eq:12}
\end{eqnarray}
	También sabemos que $\mia > \lfloor \log ^2n\rfloor\geq 1$, ya que $n\geq 2$\comentario{esto no se por que es desigualdad estricta, sin embargo es arreglable si comenzamos de n>2, ya que la ultima desigualdad pasaria a ser estricta}, entonces
	\begin{eqnarray}
		{{2\mia +1}\choose{\mia}} &=& \frac{(2\mia +1)!}{(\mia +1)!(\mia)!} \nonumber\\
		&=&\frac{(2\mia+1)(2\mia)\cdots(\mia+3)(\mia +2)(\mia +1)!}{(\mia)(\mia -1)\cdots2\cdot 1}\nonumber\\
		&=&\frac{2\mia+1}{\mia}\cdot\frac{2(\mia -1)+2}{\mia-1}\cdots\frac{2\cdot 2 + (\mia-1)}{2}\cdot\frac{2\cdot 1 +\mia}{1}\nonumber\\
		& &\text{Pero como }\mia >1 \Rightarrow \mia \geq 2\nonumber\\
		&\geq& \frac{2\mia+1}{\mia}\cdot\frac{2(\mia -1)+2}{\mia-1}\cdots\frac{2\cdot 2 + (\mia-1)}{2}\cdot (2+2)\nonumber\\
		&>&	\frac{2\mia}{\mia}\cdot\frac{2(\mia -1)}{\mia-1}\cdots\frac{2\cdot 2 }{2}\cdot (2^2)\nonumber\\
		&=& 2\cdot 2\cdots 2\cdot (2^2),\hspace{1cm}\text{en donde el número } 2 \text{ se repite }\mia-1 \text{ veces}\nonumber\\
		&=& 2^{\mia -1}\cdot 2^2\nonumber\\ 
		&=& 2^{\mia +1}\nonumber
	\end{eqnarray}
	Luego, 
	\begin{eqnarray}
		{{2\mia +1}\choose{\mia}}&>& 2^{\mia +1}\label{eq:13} 
	\end{eqnarray}
	Por último,
	\begin{eqnarray}
		 2^{\mia +1} &\geq &  2^{\lceil \sqrt{t} \log n \rceil}\nonumber\\
		 &\geq &2^{\sqrt{t}\log n}\nonumber\\
		 &=& 2^{\log {n^{\sqrt{t}}}}\nonumber\\
		 &=& n^{\sqrt{t}}\nonumber
	\end{eqnarray}
	Entonces,
	\begin{eqnarray}
		2^{\mia +1} &\geq & n^{\sqrt{t}}\label{eq:14}
	\end{eqnarray}
	Usando $\eqref{eq:10},\eqref{eq:11},\eqref{eq:12},\eqref{eq:13},\eqref{eq:14}$, ya que en $\eqref{eq:13}$ tenemos una desigualdad estricta, nos queda que 
	\begin{eqnarray}
		|\mathcal{G}|&>& n^{\sqrt{t}}\nonumber
	\end{eqnarray}
	Esto es una contradicción ya que en el Lema 4.8 habí­amos demostrado que $|\mathcal{G}|\leq n^{\sqrt{t}}$. \\
	De esta forma, como habí­amos comenzado la demostración de todos los lemas suponiendo que el algoritmo con input $n$ retornó PRIMO en el paso 6, pero $n$ era compuesto, y esto genera una contradicción, entonces $n$ debe ser primo. 
	
	\end{proof}


\bibliographystyle{abbrv}
\bibliography{referencias}

\appendix

\newpage


\section{Preliminares sobre teoría de grupos}
\label{app-grupos}
%\subsection{Definiciones y propiedades básicas}
%\begin{definition}[Grupo]
%Un conjunto $G$ y una función (total) $\circ : G \times
%G \to G$ forman un grupo si:
%\begin{enumerate}
%	\item Para cada $a,b,c \in G$: $(a \circ b) \circ c = a \circ (b
%\circ c)$

%	\item Existe $e \in G$ tal que para cada $a \in G$: $a \circ e = e \circ a = a$

%   \item Para cada $a \in G$, existe $b \in G$: $a \circ b 	= b \circ a = e$
%\end{enumerate}
%\end{definition}
%\comentarioin{Al parecer los nombres derivados de extranjeros se escriben sin mayuscula: http://aplica.rae.es/orweb/cgi-bin/v.cgi?i=huRPfQssalCJLqSk}
\begin{definition}[Grupo, neutro, inverso, grupo abeliano, orden]
Un \emph{grupo} es un par $(G,\circ)$, donde $G$ es un conjunto y $\circ \colon G\times G\to G$ es una función (total) que satisface los siguientes axiomas:
\begin{enumerate}
	\item $\circ$ es asociativa: para todos $a,b,c \in G$ se cumple que $(a \circ b) \circ c = a \circ (b
\circ c)$.

	\item Existe un $e \in G$ (llamado \emph{neutro}) tal que, para todo $a \in G$, se cumple que $a \circ e = e \circ a = a$

   \item Para cada $a \in G$ existe un $b \in G$ (llamado \emph{inverso} de $a$) tal que $a \circ b 	= b \circ a = e$
\end{enumerate}
Decimos que $(G,\circ)$ es un \emph{grupo abeliano} si es un grupo que además cumple la siguiente propiedad:
\begin{enumerate}
	\item[4.] $\circ$ es conmutativa: para todos $a,b\in G$ se cumple que $a\circ b = b\circ a$. 
\end{enumerate}
Al número $|G|$ se le llama el \emph{orden} del grupo $(G, \circ)$. \hfill$\blacksquare$
\end{definition}

Cuando la operación $\circ$ está clara por contexto, es usual referirse al grupo $(G, \circ)$ escribiendo simplemente $G$. 

\begin{proposition}
El elemento neutro de un grupo es único.
\end{proposition}

\begin{proof}
Sea $(G,\circ)$ un grupo, y supongamos que $e, e' \in G$ son ambos neutros. Como $e$ es neutro, tenemos que $e \circ e' = e'$. Pero $e'$ también es neutro, así que $e \circ e' = e$. Luego, por transitividad, $e = e'$.
\end{proof}

En el contexto de la teoría de grupos, usualmente la letra $e$ está reservada para el neutro. Sin embargo, lo ideal es siempre especificar la notación. 

\begin{proposition} \label{inverso_unico}
	Cada elemento de un grupo tiene un único inverso.
	\end{proposition}
	
	\begin{proof}
	Sea $(G,\circ)$ un grupo, $a \in G$ y supongamos que $b, c \in G$ son ambos inversos de $a$. Luego
	\begin{align*}
	b &= b \circ e && e \text{ es neutro} \\
	b &= b \circ \left( a \circ c \right) && c \text{ es inverso de } a \\
	b &= \left( b \circ a \right) \circ c && \text{$\circ$} \text{ es asociativa} \\
	b &= e \circ c && b \text{ es inverso de } A \\
	b &= c && e \text{ es neutro}.
	\end{align*}
	\end{proof}

En virtud de la proposición \ref{inverso_unico}, podemos usar la notación $a^{-1}$ para denotar al (único) inverso del elemento $a$. Para que dicha notación tenga sentido es necesario que el grupo subyacente esté claro por contexto. 
Notemos que $e^{-1} = e$, es decir, el elemento neutro es su propio inverso. En general, puede ocurrir que existan elementos en el grupo distintos al neutro que igualmente tengan esa propiedad.

Como expresiones de la forma $a \circ b \circ c$ no presentan ambigüedad (ya que $\circ$ es asociativa), los paréntesis son opcionales y usualmente solo se incluyen cuando hacen más claro un cálculo o demostración.

\begin{example} \label{ejemplo_Z}
El conjunto $\mathbb{Z}$ de los números enteros forma un grupo con la operación de suma. En este caso, el neutro es el número $0$, y el inverso de un $a \in \mathbb{Z}$ será $-a \in \mathbb{Z}$. \hfill$\blacksquare$
\end{example}

\begin{proposition}[Leyes de cancelación] \label{cancelacion_grupos}
	Sea $(G, \circ)$ un grupo, y sean $a, b, c \in G$. Se cumple que:
	\begin{enumerate}
		\item Si $a \circ c = b \circ c$, entonces $a = b$.
		\item Si $c \circ a = c \circ b$, entonces $a = b$.
	\end{enumerate}
\end{proposition}

\begin{proof}
Probaremos solo la primera afirmación: la segunda se demuestra análogamente. Tenemos que
$$a = a \circ e = a \circ \left( c \circ c^{-1} \right) = \left(a \circ c \right) \circ c^{-1} = \left(b \circ c \right) \circ c^{-1} = b \circ \left( c \circ c^{-1} \right) = b \circ e = b.$$
\end{proof}

\begin{definition}[Potenciación en un grupo]
Sea $(G, \circ)$ un grupo con neutro $e$ y $a \in G$. Definimos $a^0 \coloneqq e$ y, si $n$ es un entero positivo, definimos $a^n \coloneqq \underbrace{a \circ a \circ \cdots \circ a}_{n \text{ veces}}$ y $a^{-n} \coloneqq \underbrace{a^{-1} \circ a^{-1} \circ \cdots \circ a^{-1}}_{n \text{ veces}}$. \hfill$\blacksquare$
\end{definition}

Dejamos de ejercicio al lector verificar que la potenciación tiene las siguientes propiedades.

\begin{prop} Sea $(G, \circ)$ un grupo. Se cumple que:
	\begin{enumerate}
		\item $a^m \circ a^n = a^{m+n}$ para todo $a \in G$ y todos $m, n \in \mathbb{Z}$.
		\item $\left( a^m \right)^n = a^{mn}$ para todo $a \in G$ y todos $m, n \in \mathbb{Z}$.
		\item Si $G$ es abeliano, entonces $\left( a \circ b \right)^{n} = a^n \circ b^n$ para todos $a, b \in G$ y todo $n \in \mathbb{Z}$.
	\end{enumerate}
\end{prop}


\begin{example} \label{grupo_ciclico}
Sea $n$ un entero positivo. Consideremos el conjunto de $n$ elementos $\mathbb{Z}_n \coloneq \left\{ 0, 1, \dots, n-1 \right\}$, y sea $\circ$ la operación de suma módulo $n$, es decir, $a \circ b$ es el resto de dividir $a+b$ en $n$. Entonces $(G, \circ)$ es un grupo abeliano. En efecto, se puede verificar que $\circ$ es asociativa y conmutativa, que $0$ es el elemento neutro, y que el inverso de $a \in \mathbb{Z}_n$ es $n-a \in \mathbb{Z}_n$ (excepto si $a = 0$, que es su propio inverso).
Este grupo tiene una propiedad particular: hay un elemento (el $1$) tal que sus potencias generan todos los elementos del grupo. En otras palabras, todo elemento de $\mathbb{Z}_n$ es una potencia del elemento $1$. Por ejemplo, en $\mathbb{Z}_5$, tenemos que $2 = 1 + 1$, $3 = 1 + 1 + 1$, $4 = 1 + 1 + 1 + 1$ y $0 = 1 + 1 + 1 + 1 + 1$ (considerando la suma en módulo $5$).
\hfill$\blacksquare$
\end{example}

\begin{remark} Notemos que, en el contexto del ejemplo \ref{grupo_ciclico}, sería confuso utilizar la notación $1^4$ para referirnos a la cuarta potencia de $1$, pues lo que queremos representar es el número que se obtiene \textit{sumando} $1$ consigo mismo $4$ veces. En estos casos en los que la operación del grupo se asemeja más a la suma que a la multiplicación, se suele usar la notación aditiva para denotar a las potencias. De esta forma, si $n \in \mathbb{Z}$ y $a$ es un elemento del grupo, escribimos $n a$ en lugar de $a^n$. \hfill$\blacksquare$
\end{remark}


La siguiente definición se refiere a los grupos que viven naturalmente dentro de otros grupos.

%--------------------------------------
\begin{definition}[Subgrupo]
Sea $(G,\circ)$ un grupo. Se dice que un subconjunto $H \subseteq G$ es un \emph{subgrupo} de $(G,\circ)$ si $(H, \circ)$ es un grupo (con la misma operación). \hfill$\blacksquare$
\end{definition}

\begin{remark}
	Todos los subgrupos de un grupo abeliano son abelianos. \hfill$\blacksquare$
	\end{remark}

Para probar que $(H, \circ)$ es un subgrupo de $(G, \circ)$ debemos, incluso antes de verificar los axiomas, comprobar que $H$ sea cerrado bajo la operación binaria $\circ$. En otras palabras, necesitamos que $\circ$ restringida a $H \times H$ tenga recorrido contenido en $H$. 


Notemos que, \textit{a priori}, no sabemos si el neutro y los inversos de un subgrupo coinciden con los del grupo más grande. Demostraremos a continuación que eso efectivamente ocurre.


\begin{proposition}
Sea $(H, \circ)$ un subgrupo del grupo $(G, \circ)$. Las siguientes afirmaciones se cumplen:
	\begin{enumerate}
		\item Si $e_1$ es el neutro en $(G, \circ)$ y $e_2$ es el neutro de $(H, \circ)$, entonces $e_1 = e_2$

		\item Para cada $a \in H$, si $b$ es el inverso de $a$ en $(H, \circ)$, entonces $b$ es el inverso de $a$ en $(G, \circ)$.
	\end{enumerate}
\end{proposition}


\begin{proof} \text{ }
\begin{enumerate}
\item Como $e_1$ es el neutro en $(G, \circ)$, $e_2\in H$ y $H \subseteq G$, se cumple que $e_2 \circ e_1 = e_2$.
Por otro lado, sabemos que $e_2$ es el neutro en $(H, \circ)$, por lo que $e_2 \circ e_2 = e_2$
(notemos que se usa la misma operación en $H$ y en $G$).
De esto concluimos que $e_2 \circ e_1 = e_2 \circ e_2$. Usando la proposición \ref{cancelacion_grupos} concluimos que $e_1 = e_2$.
\item Sea $a^{-1}$ el inverso de $a$ en $(G, \circ)$. Queremos probar que $b = a^{-1}$. Por la primera propiedad de esta proposición, sabemos que $(G, \circ)$ y $(H,
\circ)$ tienen el mismo neutro, que llamaremos $e$. 
Dado que $a^{-1}$ es el inverso de $a$ en $(G, \circ)$, tenemos que $a \circ a^{-1} = e$. Además, dado que $b$ es el inverso de $a$ en $(H, \circ)$, se cumple que $a \circ b = e$. Luego $a \circ a^{-1} = a \circ b$. Usando la proposición \ref{cancelacion_grupos} concluimos que $a^{-1} = b$. \qedhere
\end{enumerate}
\end{proof}

\begin{example} \label{ejemplo_subgrupo}
Sea $m$ un entero positivo. Definimos el conjunto
$m\mathbb{Z} \coloneq \left\{mk \mid k \in \mathbb{Z}\right\}$ de los múltiplos de $m$. Notemos que $m\mathbb{Z}$ es cerrado bajo la operación de suma: si $mk_1$ y $mk_2$ son elementos de $m\mathbb{Z}$, entonces $mk_1 + mk_2 = m(k_1 + k_2) \in m\mathbb{Z}$. Dejamos al lector verificar que $m\mathbb{Z}$ es un subgrupo de $(\mathbb{Z}, +)$. \hfill$\blacksquare$
\end{example}


A continuación hablaremos sobre clases laterales y grupos cocientes. Puesto que todos los grupos que utilizaremos en el documento son abelianos, vamos a agregar esa hipótesis a los teoremas para simplificar las demostraciones. Sin embargo, hacemos notar que toda esa teoría también puede aplicarse a grupos no abelianos si se consideran algunas sutilezas adicionales. 

\begin{definition}[Clases laterales] \label{definicion_clase_lateral}
Sea $(G, \circ)$ un grupo abeliano, y sea $H$ un subgrupo de $G$. Dado un elemento $g \in G$, definimos la \emph{clase lateral} de $g$ bajo $H$ como el conjunto
$$g \circ H \coloneq \left\{ g \circ h \mid h \in H \right\}.$$ Al conjunto de las clases laterales de $G$ bajo $H$ se le denota $\faktor{G}{H}$.
\hfill$\blacksquare$
\end{definition}

Notemos que, según la definición \ref{definicion_clase_lateral}, $H$ en sí mismo es la clase lateral de $e \in G$. En general, ocurrirá que a varios elementos distintos de $G$ les corresponderá la misma clase lateral. Dejamos al lector verificar que la clase lateral de cualquier elemento de $H$ es justamente $H$. También hacemos notar que las clases laterales no necesariamente son subgrupos de $G$.

\begin{prop} \label{lema:rel bin}
Sea $(G, \circ)$ un grupo abeliano, y sea $H$ un subgrupo de $G$. Las clases laterales bajo $H$ forman una partición de $G$.
\end{prop}

\begin{proof}
Definiremos la siguiente relación binaria en $G$: $a \sim b$ si y solo si $b \circ a^{-1} \in H$. Afirmamos que $\sim$ es una relación de equivalencia. En efecto:
\begin{itemize}
\item Sea $a \in G$ arbitrario. Tenemos que $a \circ a^{-1} = e \in H$, ya que $H$ es un subgrupo. Luego $a \sim a$, y $\sim$ es refleja.
\item Sean $a, b \in G$ tales que $a \sim b$. Eso último significa que $b \circ a^{-1} \in H$. Como $H$ es un subgrupo, tenemos que $\left( b \circ a^{-1} \right)^{-1} \in H$ (ya que los subgrupos son cerrados bajo tomar inversos). Notemos que $\left( b \circ a^{-1} \right)^{-1} = a \circ b^{-1}$, pues
$$\left( b \circ a^{-1} \right) \circ \left(a \circ b^{-1} \right) = b \circ \left( a^{-1} \circ a \right) \circ b^{-1} = b \circ e \circ b^{-1} = b \circ b^{-1} = e.$$
Entonces $a \circ b^{-1} \in H$, lo que significa que $b \sim a$. Esto prueba que $\sim$ es simétrica.
\item Sean $a, b, c \in G$ tales que $a \sim b$ y $b \sim c$. Esto significa que $b \circ a^{-1} \in H$ y $c \circ b^{-1} \in H$. Pero $(c \circ b^{-1}) \circ (b \circ a^{-1}) = c \circ a^{-1}$, y sabemos que la operación
$\circ$ es cerrada en $H$. Por lo tanto, $c \circ a^{-1} \in H$. Luego
$a\sim c$, y $\sim$ es transitiva.
\end{itemize}
Ahora veremos que las clases de equivalencia que la relación $\sim$ induce en $G$ son justamente las clases laterales. Para ello, tomemos un $g \in G$ arbitrario, y denotemos por $\left[ g \right]_{\sim}$ a su clase de equivalencia bajo $\sim$. Queremos probar que $g \circ H = \left[ g \right]_\sim$. Veamos que
\begin{align*}
a \in g \circ H \qquad & \Leftrightarrow \qquad \exists\, h \in H \quad a = g \circ h \\
& \Leftrightarrow \qquad \exists\, h \in H \quad a \circ g^{-1} = h \\
& \Leftrightarrow \qquad a \circ g^{-1} \in H \\
& \Leftrightarrow \qquad g \sim a \\
& \Leftrightarrow \qquad a \in  \left[ g \right]_\sim.
\end{align*}
\end{proof}


\begin{prop} \label{cardinalidad_clases_laterales}
Sea $(G, \circ)$ un grupo abeliano finito, y sea $H$ un subgrupo de $G$. Todas las clases laterales bajo $H$ tienen la misma cardinalidad, es decir, $|a \circ H| = |b \circ H|$ para cualesquiera $a, b \in G$.
\end{prop}

\begin{proof}
Vamos a demostrar que todas las clases laterales bajo $H$ tienen cardinalidad $|H|$.

Sea $g \in G$ arbitrario. Queremos probar que $|g \circ H| = |H|$. Definimos la función $f\colon H \rightarrow (g \circ H)$ tal que $f(h) = g \circ h$. Afirmamos que $f$ es una biyección. En efecto: 
\begin{itemize}
\item Sean $h_1, h_2 \in H$ tales que $f(h_1) = f(h_2)$. Entonces $g \circ h_1 = g \circ h_2$. Usando cancelación, obtenemos que $h_1 = h_2$. Luego $f$ es inyectiva.
\item Sea $b \in g \circ H$. Entonces existe un $h \in H$ tal que $b = g \circ h$. Luego $f(h) = b$, y $f$ es sobreyectiva.
\end{itemize}
\end{proof}


El siguiente es un resultado importante en la teoría de grupos finitos, pues establece una relación entre el orden de un grupo y el orden de sus subgrupos. Hacemos notar que el teorema es válido incluso para grupos no abelianos. 

\begin{theorem}[Lagrange]
\label{teo:lagrange}
Si $(G, \circ)$ es un grupo abeliano finito y $H$ es un subgrupo de $G$, entonces $|H|$ divide a $|G|$.
\end{theorem}

\begin{proof}
Por la proposición \ref{lema:rel bin} sabemos que $|G|$ es la suma de las cardinalidades de todas las clases laterales de $G$ bajo $H$. Por otro lado, la proposición \ref{cardinalidad_clases_laterales} nos dice que cada una de dichas clases laterales tiene cardinalidad $|H|$. Luego $$|G| = \left| \faktor{G}{H}\right| \cdot |H|,$$ de lo que se deduce el resultado. 
\end{proof}

\begin{example} \label{ejemplo_lagrange}
Supongamos que $G$ es un grupo abeliano de orden $6$, y queremos encontrar todos sus subgrupos. En primer lugar, tenemos los subgrupos triviales $\{e\} \subseteq G$ y $G \subseteq G$. Lo que nos dice el teorema de Lagrange es que cualquier otro subgrupo necesariamente debe tener orden $2$ o $3$, pues estos son los divisores no triviales de $6$. Así, sabemos de antemano que, por ejemplo, no puede existir un subgrupo de orden $4$.
\hfill$\blacksquare$
\end{example}

Notemos que el teorema \ref{teo:lagrange} no garantiza que para cada divisor de $|G|$ habrá un subgrupo con ese orden. Más adelante veremos un recíproco parcial de este resultado para el caso de los divisores primos de $|G|$ (teorema \ref{teo cauchy}).


A continuación estudiaremos cómo se comporta la operación del grupo con las clases laterales. Veamos el siguiente ejemplo.

\begin{example} \label{ejemplo_clases_laterales}
Continuamos con el ejemplo \ref{ejemplo_subgrupo}. Consideremos el grupo $(\mathbb{Z}, +)$ junto con el subgrupo $4\mathbb{Z}$. Dado $a \in \mathbb{Z}$, tenemos que $a + 4\mathbb{Z}$ es el conjunto de enteros de la forma $a + 4k$ para algún $k \in \mathbb{Z}$. Notemos que, si $r$ es el resto de dividir $a$ en $4$, entonces $a + 4\mathbb{Z} = r + 4\mathbb{Z}$. Por lo tanto, tenemos que
$$\faktor{\mathbb{Z}}{4\mathbb{Z}} = \left\{ 4\mathbb{Z},\, 1+4\mathbb{Z},\, 2+4\mathbb{Z},\, 3+4\mathbb{Z}\right\}.$$
Notemos que, si $a$ es cualquier elemento de $1+4\mathbb{Z}$ y $b$ es cualquier elemento de $2+4\mathbb{Z}$, entonces $a+b$ tendrá resto $3$ al dividirse en $4$, por lo que $a+b \in 3+4\mathbb{Z}$. El mismo razonamiento sirve para cualquier otro par de clases laterales. Esto indica que, ultimadamente, no importa los representantes concretos que saquemos de las clases laterales, pues la operación del grupo se comporta bien entre ellas.  
\hfill$\blacksquare$
\end{example}

La siguiente proposición se refiere al fenómeno que observamos en el ejemplo \ref{ejemplo_clases_laterales}.

\begin{prop}[Grupo cociente] \label{grupo_cociente}
Sea $(G, \circ)$ un grupo abeliano, y sea $H$ un subgrupo de $G$. La siguiente operación binaria\footnote{El símbolo $\ast$ no es estándar para denotar la operación en el grupo cociente. Generalmente se utiliza el mismo símbolo de la operación de $G$. En este caso, evitamos hacer eso por claridad.} $\ast$ en $\faktor{G}{H}$ está bien definida, y el conjunto $\faktor{G}{H}$ forma un grupo abeliano con ella:
$$\left(a \circ H \right) \ast \left( b \circ H \right) \coloneqq  (a \circ b) \circ H .$$
\end{prop}

\begin{proof}
Sea $e$ el neutro de $G$.

Lo primero que debemos mostrar es que $\ast$ está bien definida, es decir, que $\left(a \circ H \right) \ast \left( b \circ H \right)$ no depende de los representantes concretos $a$ y $b$. Para ello, consideremos $a', b' \in G$ tales que $a \circ H = a' \circ H$ y $b \circ H = b' \circ H$. Queremos probar que $(a \circ b) \circ H = (a' \circ b') \circ H$. Como $e \in H$ y $a' = a' \circ e$, tenemos que $a' \in a' \circ H$. Como $a' \circ H = a \circ H$, se sigue que $a' \in a \circ H$. Análogamente, tenemos que $b' \in b \circ H$. Luego existen $h_1, h_2 \in H$ tales que $a' = a \circ h_1$ y $b' = b \circ h_2$. Por lo tanto, $a' \circ b' = (a \circ b) \circ (h_1 \circ h_2)$. Como $h_1 \circ h_2 \in H$, esto último muestra que $a' \circ b' \in (a \circ b) \circ H$, y luego $(a \circ b) \circ H = (a' \circ b') \circ H$. Concluimos que $\ast$ está bien definida.

Ahora debemos probar que $\left(\faktor{G}{H},\, \ast\right)$ es un grupo abeliano. La asociatividad y la conmutatividad se heredan directamente de $\circ$ hacia $\ast$. Dejamos al lector verificar que el elemento neutro es $H = e \circ H$, y que el inverso de $a \circ H$ es $a^{-1} \circ H$.
\end{proof}

Hacemos notar que, en la proposición \ref{grupo_cociente}, $\faktor{G}{H}$ no es un subgrupo de $G$. De hecho, ni siquiera es un subconjunto de $G$. Veamos otra vez el ejemplo \ref{ejemplo_clases_laterales}. Las cuatro clases laterales no son enteros, sino subconjuntos de $\mathbb{Z}$. Ahora sabemos que esas cuatro clases laterales forman un grupo, y, si miramos con atención, notaremos que ese grupo es esencialmente idéntico al grupo de $\mathbb{Z}_4 = \left\{0, 1, 2, 3\right\}$ con la operación de suma en módulo $4$. Resulta incómodo no poder afirmar que ambos grupos son idénticos solo porque sus elementos tengan naturaleza distinta. Esto motiva la siguiente definición.

\begin{definition}[Isomorfismo] \label{isomorfismo_de_grupo}
Sean $(G_1, \circ_1)$ y $(G_2, \circ_2)$ dos grupos. Diremos que una función $\phi\colon G_1 \rightarrow G_2$ es un \emph{isomorfismo} si $\phi$ es una biyección y, además,
$$\forall\, a, b \in G_1 \qquad \phi(a \circ_1 b) = \phi(a) \circ_2 \phi(b).$$
Si existe un isomorfismo, se dice que los grupos $G_1$ y $G_2$ son \emph{isomorfos}, y se escribe $G_1 \simeq G_2$. \hfill$\blacksquare$
\end{definition}

En otras palabras, un isomorfismo es una biyección entre los grupos que también induce una biyección entre las relaciones definidas por las operaciones binarias. Uno puede pensar que son traductores. Por lo tanto, según la definición \ref{isomorfismo_de_grupo}, operar dos elementos $a, b \in G_1$ y luego aplicar el traductor $\phi$ tiene el mismo efecto que aplicar el traductor a $a$ y a $b$ por separado, y luego operarlos según la operación de $G_2$.

No es difícil convencerse de que dos grupos isomorfos tienen exactamente las misma propiedades. Por ejemplo, si $G_1 \simeq G_2$, entonces $G_1$ es abeliano si y solo si $G_2$ es abeliano, o $G_1$ tiene un subgrupo de $n$ elementos si y solo si $G_2$ tiene un subgrupo de $n$ elementos.


\begin{example} \label{ejemplo_exponencial}
Se puede verificar que los números reales, $\mathbb{R}$, forman un grupo con la suma. Por otro lado, los reales positivos, $\mathbb{R}^{+}$, forman un grupo con la multiplicación. La función $\varphi\colon \mathbb{R} \rightarrow \mathbb{R}^{+}$ dada por $\varphi(x) = e^x$ es biyectiva, y además $e^{x+y} = e^x \cdot e^y$. Luego $(\mathbb{R}, +) \simeq (\mathbb{R}^{+}, \cdot)$. \hfill$\blacksquare$
\end{example}

Resumiremos nuestra discusión del ejemplo \ref{ejemplo_clases_laterales} en la siguiente proposición.

\begin{prop}
Sea $m$ un entero positivo, y considere los grupos abelianos $(\mathbb{Z}, +)$ y $(\mathbb{Z}_m, +)$. Entonces $$\faktor{\mathbb{Z}}{m \mathbb{Z}} \; \simeq \; \mathbb{Z}_m.$$
\end{prop}


La siguiente es otra definición clave en la teoría de grupos.
 
\begin{definition}[Conjunto generado]\label{def_gen}
	Sea $(G,\circ)$ un grupo, y sea $a \in G$.
	El \emph{conjunto generado} por $a$ es
	$$\langle a \rangle \; \coloneq \; \{a^k\; \mid \; k\in\mathbb{Z}\}.$$
    \hfill$\blacksquare$
\end{definition}

A pesar de que la notación no lo indica, en la definición \ref{def_gen} el conjunto generado $\langle a \rangle$ depende del grupo $G$ al cual pertenece $a$. Usualmente esto está claro por contexto, pero también podemos usar la notación $\langle a \rangle_{G}$ para evitar ambigüedades.


\begin{proposition}\label{prop-generado}
	Sea $(G,\circ)$ un grupo, y sea $a \in G$. Entonces $(\langle a\rangle , \circ)$ es un subgrupo abeliano de $(G,\circ)$. 
\end{proposition} 


\begin{proof}
Como $\langle a\rangle\subseteq G$, solo hace falta demostrar
que $(\langle a\rangle,\circ)$ es un grupo. Tenemos que verificar la clausura de la operación y los axiomas:
\begin{itemize}
\item Para la clausura, sean $b,c \in \langle
a\rangle$. Sabemos que existen $n, m \in \mathbb{Z}$ tales que $b = a^m$ y $c = a^n$. Luego $b \circ c = a^{m+n} \in \langle a\rangle$, y $\circ$ es cerrada en $\langle a \rangle$.
\item La asociatividad se hereda directamente.
\item Por definición tenemos que $a^0$ es el neutro del grupo, y $a^0 \in \langle a \rangle$. 
\item Si $b \in \langle a \rangle$, entonces $b = a^n$ para algún $n \in \mathbb{Z}$. Tenemos que $a^{-n} \in \langle a \rangle$ es el inverso de $b$, por lo que $\langle a \rangle$ es cerrado bajo tomar inversos.
\item Para la conmutatividad, sean $b,c \in \langle
	a\rangle$. Sabemos que existen $n, m \in \mathbb{Z}$ tales que $b = a^m$ y $c = a^n$. Luego $b \circ c = a^{m + n} = a^{n + m} = c \circ b$.
\end{itemize}
\end{proof}

\begin{definition}[Generador, grupo cíclico] 
Sea $(G, \circ)$ un grupo. Decimos que un elemento $a \in G$ es un \emph{generador} de $G$ si $G = \langle a \rangle$. Decimos que $(G, \circ)$ es \emph{cíclico} si tiene un generador.
\hfill$\blacksquare$
\end{definition}

\begin{remark}
Un grupo cíclico puede tener más de un generador. Por ejemplo, vimos anteriormente vimos que $1$ es un generador de $(\mathbb{Z}_5, +)$ (ver ejemplo \ref{grupo_ciclico}). Sin embargo, también $4$ es un generador, pues $4 + 4 = 3$, $4 + 4 + 4 = 2$, $4 + 4 + 4 + 4 = 1$ y $4 + 4 + 4 + 4 + 4 = 0$ (considerando las sumas en módulo $5$).
\hfill$\blacksquare$
\end{remark}

\begin{definition}[Orden de un elemento]\label{def_orden}
	Sea $(G,\circ)$ un grupo con neutro $e$, y sea $a \in G$. El \emph{orden} de $a$, denotado $\ord{G}(a)$, es el menor entero positivo $m$ tal que $a^m = e$. Si dicho $m$ no existe, decimos que $a$ tiene \emph{orden infinito}. \hfill$\blacksquare$
\end{definition} 

\begin{remark}
En cualquier grupo, el neutro es el único elemento con orden $1$. \hfill$\blacksquare$
\end{remark}

Estamos usando el término «orden» en dos sentidos distintos, pues ya habíamos definido el orden de un grupo como su cardinalidad. Esto no suele generar confusión, ya que es claro cuando se habla del grupo y cuando se habla de un elemento de él. 

\begin{proposition} \label{prop-orden}
Sea $(G, \circ)$ un grupo, y sea $a \in G$ un elemento con orden finito, digamos $\ord{G}(a) = m$. Entonces
$$\langle a \rangle = \{ a^0, a^1, \dots, a^{m-1}\}.$$
Más aún, $(\langle a \rangle, \circ) \simeq (\mathbb{Z}_m, +)$.
\end{proposition}

\begin{proof} Sea $e$ el neutro de $(G, \circ)$. Por definición, $m$ es el menor entero positivo tal que $a^m = e$.

Veamos primero que $\langle a \rangle \subseteq \{ a^0, a^1, \dots, a^{m-1}\}$. Sea $b \in \langle a \rangle$ arbitrario, de modo que $b = a^k$ para algún $k \in \mathbb{Z}$. Por el algoritmo de la división de enteros, existe un $q \in \mathbb{Z}$ y un $r \in \{0, 1, \dots, m-1\}$ de modo que $k = mq+r$. Entonces $b = a^{mq+r} = \left(a^m\right)^q \circ a^r = e^q \circ a^r = a^r$. Como $a^r \in \{ a^0, a^1, \dots, a^{m-1}\}$, esto prueba que $\langle a \rangle \subseteq \{ a^0, a^1, \dots, a^{m-1}\}$. Por otro lado, es claro que $\{ a^0, a^1, \dots, a^{m-1}\} \subseteq \langle a \rangle$, así que tenemos la igualdad.

Ahora estableceremos el isomorfismo. Esto, en particular, implicará que $\abs{\langle a \rangle} = m$. Consideremos la función $\phi\colon \mathbb{Z}_m \rightarrow \{ a^0, a^1, \dots, a^{m-1}\}$ dada por $\phi(k) = a^k$. Claramente $\phi$ es sobreyectiva.

Para la inyectividad, sean $i, j \in \mathbb{Z}_m$ tales que $a^i = \phi(i) = \phi(j) = a^j$. Sin pérdida de generalidad, podemos asumir que $i \geq j$. Entonces $$a^{i - j} = a^i \cdot a^{-j} = a^i \cdot \left(a^j \right)^{-1} = a^i \cdot \left(a^i \right)^{-1} = e.$$
Como $0 \leq i - j < m$, no puede ocurrir que $i-j$ sea un entero positivo, pues $m$ es el menor entero positivo $k$ tal que $a^k = e$. Esto obliga a que $i-j = 0$, es decir, $\phi$ es inyectiva.

Por último, veamos que $\phi$ es un isomorfismo. Sean $i, j \in \mathbb{Z}_m$. Tenemos que
$$\phi(i + j) = a^{i+j} = a^i \cdot a^j = \phi(i) \cdot \phi(j).$$ Esto completa la demostración.
\end{proof}

Lo que dice el isomorfismo de la proposición \ref{prop-orden} es que todos los grupos cíclicos del mismo orden son esencialmente iguales.

\begin{corollary} \label{orden de elemento divide a orden de grupo}
Si $(G, \circ)$ es un grupo abeliano finito y $a \in G$, entonces $\ord{G}(a)$ divide a $\abs{G}$.
\end{corollary}

\begin{proof}
Como $\langle a \rangle$ es un subgrupo de $G$, el teorema de Lagrange implica que $\abs{\langle a \rangle} = \ord{G}(a)$ divide a $\abs{G}$.
\end{proof}

\begin{remark}
Dado un grupo $G$, un elemento $a \in G$ es un generador si y solo si $\ord{G}(a) = |G|$.
\hfill$\blacksquare$
\end{remark}

Notemos que el orden del subgrupo generado por un elemento coincide con el orden de dicho elemento. Esta es la razón por la que el término «orden» se utiliza en ambos sentidos.

%https://resources.saylor.org/wwwresources/archived/site/wp-content/uploads/2011/05/Order-group-theory.pdf


De la demostración de la proposición \ref{prop-orden} también se desprende el siguiente resultado:

\begin{corollary} \label{observacion orden}
Sea $(G, \circ)$ un grupo con neutro $e$, y sea $a \in G$ un elemento con orden finito, digamos $\ord{G}(a) = m$. Dado un $k \in \mathbb{Z}$, tenemos que $a^k = e$ si y solo si $m$ divide a $k$.
\end{corollary}

El siguiente es un resultado fundamental en la teoría de grupos. Se desprende directamente de los corolarios que acabamos de discutir. Nuevamente, la hipótesis de abelianidad no es necesaria, y solo la agregamos por simplicidad.

\begin{prop} \label{PTF para grupos finitos}
Sea $(G, \circ)$ un grupo abeliano finito con neutro $e$. Entonces $a^{|G|} = e$ para todo $a \in G$.
\end{prop}

\begin{proof}
Sea $m = \ord{G}(a)$. Por el corolario \ref{orden de elemento divide a orden de grupo} tenemos que $m$ divide a $|G|$. Luego, por el corolario \ref{observacion orden}, $a^{|G|} = e$.
\end{proof}


Los siguientes resultados exponen las primeras conexiones entre la teoría de grupos y la teoría de números.

%\comentarioin{nuevo teorema y corolario para no usar isomorfismos}
\begin{prop}\label{orden del producto coprimos}
	Sea $(G,\circ)$ un grupo abeliano, y sean $a,b\in G$ con $\ord{G}(a)=m$ y $\ord{G}(b)=n$. Si $m$ y $n$ son coprimos, entonces $\ord{G}(a\circ b)=mn$.
\end{prop}

\begin{proof}
	Sea $e$ el neutro. Sea $r=\ord{G}(a\circ b)$. Debemos demostrar que $r = mn$. Tenemos que 
	$$(a\circ b)^{mn} = a^{mn}\circ b^{mn} = (a^m)^n\circ (b^n)^m = e^n\circ e^m = e.$$
Luego, por el corolario \ref{observacion orden}, tenemos que $r$ divide a $mn$, y por lo tanto $r\leq mn$. Nos falta probar que $mn\leq r$. Para ello, notemos que, como $e = (a\circ b)^r$, entonces
\begin{eqnarray*}
	 e \ = \ e^n & =&  (a\circ b)^{rn} \\
	 & = & a^{rn} \circ b^{rn}\\
	 & = & a^{rn} \circ (b^{n})^{r}\\
	 & = & a^{rn} \circ e^{r}\\
	 &=& a^{rn}.
\end{eqnarray*}
 De la misma forma, tenemos que
\begin{eqnarray*}
	 e \ = \ e^m & =&  (a\circ b)^{rm} \\
	 & = & a^{rm} \circ b^{rm}\\
	 & = & (a^{m})^r \circ b^{rm}\\
	 & = & e^{r} \circ b^{rm}\\
	 &=& b^{rm}.
\end{eqnarray*}
Como $a^{rn}=e$ y $b^{rm} = e$, usando el corolario \ref{observacion orden} deducimos que $m \divi rn$ y $n\divi rm$. Como $m$ y $n$ son coprimos, necesariamente $m \divi r$ y $n\divi r$. Así, tenemos que $mn\divi r$, ya que $\MCD(m,n)=1$. De eso se sigue que $mn \leq r$, y esto concluye la demostración. 
\end{proof}

Podemos generalizar la proposición \ref{orden del producto coprimos} de la siguiente manera. 

\begin{corollary}\label{corolario orden}
Sea $(G,\circ)$ un grupo abeliano, y sean $a,b\in G$ con $\ord{G}(a)=m$ y $\ord{G}(b)=n$. Entonces existe un elemento $c\in G$ tal que $\ord{G}(c)=\MCM(m,n)$.
\end{corollary}
\begin{proof}
Sea $e$ el neutro. Sean $p_1,\dots,p_k$ todos los primos que dividen a $m$ o a $n$. Podemos escribir $m$ y $n$ como 
$$m \ = \ \prod\limits_{i=1}^k p_i^{\alpha_i} \ \qquad\text{y}\qquad \ n \ = \ \prod\limits_{i=1}^k p_i^{\beta_i},$$ donde $\alpha_i,\beta_i \geq 0$ (son iguales a 0 cuando el primo $p_i$ no es divisor del número). Además, sabemos que
$$\MCM(m, n) \ = \ \prod\limits_{i=1}^k p_i^{\max(\alpha_i,\, \beta_i)}.$$

Ahora definimos
$$m' \ = \ \prod\limits_{i\,:\,\alpha_i\geq \beta_i} p_i^{\alpha_i} \ \qquad\text{y}\qquad \ n' \ = \ \prod\limits_{i\,:\,\beta_i> \alpha_i} p_i^{\beta_i}.$$
Por ejemplo, si $m=2^3\cdot 3^2\cdot 5^1$ y $n=2^1\cdot 3^2\cdot 7^1$, entonces $m' = 2^3\cdot 3^2\cdot 5^1$ y $n' = 7^1$. Notemos que las siguientes afirmaciones sobre $m'$ y $n'$ se cumplen:
\begin{itemize}
	\item $m'$ divide a $m$
	\item $n'$ divide a $n$
	\item $m'$ y $n'$ son coprimos
	\item $\MCM(m,n) = m'n'$
\end{itemize}
Sean $q_1 = \frac{m}{m'}$ y $q_2 = \frac{n}{n'}$, y consideremos los elementos $a' = a^{q_1}$ y $b' = b^{q_2}$. Vamos a mostrar que $\ord{G}(a')=m'$ y $\ord{G}(b')=n'$. Primero, sea $k=\ord{G}(a')$. Entonces $$e\ = \ (a')^k\ = \ (a^{q_1})^k \ = \ a^{q_1k}$$
Por el corolario \ref{observacion orden}, tenemos que $\ord{G}(a) = m$ divide a $q_1k$. Es decir, existe un $q_3 \in \mathbb{Z}$ tal que $q_3 m = q_1k$, de lo que se deduce que $q_3 m' = k$. Como $k$ y $m'$ son enteros positivos, eso implica que $m' \leq k$. Por otro lado, $$(a')^{m'} \ = \ (a^{q_1})^{m'} \ = \ a^m \ = \ e$$ 
%Nuevamente, por la observación \ref{observacion orden} necesariamente $k$ divide a $m'$, y
Como $k=\ord{G}(a')$, deducimos que $k\leq m'$. Luego, $\ord{G}(a')=k=m'$. De forma análoga, se puede mostrar que $\ord{G}(b')=n'$.

%Una propiedad importante que se sigue del teorema \ref{orden del producto coprimos} y el corolario \ref{corolario orden} es la siguiente.

Por último, definimos $c=a'\circ b'$. Como $(G,\circ)$ es un grupo abeliano, y $a',b'\in G$ son tales que $\ord{G}(a') = m'$, $\ord{G}(b')= n'$, y $\MCD(m', n') = 1$, la proposición \ref{orden del producto coprimos} nos permite concluir que $\ord{G}(c) = m'n' = \MCM(m,n)$. Esto concluye la demostración.
\end{proof}

El último resultado que demostraremos en esta sección es el teorema de Cauchy. Nuevamente, por simplicidad solo demostraremos el caso abeliano (que es el que necesitaremos más adelante), pero hacemos notar que el teorema es válido para cualquier grupo finito. En cierto sentido, es un recíproco parcial del teorema de Lagrange.

\begin{theorem}[{Teorema de Cauchy}]\label{teo cauchy}
Sea $p$ un número primo que divide al orden de un grupo abeliano finito
$(G,\circ)$. Entonces existe un elemento de $G$ que tiene orden $p$.
\end{theorem}

\begin{proof}
Dejaremos fijo $p$ y haremos inducción en $n \coloneqq |G|$. 

Notemos que el resultado es vacuamente cierto si $p$ no divide a $n$, por lo que solo debemos revisar los casos en que $n$ sea efectivamente un múltiplo de $p$. 

El caso base es $n = p$. Como $p \geq 2$, existe un elemento $a \in G$ con $\ord{G}(a) > 1$ (ya que el único elemento con orden $1$ es el neutro). Por el teorema de Lagrange, tenemos que $\ord{G}(a)$ divide a $p$, y, como $p$ es primo, necesariamente $\ord{G}(a) = p$.

Para el paso inductivo, nuevamente consideramos un elemento $a \in G$ con $\ord{G}(a) = m > 1$. Hay dos casos:
\begin{itemize}
\item El primer caso es que $p$ divide a $m$. Entonces $a^{\frac{m}{p}}$ es un elemento de orden $p$.
\item El segundo caso es que $p$ no divide a $m$. Sabemos que $\langle a \rangle$ es un subgrupo, así que podemos considerar el cociente $H = \faktor{G}{\langle a \rangle}$, que también será un grupo abeliano. Notemos que
$$|G| = \left|\faktor{G}{\langle a \rangle}\right| \cdot \left|\langle a \rangle\right|.$$
Como $p$ divide a $|G|$ y no divide a $\left|\langle a \rangle\right| = m$, tenemos que $p$ divide a $|H|$. Como $H$ es un grupo abeliano con $|H| < |G| = n$, podemos aplicar la hipótesis de inducción: existe un $B \in H$ con $\ord{H}(B) = p$. Dicho $B$ es una clase lateral de $G$ bajo $\langle a \rangle$, es decir $B = c \circ \langle a \rangle$ para algún $c \in G$. Sea $k = \ord{G}(c)$. Notemos que, por definición de la operación en el grupo cociente $H$ (proposición \ref{grupo_cociente}), se cumple que
$$B^k = (c \circ \langle a \rangle)^k = c^k \circ \langle a \rangle = e \circ \langle a \rangle,$$
por lo que $\ord{H}(B)$ divide a $k$ por el corolario \ref{observacion orden}. Como $\ord{H}(B) = p$ y $k = \ord{G}(c)$, tenemos que $c^{\frac{k}{p}}$ es un elemento de orden $p$.
\end{itemize}
Como en ambos casos encontramos un elemento que tiene orden $p$, esto completa la inducción.
\end{proof}


% Ale: no estoy segura de si los lemas que estaban demostrando a continuación son necesarios para el documento, o si solo eran un camino para demostrar la versión no abeliana del teorema de Cauchy. Por el momento, escribiré una demostración simplificada para el caso abeliano, y dejaré la original comentada.

\begin{comment}
Para demostrar el teorema \ref{teo cauchy}, vamos a demostrar el lema \ref{tcf}, que es una versión más fuerte del teorema de Cauchy.
\begin{lemma}\label{tcf}
Sea $p$ un número primo que divide el orden de un grupo finito
$(G, \circ)$. Entonces $G$ tiene $p\cdot k$ elementos que son solución
de la ecuación $$x^p=e,$$ donde $e$ es el neutro de $(G, \circ)$ y $k$
es un número entero positivo.
%\comentarioin{al parecer este es el teorema de cauchy original (?)}
\end{lemma}
%\comentarioin{hay que poner en la demostracion que estamos demostrando el lema A13? o se entiende?}
\begin{proof}
Sea $|G|= n $, y suponga que $p\divi n$, es decir, $n= p\cdot c$ con
$c\in \mathbb{N}$. Definimos el conjunto
\begin{eqnarray*}
S & = & \{(g_0,g_1,\ldots,g_{p-1})\mid g_i\in G \text{ para cada }
i \in [0,p-1] \text{ y } g_0 \cdot g_1 \cdot \ldots \cdot g_{p-1} =
e\}.
\end{eqnarray*}
Notemos que si $(g_0,g_1,\ldots,g_{p-1})\in S$, entonces al escoger
los elementos $g_0,\ldots,g_{p-2}$ el elemento $g_{p-1}$ queda
únicamente determinado: como $g_0 \cdot \ldots \cdot g_{p-2}\cdot
g_{p-1}=e$, entonces necesariamente $g_{p-1} = (g_0 \cdot \ldots \cdot
g_{p-2})^{-1}$. De esto podemos concluir que $|S|=n^{p-1}$, usando el
argumento que para cada $i\in [0,p-2]$, $g_i\in G$, y tenemos $n$
posibles elementos que podemos escoger. Definamos la relación binaria
$\sim$ sobre las tuplas de $S$ de la siguiente forma:
\begin{multline*}
(g_0,g_1,\ldots,g_{p-1})\sim (h_0,h_1,\ldots,h_{p-1}) \ \ \text{si y solo si} \\
(h_0,h_1,\ldots,h_{p-1})\text{ es una permutación cíclica de } (g_0,g_1\ldots,g_{p-1}), 
\end{multline*}
es decir, si $(h_0,h_1,\ldots,h_{p-1}) = (g_{(0+k) \mods p},
g_{(1+k) \mods p}, \ldots g_{(p-1+k) \mods p})$ para algún $k\in
[0,p-1]$. Demostrar que $\sim$ es una relación de equivalencia es un
ejercicio sencillo y queda propuesto para el lector.  Ya que $\sim$
satisface esta propiedad, podemos analizar las clases de equivalencia
bajo esta relación.  Podemos ver que existen dos tipos de tuplas en
$S$.  El tipo I ocurre cuando todos los elementos de la tupla son el
mismo, en cuyo caso tenemos que $|[(g_0,\ldots,g_{p-1})]_{\sim}|=1$
(al hacer permutaciones cíclicas obtenemos la misma tupla).  El tipo
II occure cuando en la tupla existen elementos distintos. Nótese que, en general, si permutamos una tupla del tipo II podríamos eventualmente obtener la misma tupla. Por ejemplo, si nuestra tupla es $(a,b,a,b)$, entonces podemos hacer una permutación cíclica de dos hacia la izquierda (vale decir, $k = 2$) y obtenemos la misma tupla $(a,b,a,b)$. En la afirmación \ref{afirmacion-perm-cicl} demostramos que
%si el tamaño de la tupla es un número primo (como en nuestro caso)
en nuestro caso no sucede lo anterior porque $p$ es un número primo.
%entonces no sucede lo anterior, en cuyo caso 
De esta forma concluimos que $|[(g_0,\ldots,g_{p-1})]_{\sim}|=p$.


\begin{afirmacion}\label{afirmacion-perm-cicl}
%Sea $p$ un número primo.
Si $(g_0, g_1, \ldots, g_{p-1}) = (g_{(0+k) \mods p}, g_{(1+k) \mods
p}, \ldots, g_{(p-1+k) \mods p})$ para algún $k \in [1,p-1]$, entonces
$g_0 = g_1 = \cdots = g_{p-1}$.
\end{afirmacion}

\begin{proof}
Suponga que $(g_0, g_1, \ldots, g_{p-1}) = (g_{(0+k) \mods p},
g_{(1+k) \mods p}, \ldots, g_{(p-1+k) \mods p})$ para algún $k \in
[1,p-1]$. Así, tenemos que $g_i = g_{(i+k) \mods p}$ para cada
$i \in [0,p-1]$. Nótese que de esto se deduce que $g_{((j\mods p) +k)\mods p} = g_{(j+k)\mods p}$, de lo cual se concluye que
\begin{align}\label{eq-lem-perm-cicl}
g_0 = g_k = g_{(2\cdot k) \mods p} = g_{(3\cdot k) \mods p} = \cdots = g_{((k-1) \cdot k) \mods p}.
\end{align}
Por lo tanto, si demostramos que
\begin{eqnarray*}
(i \cdot k) \mods p & \neq & (j \cdot k) \mods p
\end{eqnarray*}
para cada $i,j \in [0,p-1]$ tales que $i < j$, entonces concluimos que
$g_0 = g_1 = \cdots = g_{p-1}$ ya que todos los elementos de la tupla
$(g_0, g_1, \ldots, g_{p-1})$ son mencionados
en \eqref{eq-lem-perm-cicl}. Por el contrario, supongamos que $(i \cdot
k) \mods p \ = \ (j \cdot k) \mods p$ para algún par $i,j \in [0,p-1]$ tal
que $i < j$. Tenemos entonces que $i \cdot k \equiv j \cdot k \modl
p$, vale decir, $(j-i) \cdot k \equiv 0 \modl p$. Así, dado que $k \in
[1,p-1]$ y $p$ es un número primo, concluimos que $(j-i) \equiv
0 \modl p$. Pero $0 < j-i < p$, por lo que obtenemos una
contradicción.
\end{proof}

Continuando entonces con la demostración del lema \ref{tcf},
supongamos que hay $r$ clases de equivalencia de tuplas del tipo I y
$q$ clases de equivalencia de tuplas del tipo II, y observemos que
$r\geq 1$ ya que $(e,e,\ldots,e)$ es de tipo I. Luego, como todas las
tuplas de $S$ caen en una clase de equivalencia de $\sim$, se cumple
que $r\cdot 1 \ + \ q\cdot p = |S|$. Además, tenemos que:
\begin{eqnarray*}
r\cdot 1 \ + \ q\cdot p \ = \ |S| &\Rightarrow&  r \ + \ q\cdot p \ = \ n^{p-1}\\ 
	&\Rightarrow& r \ = \ n^{p-1} \ - \ q\cdot p\\
	&\Rightarrow& r \ = \ (p \cdot c)^{p-1} - q \cdot p \quad\quad\quad \text{dado que } n= p\cdot c\\
	&\Rightarrow& r \ = \ p\cdot (c\cdot (p \cdot c)^{p-2}-q).
\end{eqnarray*}
Sea $k = c\cdot (p \cdot c)^{p-2}-q$. Dado que $p\geq 2$, tenemos que
$k \in \mathbb{Z}$. Así, dado que $r = p \cdot k$ y $r \geq 1$,
concluimos que $k$ es un número entero positivo.
%entonces
%necesariamente $k\geq 1$ (un número natural).
Luego, existen $r = p\cdot k$ tuplas de tipo I,
%que sus clases de equivalencia son del primer
%tipo,
es decir, $r=p\cdot k$ elementos distintos $g\in G$ tales que $g \cdot
g \cdot \ldots \cdot g = g^p = e$, los cuales corresponden a las
soluciones de la ecuación $x^p=e$.
\end{proof}
Habiendo demostrado el lema \ref{tcf}, podemos hacer la demostración del teorema de Cauchy.
\begin{proof}[Demostración del teorema \ref{teo cauchy}]
Supongamos que $(G,\circ)$ es un grupo finito, y $p$ es un número
primo que divide el orden de $G$. Por el lema \ref{tcf}, sabemos que
$G$ tiene $p\cdot k$ soluciones de la ecuación $x^p=e$, donde $e$ es
el neutro de $(G,\circ)$ y $k$ es un número entero positivo. Como
$p\cdot k \geq 2$, entonces existe un elemento $a\in G$ tal que $a\neq
e$ y $a^p=e$. Supongamos que $O_G(a)=m$ con $m<p$. Nótese que $m > 1$
puesto que $a \neq e$, y $e = a^m = a^p$. Usando el hecho de que
podemos escribir $p=\alpha \cdot m+\beta$, con $\alpha \in \mathbb{N}$
y $\beta = p \mods m$, concluimos que
\begin{eqnarray*}
	e & = & a^p \\
	&=& a^{\alpha \cdot m + \beta}\\
	&=& (a^{m})^{\alpha} \circ a^{\beta}\\
	&=& e^{\alpha}\circ a^{\beta}\\
	&=& a^{\beta}
\end{eqnarray*} 
Como $p$ es primo, $1<m<p$ y $\beta = p \mods m$, tenemos que
$\beta \in [1,m-1]$. Pero esto contradice el hecho que $O_G(a) = m$,
puesto que $a^{\beta}=e$. De esta forma concluimos que $O_G(a) = p$,
lo que demuestra que existe un elemento de $G$ que tiene orden $p$.
\end{proof}
\end{comment}
%\comentarioin{Bernardo: hay que cerrar de alguna forma este preliminar? o se corta aqui nomas?}



\section{Preliminares Teoría de Cuerpos}
\label{app-cuerpos}
\subsection{Definiciones y propiedades básicas}
%\comentarioin{Habíamos hablado de poner $\oplus,\otimes$ para denotar la suma y multiplicacion en un cuerpo, y así dejar $q\cdot$ para denotar la suma de $q$ veces un cuerpo. Sin embargo, empecé a cambiar las cosas y la notación se volvió muy fea y además engorrosa (no podíamos denotar $-a$ por ejemplo. Creo que lo mejor es escribir la suma de $q$ términos de $a$ como $qa$ o $\textbf{q}a$. En cambio, cuando en el cuerpo estámos hablando del elemento neutro, preferí poner la notación $\mathbb{0},\mathbb{1}$ para denotarlos}
\begin{definition}[Cuerpo]
Una tupla $(F, +,\cdot)$, dónde $F$ es un conjunto y $ +,\cdot :F\times F\to F$ son funciones (totales), es un cuerpo si:
\begin{enumerate}
	\item $(F,+)$ es un grupo abeliano

	\item $(F\setminus \{\0\},\cdot)$ es un grupo abeliano, con $\0$ el elemento neutro de $(F,+)$

   \item La operación $\cdot$ distribuye sobre $+$, es decir, se cumple que para todo $a,b,c$ en $F$:
    $$(a+b)\cdot c \ = \ (a \cdot c ) + (b\cdot c)$$
\end{enumerate}
\end{definition}

%\begin{definition}[Cuerpo]
%Un conjunto $F$ y dos funciones (totales) $\circ,+ : F\times
%F \to F$ forman un cuerpo si ambas estructuras algebraicas $(F,\circ)$ y $(F\setminus \{e_+\})$ son un grupo, donde $e_+$ el elemento neutro de $(F,\circ)$, y se cumplen las siguientes propiedades:
%\begin{enumerate}
%	\item $F$ es cerrado bajo las operaciones $\circ$ y $+$
%	\item Las operaciones $\circ, +$ son conmutativas \comentarioin{esto no iria si es que los grupos los definimos con conmutatividad}
%	\item La operación $\circ$ distribuye sobre $+$, es decir, 
%	$$(a+b)\circ(c + d) \ = \ (a \circ c ) + (a\circ d)+(b \circ c ) + (b\circ d)$$
%\end{enumerate}
%\end{definition}
Dado un cuerpo $(F,+,\cdot)$, hablamos de adición o suma para
referirnos a $+$, y de multiplicación para referirnos a $\cdot$. El
elemento neutro de la suma es $\0$ y el de la multiplicación es
$\1$. Además, para $a\in F$ decimos que $-a$ es su inverso bajo $+$ y
$a^{-1}$ es su inverso bajo $\cdot$.
%También, el elemento neutro de la
%suma es $\0$ y el de la multiplicación es $\1$.
También, dado $n \in \mathbb{N}$, usamos las siguientes
abreviaciones:
\begin{eqnarray*}
		na & = & \underbrace{a +\ldots + a}_\text{$n$ veces}\\
                (-n)a & = & n(-a)\\
		a^n & = & \underbrace{a \cdot \ldots \cdot a}_\text{$n$ veces}\\
		a^{-n} & = & (a^{-1})^n
\end{eqnarray*}
En particular, tenemos que $0a = \0$ y $a^0 = \1$. Por último, con el
objetivo de no sobrecargar la notación, la multiplicación
tiene preferencia sobre la adición, es decir, $a\cdot b + c\cdot d
= (a\cdot b)+(c\cdot d)$.
%\comentarioin{Aca poner que $a+\cdots +a = qa$ sin el $\cdot$.}
A continuación presentamos algunas propiedades básicas que se cumplen
en los cuerpos.

%\comentarioin{Agregue el punto 3 a la siguiente proposición: es algo importante sobre cuerpos y ademas lo utilizo para demostrar que la caracteristica de un cuerpo finito es primo}
\begin{proposition}\label{proposicion cuerpos}
\hfill
\begin{enumerate}
\item Para todo $a\in F$ se cumple que $a\cdot \0 =\0\cdot a= \0$

\item Para todo $a,b\in F$ se cumple que $(-a)\cdot b=-(a\cdot b) = a\cdot (-b)$

\item Para todo $a \in F$ y $p \in \mathbb{Z}$ se cumple que $-(pa) = (-p)a = p(-a)$.

\item Para todo $a \in F$ y $p,q \in \mathbb{Z}$ se cumple que $(pq)a = p(qa)$.

\item Para todo $a,b\in F$ y $p \in \mathbb{Z}$ se cumple que $p(a\cdot b) = (pa)\cdot b = a \cdot (pb)$. De esto y la propiedad 4 se concluye que para todo $a,b\in F$ y $p,q \in \mathbb{Z}$, se cumple que $(pq)(a\cdot b) = (pa)\cdot (qb)$.

\item Para todo $a,b\in F$ se cumple que $a\cdot b = \0$ si y solo si $a=\0$ o $b=\0$\label{inciso prop}

\end{enumerate}
\end{proposition}
%\comentarioin{agregue este comentario al inciso 6 porque creo que es algo muy relevante}
En particular, el inciso \ref{inciso prop} de la proposición anterior comienza a darnos una mirada de las buenas propiedades en los cuerpos: nos permite hacer ``cancelación'' en las ecuaciones. Vamos a profundizar en esta propiedad con un ejemplo.
\begin{example}
Considere la estructura $(\mathbb{Z}_4,+,\cdot)$. Notemos que si
queremos resolver la ecuación $$ 2\cdot x\ = \ 2\cdot 2,$$ no podemos
``cancelar'' el 2 en ambos lados de la ecuación y concluir que $x =
2$. De hecho, si bien $x =2$ es una raiz de la ecuación, $x=0$ también
lo es. Esto es debido a que en esta estructura el elemento $2$ no
tiene inverso multiplicativo.  En general, cuando consideramos
$(\mathbb{Z}_n,+,\cdot)$ para $n \geq 2$ un número compuesto, no vamos
a tener un cuerpo y
%no estamos trabajando sobre un cuerpo,
vamos a encontrar elementos de la estructura $a\neq 0$ y $b\neq 0$
tales que $a\cdot b = 0$. \qed
\end{example}
\begin{proof}[Demostración de la proposición \ref{proposicion cuerpos}]\hfill
	\begin{enumerate}
		\item 
		Sabemos que $\0+\0 = \0$. Luego:
		\begin{eqnarray*}
\0+\0 = \0		&\Rightarrow & a\cdot(\0+\0) = a\cdot \0\\
			&\Rightarrow &a\cdot \0+a\cdot \0 = a\cdot \0\\
                        &\Rightarrow &a\cdot \0+a\cdot \0 + -(a\cdot \0) = a\cdot \0 + -(a \cdot \0)\\
                        &\Rightarrow &a\cdot \0+ \0 = \0\\
			&\Rightarrow &a\cdot \0 = \0 
		\end{eqnarray*}
	%Donde la notación $x-y = x +(-y)$.
        Por lo tanto, $a\cdot \0
=\0 $. Por la conmutatividad de la multiplicación también podemos
obtener que $\0 \cdot a=\0 $.

	\item Tenemos que:
        \begin{eqnarray*} (-a)\cdot b + a \cdot b & = & ((-a) +
        a) \cdot b\\ & = & \0 \cdot b\\ 
        & = & \0 \quad\quad\quad \text{por la propiedad 1}
        \end{eqnarray*}
Así, tenemos que $-(a \cdot b) = (-a)\cdot b$. Además, por esta propiedad y la 
        conmutatividad de la multiplicación obtenemos $a\cdot (-b) = (-b) \cdot a = -(b\cdot a) = -(a\cdot b)$.

        \item Si $p = 0$, concluimos que $-(pa) = (-p)a = p(-a)$ por
        la definición de la notación $nb$ (para $n \in \mathbb{Z}$ y $b \in
        F$). Si $p > 0$, tenemos que:
\begin{eqnarray*}
(-p)a + pa & = & p(-a) + pa\\
& = & \underbrace{(-a) +\cdots + (-a)}_\text{$p$ veces} \ + \ \underbrace{a +\cdots + a}_\text{$p$ veces}\\ 
& = & \underbrace{((-a) + a) + \cdots + ((-a) + a)}_\text{$p$ veces}\\
& = & \underbrace{\0 + \cdots + \0}_\text{$p$ veces}\\
& = & \0
\end{eqnarray*}
Por lo tanto, $-(pa) = (-p)a$. Además, por definición tenemos que $p(-a) = (-p)a = -(pa)$. El caso $p < 0$ es dejado como ejercicio para el lector (para resolverlo es útil considerar el caso $p > 0$ y la propiedad $p = -|p|$). 

\item Si $p = 0$ o $q = 0$, concluimos que $(pq)a = p(qa)$ por
        la definición de la notación $nb$ (para $n \in \mathbb{Z}$ y $b \in
        F$). Si $p > 0$ y $q > 0$, tenemos que:
\begin{eqnarray*}
(pq)a &=& \underbrace{a +\cdots + a}_\text{$pq$ veces}\\
 &=& \underbrace{c +\cdots + c}_\text{$p$ veces} \quad\quad\quad \text{para }
 c = \underbrace{a +\cdots + a}_\text{$q$ veces}\\
 & = & \underbrace{qa +\cdots + qa}_\text{$p$ veces}\\
 & = & p(qa)
\end{eqnarray*}
Si $p < 0$ y $q < 0$, tenemos por la propiedad anterior que:
\begin{eqnarray*}
(pq)a &=& ((-p)(-q))a\\
 &=& (-p)((-q)a)\\
 &=& (-p)(-(qa)) \quad\quad\quad \text{por la propiedad 3}\\
 &=& p(-(-(qa))) \quad\quad\quad \text{por la propiedad 3}\\
 &=& p(qa)
\end{eqnarray*}
Los casos faltantes pueden ser demostrados utilizando las mismas ideas, y son dejados como ejercicios para el lector.

	\item Si $p = 0$, concluimos que $p(a\cdot b) = (pa)\cdot b = a \cdot (pb)$ por la propiedad 1 y la definición de la notación $nb$ (para $n \in \mathbb{Z}$ y $b \in F$). Si $p > 0$, tenemos que:
	\begin{eqnarray*}
		p(a\cdot b)&=&\underbrace{a\cdot b +\cdots + a\cdot b}_\text{$p$ veces}\\
		&=&(\underbrace{a +\cdots + a}_\text{$p$ veces}) \cdot b\\
		&=&(pa) \cdot b
	\end{eqnarray*}		
Además, utilizando esta propiedad y la conmutatividad de la multiplicación obtenemos $p(a\cdot b) = p(b \cdot a) = (pb) \cdot a = a \cdot (pb)$. Si $p < 0$, utilizando la propiedad anterior concluimos que:
	\begin{eqnarray*}
		p(a\cdot b)&=& (-|p|)(a \cdot b)\\
		&=& |p|(-(a \cdot b))\\
		&=& |p|((-a) \cdot b) \quad\quad\quad \text{por la propiedad 2}\\
		&=& (|p|(-a)) \cdot b\\
		&=& ((-|p|)(a)) \cdot b\\
		&=& (pa) \cdot b
	\end{eqnarray*}
	Finalmente, utilizando esta propiedad y la conmutatividad de la multiplicación obtenemos $p(a\cdot b) = a \cdot (pb)$. 

	\item ($\Rightarrow$)
	Supongamos por contradicción que $a\cdot b=\0$, $a\neq \0$ y $b \neq \0$. Luego, $b$ tiene inverso bajo $\cdot$. Así, utilizando nuevamente la primera propiedad obtenemos:
	\begin{eqnarray*}
		a\cdot b = \0 &\Rightarrow& a\cdot b\cdot b^{-1} = \0 \cdot b^{-1}\\
       &\Rightarrow& a\cdot \1 = \0\\
       &\Rightarrow& a = \0
	\end{eqnarray*}
	Lo cual es una contradicción ya que supusimos que $a\neq \0$.
	
	($\Leftarrow$)
	Esta dirección se cumple por la propiedad 1.
	
	\qedhere
	\end{enumerate}
\end{proof}
Por último, vamos a definir lo que es un subcuerpo. Además, introducimos el concepto de cuerpo de extensión.
\begin{definition}
Sea $(F,+,\cdot)$ un cuerpo y $K\subseteq F$. Decimos que $(K,
+, \cdot)$ es un subcuerpo de $(F,+,\cdot)$ si $(K, +, \cdot)$ es un
cuerpo. En la misma línea, bajo las condiciones anteriores decimos que
$(F,+,\cdot)$ es un cuerpo de extensión de $(K,+,\cdot)$.
\end{definition}

Nótese que en la definición anterior usamos las mismas operaciones en
$(F,+,\cdot)$ y $(K, +, \cdot)$. En particular, se puede demostrar que
$(K, +, \cdot)$ y $(F,+,\cdot)$ tienen los mismos neutros bajo $+$ y
$\cdot$, lo cual es dejado como un ejercicio para el lector. Así, por
ejemplo, $(\mathbb{Q}, +, \cdot)$ es un subcuerpo de $(\mathbb{R},
+, \cdot)$, donde $+$ y $\cdot$ son la suma y multiplicación
usuales. Finalmente, nótese que si $(K, +, \cdot)$ es un subcuerpo de
$(F,+,\cdot)$, entonces $(K, +)$ y $(K \setminus \{\0\}, \cdot)$ son
subgrupos de $(F, +)$ y $(F \setminus \{\0\}, \cdot)$,
respectivamente.

%\comentarioin{aqui falta concluir con algo del estilo: los subcuerpos son subgrupos, etc}

%\comentarioin{Marcelo: hasta aquí revisé}

\subsection{Cuerpos finitos}
Los cuerpos finitos son de gran importancia para el desarrollo de este
documento.

Un cuerpo $(F,+,\cdot)$ se dice finito si $F$ es un conjunto
finito. Por ejemplo $(\mathbb{Z}_p, +, \cdot)$ es un cuerpo finito si
$p$ es un número primo y las operaciones $+, \cdot$ son realizadas en
módulo $p$.
%A continuación, vamos a definir formalmente el caso particular de los
%cuerpos finitos, los cuales 
%\begin{definition}[Cuerpo finito]
%	Una cuerpo finito
%        %(cuerpo de Galois)
%        $(F,+,\cdot)$ es un cuerpo con una cantidad finita de elementos.
%\end{definition}
Si $(F,+,\cdot)$ es un cuerpo finito, entonces $(F,+)$ es un grupo
finito, de lo cual concluimos por la proposición \ref{prop-orden} que
existe $n \in \mathbb{N}$ tal que $n \geq 1$ y $n \1 = \0$. Esta propiedad es
utilizada para definir la característica de un cuerpo finito, una
noción fundamental asociada a estas estructuras.
\begin{definition}
La característica de un cuerpo finito $(F, +, \cdot)$ se define como el menor $p\in \mathbb{N}$ tal que $p \geq 1$ y $p \1 = \0$, donde $\0$ y $\1$ son los neutros de las operaciones $+$ y $\cdot$, respectivamente.
%$\overbrace{\1 +\1 +\cdots +\1}^\text{$p$ veces}  = p\1 = \0 $ (suma de $p$ términos).
\end{definition}
%Si bien la característica de un cuerpo se puede definir de forma general, nosotros solo nos restringimos a los cuerpos finitos.\comentarioin{este comentario aportará? yo creo que no omitimos si lo ponemos}
Notemos que la característica de un cuerpo finito no puede ser 1 , ya
que $\1 \neq \0 $. El siguiente teorema nos provee más información
sobre la característica de un cuerpo finito.
\begin{theorem}\label{caracteristica primo}
Si $(F,+,\cdot)$ es un cuerpo finito de característica $p$, entonces
$p$ es un número primo.
\end{theorem}
\begin{proof}
Supongamos, por contradicción, que la característica $p$ de un cuerpo
finito $(F, +, \cdot)$ satisface que $p = ab$, con $1<a, b<p$. Luego,
\begin{eqnarray*}
	p\1  = \0 &\Rightarrow & (a b)\1 = \0 \\
        &\Rightarrow & (a b) (\1 \cdot \1) = \0 \\
	&\Rightarrow & (a\1) \cdot (b\1) =  \0 \quad\quad\quad \text{por la propiedad 5 de la proposición \ref{proposicion cuerpos}}\\
	&\Rightarrow& a \1  = \0 \ \text{ o } \ b\1  = \0 \quad\quad\quad \text{por la propiedad 6 de la proposición \ref{proposicion cuerpos}}
\end{eqnarray*}
En cualquiera de los dos casos, concluimos que existe un número $q <
p$ tal que $q > 1$ y $q \1 = \0$, lo cual contradice el hecho de que
$p$ es la característica de $(F, +, \cdot)$.
%Esto contradice nuestra elección de $p$, ya que cualquiera de los dos casos al sumar $\1$ una cantidad de veces menor que $p$ obtenemos que $\underbrace{\1 +\1 +\cdots +\1}_\text{$a$ o $b$ veces}  = \0 $. 
\end{proof}
El siguiente teorema relaciona la característica de un cuerpo con el
orden de este. Esto nos permite apreciar como es la estructura de un
cuerpo finito.
\begin{theorem}\label{orden cuerpo pot carac}
	Sea $(F,+,\cdot)$ un cuerpo finito y sea $p$ su
	característica. Entonces, el orden de $F$ es una potencia de
	$p$, es decir, $|F|=p^n$ para algún $n\in \mathbb{N}$.
\end{theorem}

\begin{proof}
Sea $(F,+,\cdot)$ un cuerpo finito de característica $p$, y
supongamos, por contradicción, que existe un primo $q\neq p$ tal que
$q$ divide a $|F|$. Como $(F,+)$ es un grupo finito, entonces por el
teorema de Cauchy (teorema \ref{teo cauchy}), existe un elemento $a\in
F$ que tiene orden $q$. Así, bajo la operación $+$ tenemos que
$\langle a\rangle = \{ 0 a, 1 a,...,(q-1) a\}$, y además tenemos que
$q a = \0$.
%y podemos concluir que
Adicionalmente, como $(F,+,\cdot)$ tiene característica $p$, se tiene
que $p x = \0 $ para todo $x \in F$, por la
proposición \ref{proposicion cuerpos} y el hecho de que $p \1 =\0 $
(en particular, sabemos que $\1 \cdot x = x$ y $\0 \cdot x
= \0$). Como $p$ y $q$ son primos relativos, por la identidad de
Bézout sabemos que existen $s,t\in \mathbb{Z}$ tales que $p s + q t
= \MCD(p,q) = 1$.\footnote{Si el lector no está familiarizado con la
identidad de Bézout puede
visitar \url{https://es.wikipedia.org/wiki/Identidad_de_B\%C3\%A9zout}.}
Nótese que podemos ocupar la identidad de Bézout ya que
$p,q\in\mathbb{N}$.
%(no necesariamente son elementos de $F$).
Luego,
\begin{eqnarray*}
	a &=& 1a\\
        &=& (p s + q t) a\\
        &=& (p s) a + (q t) a \quad\quad\quad \text{por la definición de $m b$ para $m \in \mathbb{Z}$ y $b \in F$}\\
        &=& (p s) (a \cdot \1) + (q t) (a \cdot \1)\\
	&=& (pa) \cdot (s\1) + (qa)\cdot (t\1) \quad\quad\quad \text{por la propiedad 5 de la proposición \ref{proposicion cuerpos}}\\
	&=& \0 \cdot (s\1) + \0 \cdot (t\1)\\
	&=& \0 + \0 \quad\quad\quad \text{por la propiedad 1 de la proposición \ref{proposicion cuerpos}}\\
        &=& \0
\end{eqnarray*}
Pero esto es una contradicción ya que el orden de $a$ en $(F,+)$ es
$q\geq 2$, por lo que $a \neq \0$ (puesto que $a = 1a$ y $1a \neq \0$).
\end{proof}
%\comentarioin{Aca hay que hacer un lazo que de mas continuidad.}

%\comentarioin{¿El siguiente ejemplo es necesario?}
%Un ejemplo particular de cuerpos finitos es el cuerpo $$F_p := (\mathbb{Z}_p,+,\cdot),$$ donde $p$ es un número primo. \comentarioin{desde aqui hacia abajo revisar}
%Este cuerpo finito $F_q$ cobra importancia debido a que podemos demostrar que cualquier cuerpo $K$ de orden $q$ es isomorfo a $F_q$, es decir, podemos $K$ y $F_q$ tienen la misma forma. Esto implica que al trabajar con $K$ podemos asumir que estamos trabajando en $F_q$.
%\comentarioin{revisar esta parte. Encontré demostraciones pero implicaría mucha más teoría. Quizás se podría hablar de qué es un isomorfismo y por qué podríamos trabajar con $F_q$}

%\comentarioin{Marcelo: hasta aquí llegué}

A continuación vamos a mostrar la forma estándar para construir un
cuerpo finito con característica $p$ y orden $p^n$, para $n \geq 1$.
Sea $h(X)$ un polinomio en $\mathbb{Z}_p[X]$ de grado $n$. Definimos
el conjunto
\begin{eqnarray*}
\mathbb{Z}_p[X]/h(X) & := & \{p(X) \mods h(X) \ \mid \ p(X)\in \mathbb{Z}_p[X]\}\\
&=& \{p(X) \mods (h(X),p) \ \mid \ p(X)\in \mathbb{Z}[X]\}
\end{eqnarray*}
En el caso de que $h(X)$ es irreducible en $\mathbb{Z}_p[X]$, se tiene
que $(\mathbb{Z}_p[X]/h(X), +, \cdot)$ es un cuerpo finito de
característica $p$ y orden $p^n$, donde las operaciones $+$ y $\cdot$
sobre polinomios son definidos en módulo $(h(X),p)$.
%\comentarioin{aca tengo que explicar como saber que un polinomio de grado mayor que 3 es irreducible en $Z_p[X]$, ya que al parecer, que no tenga raíces en $Z_p$ no implica que no sea irreducible.}
\begin{example}
Para construir un cuerpo de 9 elementos, debemos construir el cuerpo
$\mathbb{Z}_3[X]/h(X)$, donde $h(X)$ es un polinomio irreducible en
$\mathbb{Z}_3[X]$ y tiene grado 2. Para esto, notamos que $h(X) =
X^2+2X+2$ es irreducible en $\mathbb{Z}_3[X]$ puesto que $h(0) = 2$,
$h(1) = 2$ y $h(2) = 1$, y un polinomio de grado 2 es irreducible en
$\mathbb{Z}_3[X]$ si y sólo si no tiene raíces en este cuerpo. Tenemos
que:
\begin{eqnarray*}
	\mathbb{Z}_3[X]/(X^2+2X+2) &=& \{a_0 +a_1X \ \mid \
	a_0,a_1 \in \mathbb{Z}_p \}\\
	&=& \{0,\,1,\,2,\,X,1+X,\,2+X,\,2X,1+2X,2+2X\}
\end{eqnarray*}
Por lo tanto, tenemos que $(\mathbb{Z}_3[X]/(X^2+2X+2), +, \cdot)$ es
un cuerpo, donde $0, 1$ son los elementos neutros de la suma y la
multiplicación, respectivamente. Por ejemplo, $(1+2X) + (1+X) = 2$ en
este cuerpo, puesto que $(1+2X) + (1+X) \equiv 2 \modl 3$, de lo cual
se deduce que $(1+2X) + (1+X) \equiv 2 \modl (X^2+2X+2, 3)$. Siguiendo
esta línea de razonamiento, concluimos que $2+X$ es el inverso aditivo
de $1+2X$ puesto que $(1+2X) + (2+X) \equiv 0 \modl (X^2+2X+2, 3)$. De
la misma forma, $2X+2$ es el inverso multiplicativo de $X+1$ puesto
que
\begin{eqnarray*}
(X+1)(2X+2) & = & 2X^2 + 4X + 2\\
& = & 2X^2 + 4X + 4 - 2\\
& = & 2(X^2 + 2X + 2) - 2\\
& \equiv & 2(X^2 + 2X + 2) + 1 \modl 3,
\end{eqnarray*}
y de esto se deduce que $(X+1)(2X+2) \equiv 1 \modl (X^2+2X+2,
3)$. Finalmente, nótese que $(\mathbb{Z}_3, +, \cdot)$
es un subcuerpo de $(\mathbb{Z}_3[X]/(X^2+2X+2), +, \cdot)$. \qed
\end{example}

%\comentarioin{2 nuevos teoremas para no usar isomorfismos}
El siguiente teorema nos reafirma sobre las buenas propiedades de los polinomios sobre cuerpos.
\begin{theorem}\label{raices en polinomios sobre cuerpos}
Si $F$ es un cuerpo y $p(X)$ un polinomio no nulo de grado $k$ sobre $F[X]$, entonces $p(X)$ tiene a lo más $k$ raíces en $F$.
\end{theorem}
\begin{proof}
Vamos a demostrar el teorema por inducción sobre el grado del
polinomio. Para el caso base consideramos $k=0$. Como $p(X)$ es no
nulo, concluimos que $p(X)$ no tiene raíces y se cumple lo
requerido. Para el paso inductivo supongamos que todo polinomio no
nulo sobre $F[X]$ de grado $k$ tiene a los más $k$ raíces en $F$, y
sea $p(X)$ un polinomio sobre $F[X]$ de grado $k+1$. Si $p(X)$ no
tiene raíces en $F$ entonces el resultado se cumple trivialmente, por
lo que vamos a asumir que existe una raíz $c \in F$ de
$p(X)$. Luego, podemos escribir $p(X)$ como
\begin{eqnarray*}
p(X) & = & (X-c)\cdot q(X),
\end{eqnarray*}
donde $q(X)$ es un polinomio de grado $k$ (dejamos como ejercicio para
el lector la demostración de este propiedad).
%Es claro que $q(X)$ es de grado $k$, y
Por la hipótesis de inducción, sabemos que $q(X)$ tiene a lo más $k$
soluciones en $F$. Es importante observar que todo elemento que $d\in
F$ que no sea $c$ o una raiz de $q(X)$ no puede ser una raiz de
$p(X)$, ya que tendríamos que $(d-c)\neq \0$ y $q(d)\neq \0$, y por el
inciso \ref{inciso prop} de la proposición \ref{proposicion cuerpos}
sabemos inmediatamente que $p(d)\neq\0$ (si $F$ no fuera un cuerpo
esta propiedad no necesariamente se cumpliría).  Luego, $p(X)$ tiene a
lo más $k+1$ soluciones en $F$.
%\comentarioin{\ref{proposicion cuerpos} inciso \ref{inciso prop} es la respuesta a :No se por qué esta demostración no funcionaría con anillos que no son cuerpos. En que parte usan que tenemos un cuerpo? Yo creo que falta eso}
\end{proof}
Notemos que el resultado del teorema \ref{raices en polinomios sobre
cuerpos} no es cierto para todas las estructuras algebraicas. Por
ejemplo, si tomamos la estructura $(\mathbb{Z}_6,+,\cdot)$, que no es
un cuerpo, y el polinomio $p(X) = 2X+4$ sobre $\mathbb{Z}_6[X]$,
entonces tenemos que $p(X)$ es de grado 1 pero tiene 2 raíces: 1 y 4.

Para finalizar esta sección mostraremos un resultado fundamental sobre
teoría de cuerpos, que utilizaremos en el desarrollo del documento.
\begin{theorem}\label{subgrupo de cuerpo  es ciclico}
Sea $(F,+,\cdot)$ un cuerpo y $(G,\cdot)$ un subgrupo finito del grupo
de $(F\setminus\{\0\},\cdot)$.
%multiplicativo de este. 
Luego, este subgrupo es un grupo cíclico, es decir, existe un elemento
$a\in G$ tal que $\langle a\rangle = G$.
\end{theorem}
\begin{proof}
%Sean $a,b\in G$, con $O_G(a) = m$ y $O_G(b)=n$. Por el
%corolario \ref{corolario orden}, sabemos que existe un elemento en $G$
%con orden multiplicativo igual a $\MCM(m,n)$.
Sea $c\in G$ el elemento de mayor orden multiplicativo. Esto quiere
decir que para todo $d\in G$:
\begin{eqnarray*}
O_G(d) & \leq & O_G(c).
\end{eqnarray*}
Nótese que este elemento $c$ existe ya que $(G,\cdot)$ es un grupo
finito.  Mostraremos a continuación que el orden multiplicativo de
cualquier elemento en $G$ divide a $O_G(c)$. Sea $d\in G$. Por el
corolario \ref{corolario orden}, sabemos que existe un elemento $c'$
con $O_G(c') = \MCM(O_G(d),O_G(c))$. Por definición del mínimo común
múltiplo sabemos que $O_G(c)$ divide a $\MCM(O_G(d),O_G(c))$, de lo
que concluimos que $O_G(c)\leq O_G(c')$. Por otro lado, como $c'\in G$
y $c$ es el elemento de mayor orden, tenemos que $O_G(c')\leq
O_G(c)$. De esto concluimos que $O_G(c)=O_G(c')$. Así, sabemos que
$O_G(d)$ divide a $\MCM(O_G(d),O_G(c)) = O_G(c') = O_G(c)$, que es
exactamente lo que afirmamos.
	
	Por último, notemos que cada elemento $d\in G$ cumple con:
	\begin{eqnarray*}
		d^{O_G(c)} &=& d^{\alpha O_G(d)}\\
		&=& (d^{O_G(d)})^{\alpha}\\
		&=& \1^{\alpha}\\
		&=& \1
	\end{eqnarray*}
%
De lo anterior podemos concluir que cada elemento de $G$ es una raiz
del polinomio $p(X)=X^{O_G(c)} - 1$ sobre $F[X]$. Por el
teorema \ref{raices en polinomios sobre cuerpos}, sabemos que $p(X)$
tiene a lo más $O_G(c)$ raíces. Así, podemos afirmar que $|G|\leq
O_G(c)$. Finalmente, sabemos que el orden de cualquier elemento de un
grupo es menor o igual al orden del grupo, por lo que también podemos
afirmar que $O_G(c)\leq |G|$. De esta forma se deduce que
$O_G(c)= |G|$. Por lo tanto, se tiene que $\langle c\rangle = G$, puesto
que $\langle c\rangle \subseteq G$ (ya que $c \in G$), y queda
demostrado el~resultado.
\end{proof}



\subsection{Anillos de polinomios}


\section{Demostraciones Intermedias}
\label{sec-demos-inter}
\subsection{Demostración del lema \ref{lem-mcm}}
\label{app-lem-mcm}		

Dado $n \geq 1$, sea $d_n=\MCM(n)$, y para $m \in \{1, \ldots, n\}$, sea  
		\begin{eqnarray*}
		I(m,n) &=& \int_{0}^{1}x^{m-1}(1-x)^{n-m}dx.
		\end{eqnarray*}
Se tiene que
		\begin{eqnarray*}
			\int_{0}^{1}x^{m-1}(1-x)^{n-m}dx  &=&
				  \int_{0}^{1}\sum_{i=0}^{n-m}{n-m\choose i}(-1)^ix^{i+m-1}dx\\
				  &=&
				  \sum_{i=0}^{n-m}\bigg[{n-m\choose i}(-1)^i \int_{0}^{1}x^{i+m-1}dx\bigg]\\
				  &=& \sum_{i=0}^{n-m}\bigg[{n-m\choose i}(-1)^i \bigg(\frac{x^{m+i}}{m+i}\bigg)\bigg\rvert_0 ^1\bigg]\\
				  &=&\sum_{i=0}^{n-m}{n-m\choose i}(-1)^i \frac{1}{m+i}
		\end{eqnarray*}
		Del desarrollo anterior concluimos que $I(m,n)\cdot d_n\in \mathbb{N}$, ya que $1 \leq m+i\leq n$ para cada $i \in \{0, \ldots, n-m\}$ y $\int_{0}^{1}x^{m-1}(1-x)^{n-m}dx \geq 0$, puesto que en el intervalo $[0,1]$ se tiene $x \geq 0$ y $(1-x) \geq 0$.
%		\comentario{que paso aca? no hice new page}
		Por otro lado, si calculamos $I(m,n)$ por partes, tenemos:	
		\begin{eqnarray*}
		I(m,n)&=&\int_{0}^{1}x^{m-1}(1-x)^{n-m}dx\\
		&=&\bigg(x^{m-1}(-1)\frac{(1-x)^{n-m+1}}{n-m+1}\bigg)\bigg\rvert_0 ^1 -\int_0^1(-1)\frac{(1-x)^{n-m+1}}{n-m+1}(m-1)x^{m-2}dx\\
		&=& \frac{m-1}{n-m+1}\int_{0}^{1}x^{m-2}(1-x)^{n-m+1}dx\\
		&=& \frac{m-1}{n-m+1}\bigg[\bigg(x^{m-2}(-1)\frac{(1-x)^{n-m+2}}{n-m+2}\bigg)\bigg\rvert_0 ^1 -\int_0^1(-1)\frac{(1-x)^{n-m+2}}{n-m+2}(m-2)x^{m-3}dx\bigg]\\
		&=& \frac{(m-1)(m-2)}{(n-m+1)(n-m+2)}\int_{0}^{1}x^{m-3}(1-x)^{n-m+2}dx\\
		& \vdots &\\
		&=& \frac{(m-1)(m-2)\cdots(m-(m-1))}{(n-m+1)(n-m+2)\cdots(n-m+(m-1))}\int_{0}^{1}x^{m-m}(1-x)^{n-m+(m-1)}dx\\
		&=& \frac{(m-1)(m-2)\cdots 2\cdot 1}{(n-m+1)(n-m+2)\cdots(n-1)}\int_{0}^{1}(1-x)^{n-1}dx\\
		&=& \frac{(m-1)!}{\frac{(n-1)!}{(n-m)!}}\bigg((-1)\frac{(1-x)^n}{n}\bigg)\bigg\rvert_0^1\\
		&= &\frac{(m-1)!(n-m)!}{n(n-1)!} \ = \ \frac{m(m-1)!(n-m)!}{m \cdot n!} \ = \ \frac{m!(n-m)!}{m\cdot n!} \ = \ \frac{1}{m{n\choose m}}
		\end{eqnarray*}
%		\begin{align}
%		I(m,n)&=\int_{0}^{1}x^{m-1}(1-x)^{n-m}dx\nonumber\\
%		&=x^{m-1}(-1)(1-x)^{n-m+1}\frac{1}{n-m+1}\bigg\rvert_0 ^1 -\int_0^1(-1)\frac{(1-x)^{n-m+1}}{n-m+1}(m-1)x^{m-2}dx\nonumber\\
%		&=\frac{m-1}{n-m+1}\int_{0}^{1}x^{m-2}(1-x)^{n-m+1}dx\nonumber\\
%		&=\frac{m-1}{n-m+1}(x^{m-2}(-1)(1-x)^{n-m+2}\frac{1}{n-m+2}\bigg\rvert_0 ^1 -\int_0^1(-1)\frac{(1-x)^{n-m+2}}{n-m+2}(m-2)x^{m-3}dx)\nonumber\\
%		&=\frac{(m-1)(m-2)}{(n-m+1)(n-m+2)}\int_{0}^{1}x^{m-3}(1-x)^{n-m+2}dx\nonumber\\
%		&\hspace{5cm}\vdots\nonumber\\
%		&=\frac{(m-1)(m-2)\cdots(m-(m-1))}{(n-m+1)(n-m+2)\cdots(n-m+(m-1))}\int_{0}^{1}x^{m-m}(1-x)^{n-m+(m-1)}dx\nonumber\\
%		&=\frac{(m-1)(m-2)\cdots 2\cdot 1}{(n-m+1)(n-m+2)\cdots(n-1)}\int_{0}^{1}(1-x)^{n-1}dx\nonumber\\
%		&=\frac{(m-1)!}{(n-1)!/(n-m)!}(-1)\frac{(1-x)^n}{n}\bigg\rvert_0^1\nonumber\\
%		&=\frac{m(m-1)!(n-m)!}{m\cdot n(n-1)!}=\frac{m!(n-m)!}{m\cdot n!}=\frac{1}{m{n\choose m}}
%		\end{align}
		Dado que $I(m,n)\cdot d_n\in \mathbb{N}$, concluimos entonces que $m{n\choose m}\divi  d_n$, para todo $m \in \{1, \ldots, n\}$.
		En particular, se tiene que $n{2n\choose n}$ divide a $d_{2n}$ y por lo tanto se cumple también que $n{2n\choose n}$ divide a $d_{2n+1}$.
		Por otro lado, se tiene que $(2n+1){2n\choose n}=(n+1){2n+1\choose n+1}$, y de esto se deduce que $(2n+1){2n\choose n}$ divide a $ d_{2n+1}$ (dado que $(n+1){2n+1\choose n+1}$ divide a $d_{2n+1}$). Por lo tanto, sabemos que existen $\alpha, \beta \in \mathbb{N}$ tales que:
		\begin{eqnarray*}
		\alpha \cdot n{2n\choose n} & = & d_{2n+1}\\
		\beta \cdot (2n+1){2n\choose n} & = & d_{2n+1}.
		\end{eqnarray*}
Se tiene entonces que $\alpha \cdot n = \beta \cdot (2n+1)$, de lo cual se deduce que $\beta = \gamma \cdot n$ para $\gamma \in \mathbb{N}$, puesto que $\MCD(2n+1,n)=1$. Concluimos que $n(2n+1){2n\choose n}$ divide a $d_{2n+1}$,
%(esto ya que $n$ divide a $\frac{d_{2n+1}}{{2n\choose n}}$ y $(2n+1)$ divide a $\frac{d_{2n+1}}{{2n\choose n}}$, luego $d_{2n+1}/{2n\choose n} = n\cdot (2n+1)\cdot k$, con $k$ constante) 
de lo cual se deduce que $n(2n+1){2n\choose n} \leq d_{2n+1}$.
		Además, 
		\begin{eqnarray*}
		4^n \ = \ 2^{2n} \ = \ (1+1)^{2n} \ = \ \sum_{i=0}^{2n}{2n\choose i} \ \leq \ (2n+1)\max\bigg\{{2n\choose i} \,\bigg|\, 0\leq i\leq 2n\bigg\}\ =\ (2n+1){2n\choose n}.
		\end{eqnarray*}
		Multiplicando por $n$ ambos lados de la desigualdad y considerando que $n(2n+1){2n\choose n} \leq d_{2n+1}$, nos queda
		\begin{eqnarray*}
			n\cdot 4^n \ \leq \ n(2n+1){2n\choose n} \ \leq \ d_{2n+1}
		\end{eqnarray*}
		Así, si $n\geq 2$ entonces $d_{2n+1}\geq n\cdot 4^{n} = n\cdot 2^{2n}\geq 2\cdot 2^{2n}=2^{2n+1}$, y la propiedad mencionada en el lema se cumple para todos los impares mayores o iguales a 5. Además, si $n\geq 4$ entonces $d_{2n+1}\geq n\cdot 4^{n}\geq 4\cdot 4^{n}=2^{2n+2}$. Así, dado que $d_{2n+2}\geq d_{2n+1}$, se concluye que $d_{2n+2}\geq 2^{2n+2}$ y la propiedad mencionada en el lema se cumple para todos los pares mayores o iguales a 10.
%		(pares mayores o iguales a 9). 
Finalmente, como $d_8=840 \geq 2^8$, podemos afirmar que $\MCM(n) \geq 2^n$ para todo $n \geq 7$, lo cual concluye la demostración del lema. 

Como comentario final, nótese que la propiedad no se cumple para $n = 6$ puesto que $\MCM(6) = 60$ y $2^6 = 64$. Dejamos al lector como ejercicio verificar que la propiedad tampoco se cumple para $n=4$.
%$\forall N\geq 7$, $d_N\geq 2^N$ (se puede verificar que para los $N\leq 4$ y $N = 6$ no se cumple esta propiedad).
%		Con esto, queda demostrado el lema.
		
%\subsection{Demostración del Corolario \ref{mm-mcm}}
%	Sabemos por el Lema \ref{lem-mcm} se cumple para cualquier $m\geq 7$. Sea $p$ un primo tal que $p^{\alpha}$ es la mayor potencia de $p$ que divide a $d_{m}$. Como $p$ es primo, entonces $p^{\alpha}$ divide a algún $j$ en $1\leq j\leq m$\comentario{esto pensarlo}. Esto implica que $p^{\alpha}\leq m=p^{\log_p m}=p^{\frac{\log m}{\log p}}$\comentario{que feo este exponente}. Como esto pasa para todos los primos menores o iguales que $m$, entonces
%	\begin{align}
%		d_{m}\leq \prod_{p\leq m}p^{\frac{\log m}{\log p}}\nonumber
%	\end{align}\comentario{esto explicar un poco mas?}
%	Luego, por el Lema \ref{lem-mcm} tenemos que 
%	\begin{align}
%		&2^{m}\leq \prod_{p\leq m}p^{\frac{\log m}{\log p}},\hspace{0.5cm} \text{y aplicando logaritmo en ambos lados}\nonumber\\
%		\Leftrightarrow &m\log 2\leq \log (m)\cdot \pi(m)\nonumber\\
%		\Leftrightarrow &\pi(m)\geq \frac{m\log 2 }{\log m}\nonumber	
%	\end{align}
%	Esta desigualda se cumple para todo $m\geq 7$. Se puede verificar rápidamente que tambien se cumple cuando $m$ toma los valores de 4, 5 y 6. Con esto terminamos la demostración del corolario.
		
\subsection{Lema \ref{prop-1}: demostración de que $\log^2 n + 2\lfloor \log B \rfloor \leq \log^4 n$, para $n\geq 5$}
\label{app-prop-1}
Sabemos que si $n$ es un natural positivo entonces $\log n \leq n$. Supongamos que $n\geq 10$. Si desarrollamos la primera desigualdad tenemos que 
\begin{eqnarray*}
\log^5 n \leq n^5 &\Rightarrow& \lceil\log^5 n\rceil \leq \lceil n^5\rceil \ = \ n^5\\
	&\Rightarrow& B \leq n^5\\
	&\Rightarrow& \log B  \leq \log {n^5}\\
	&\Rightarrow& \lfloor\log B\rfloor  \leq \log {n^5} \ = \ 5\log n	 
\end{eqnarray*}

De lo anterior podemos afirmar que
\begin{eqnarray*}
	\log^2 n + 2\lfloor \log B \rfloor
	&\leq& \log^2 n + 10\log n\\
	&\leq& \log^2 n + 10\log^2 n \ = \ 11\log^2 n
\end{eqnarray*}
Además, como $11\leq\log^2 10$, entonces
\begin{eqnarray*}
	11\log^2 n &\leq& \log^2 10 \cdot\log^2 n\\
	&\leq& \log^2n\cdot \log^2n\\
	&=& \log^4 n
\end{eqnarray*}
De esta forma concluimos que si $n\geq 10$, entonces $\log^2 n + 2\lfloor \log B \rfloor \leq \log^4 n$. Para $n\in \{5,\ldots, 9\}$ podemos calcular manualmente los valores de ambos lados de la desigualdad, y ver que esta también se cumple.





\section{Algoritmos intermedios}
\label{sec-app-alg-int}
%{\bf Supuestos para el análisis de complejidad:} 
Para el análisis de los algoritmos mostrados en esta sección, es importante considerar que las operaciones aritméticas (suma, resta, multiplicación, división, resto, $\lfloor \log n \rfloor$, $\lfloor \frac{n+m}{2} \rfloor$), de comparación ($=$, $<$, $\leq$, $>$, $\geq$) y de asignación ($:=$) para números enteros pueden ser realizados en tiempo polinomial en el largo de la entrada. Por ejemplo, existe un algoritmo tal que, dados dos números enteros $n$ y $m$, calcula $n\cdot m$ en tiempo $O(\poly(\log n, \log m))$, vale decir, en tiempo polinomial en el largo de las entradas $n$ y $m$. De esta forma, el análisis de los algoritmos se enfoca en las otras operaciones realizadas en ellos.
% Además, si demostramos que un algoritmo puede ser ejecutado en tiempo polinomial bajo este supuesto, entonces sabemos que también pueden ser ejecutado en tiempo polinomial considerando el tiempo real de ejecución de las operaciones aritméticas y de comparación, ya que cada una de ellas puede ser llevada a cabo en tiempo polinomial. 


\subsection{Algoritmo de exponenciación rápida}
\label{app-fast_exp}
\begin{algorithm}[H]
\caption{\quad\textbf{EXP}}
\label{alg:fast_exp}
\hspace*{\algorithmicindent} \textbf{Input:} un par $n, k$, donde $n$ es un entero y $k$ es un natural positivo\\
\hspace*{\algorithmicindent} \textbf{Output:} el valor de $n^k$
\begin{algorithmic}[1]
	\IF {$k = 0$}
		\RETURN $1$
    \ELSIF {$k = 1$} 
       \RETURN $n$
    \ELSIF {$k$ es par}
       \STATE $val :=\textbf{EXP}(n,\frac{k}{2})$
       \RETURN $val\cdot val$
    \ELSE
   	   \STATE $val :=\textbf{EXP}(n,\frac{k-1}{2})$
	   \RETURN $val\cdot val\cdot n$   
	   \ENDIF         
\end{algorithmic}
\end{algorithm}
El algoritmo \ref{alg:fast_exp} es un algoritmo recursivo. El número de llamadas recursivas realizadas por el algoritmo 
%tiempo de ejecución del algoritmo 
está dado por la siguiente ecuación de recurrencia:
% bajo los supuestos hechos en este capítulo:
\begin{eqnarray*}
	T(k) &=& 
	\begin{cases}
	T(\lfloor\frac{k}{2}\rfloor) + 1& \text{si } k > 2\\
	1 & \text{si } k = 2 
	\end{cases}
\end{eqnarray*}
De esto concluimos la cantidad de llamadas recursivas realizadas por el algoritmo \ref{alg:fast_exp} es $O(\log k)$, vale decir, lineal en el tamaño de la entrada~$k$. Además, las operaciones artiméticas, de comparación y de asignación pueden ser realizadas en tiempo $O(\poly_1(\log n^k)) = O(\poly_1(k \cdot \log n))$, dado que los números enteros utilizados en los distintos pasos del algoritmo están acotados por $n^k$. Por lo tanto, el algoritmo \ref{alg:fast_exp} funciona en tiempo $O(\log k \cdot \poly_1(k \cdot \log n))$, vale decir, en tiempo $O(\poly_2(\log n,k))$. 
	
\subsection{Algoritmo para verificar si existe un $m\in \{i,...,j\}$ tal que $n = m^k$}
\label{app-tiene_raiz_entera}
\begin{algorithm}[H]
\caption{\quad\textbf{TieneRaízEntera}}
\label{alg:tiene_raiz_entera}
\hspace*{\algorithmicindent} \textbf{Input:} una secuencia $n,k,i,j$, donde $n,k,i,j$ son números naturales  tales que $i \leq j \leq n$\\
\hspace*{\algorithmicindent} \textbf{Output:} \textbf{true} si existe un $m\in \{i,...,j\}$ tal que $n = m^k$, \textbf{false} si no 
\begin{algorithmic}[1]
   	\IF {$i=j$}
   		\IF {\textbf{EXP}$(i,k)=n$}
   			\RETURN \TRUE
   		\ELSE
   			\RETURN \FALSE
   		\ENDIF
   	\ELSIF {$i<j$}
   		\STATE $p:=\lfloor \frac{i+j}{2}\rfloor$
   		\STATE $val:=\textbf{EXP}(p,k)$
   		\IF {$val = n$}
   			\RETURN \TRUE
   		\ELSIF {$val<n$}
   			\RETURN \textbf{TieneRaízEntera}$(n,k,p+1,j)$
   		\ELSE
                        \RETURN \textbf{TieneRaízEntera}$(n,k,i,p-1)$
%			\IF {$i \leq p -1$}  				
%			\ELSE
%				\RETURN \FALSE
%			\ENDIF
   		\ENDIF
        \ELSE
                \RETURN \FALSE
   	\ENDIF
\end{algorithmic}
\end{algorithm}
El algoritmo \ref{alg:tiene_raiz_entera} en el peor caso realiza  $O(\log (j - i))$ llamadas recursivas, vale decir, $O(\log n)$ llamadas recursivas puesto que $j - i \leq j \leq n$. 
Dado el análisis de la sección~\ref{app-fast_exp}, las llamadas a la función \textbf{EXP} pueden ser realizadas en tiempo $O(\poly_1(\log n,k))$, dado que estas llamadas son de la forma \textbf{EXP}$(p,k)$ con $p \leq j \leq n$. Además, las operaciones artiméticas, de comparación y de asignación son realizadas en tiempo $O(\poly_2(\log n, k))$, dado que $i \leq j \leq n$. 
%Dados los supuestos iniciales de esta sección, en cada una de estas llamadas lo que predomina en orden de complejidad es la llamada a la función \textbf{EXP}, que demora $O(\log k)$ en el peor caso (\ref{app-fast_exp}). 
De esta forma, concluimos que el algoritmo \ref{alg:tiene_raiz_entera} funciona en tiempo
%está función tiene complejidad $O(\log (j-i)\cdot\log k)$.   
$O(\log n \cdot (\poly_1(\log n,k) + \poly_2(\log n, k))$, vale decir, en tiempo $O(\poly_3(\log n, k))$.


\subsection{Algoritmo para verificar si un número es potencia de otro}
\label{app-es_potencia}
\begin{algorithm}[H]
\caption{\quad\textbf{EsPotencia}}
\label{alg:es_potencia}
\hspace*{\algorithmicindent} \textbf{Input:} un número natural $n>1$\\
\hspace*{\algorithmicindent} \textbf{Output:} \textbf{true} si $n = a^b$ para $a,b \in \mathbb{N}$ tales que $b \geq 2$, \textbf{false} si no lo es
\begin{algorithmic}[1]
   \IF {$n\leq 3$}
    	\RETURN \FALSE
    \ELSE
    	\FOR {$k = 2$ \TO $\lfloor \log n\rfloor$}
    		\IF {\textbf{TieneRaízEntera}$(n,k,1,n)$}
    			\RETURN \TRUE
    		\ENDIF
		\ENDFOR    		
    	\RETURN \FALSE
    \ENDIF	 
\end{algorithmic}
\end{algorithm}
El algoritmo \ref{alg:es_potencia} realiza a lo más $(\lfloor \log n\rfloor - 1)$ llamadas al algoritmo \textbf{TieneRaízEntera}$(n,k,1,n)$ presentado en la sección~\ref{app-tiene_raiz_entera}, el cual tiene orden de complejidad $O(\poly_1(\log n, k))$. Como $k \leq \lfloor \log n\rfloor \leq \log n$ para $n > 1$, 
%De esta forma 
concluimos que  
%O(\log n\cdot \log (\log n))\leq O(\log n^2)$ en el peor caso (ver \ref{app-tiene_raiz_entera}).  De esta forma, este 
el algoritmo \ref{alg:es_potencia} funciona en tiempo $O(\log n \cdot \poly_1(\log n, \log n))$, vale decir, en tiempo~$O(\poly_2(\log n))$.
%\comentario{revisar los de $\log \log n$}


\subsection{Algoritmo para calcular el máximo común divisor entre dos números}
\label{app-mcd}
\begin{algorithm}[H]
\caption{\quad\textbf{MCD}}
\label{alg:mcd}
\hspace*{\algorithmicindent} \textbf{Input:} un par $a,b$, donde $a$ y $b$ son números naturales\\
\hspace*{\algorithmicindent} \textbf{Output:} el máximo común divisor de $a$ y $b$
\begin{algorithmic}[1]
   	\IF {$a = 0$ \AND $b=0$}
    	\RETURN \textbf{error}
    \ELSIF {$a=0$}
    	\RETURN $b$
    \ELSIF {$b=0$}
    	\RETURN $a$
    \ELSIF {$a\geq b$}
    	\RETURN $\textbf{MCD}(b,\, a \mods b)$
	\ELSE
		\RETURN $\textbf{MCD}(a,\, b \mods a)$    	
   	\ENDIF
\end{algorithmic}
\end{algorithm}
Para demostrar que el algoritmo es correcto, es necesario demostrar que para $a \geq b > 0$, se tiene que $\MCD(a,\, b) = \MCD(b,\, a \!\! \mod b)$. 
Para el análisis de la complejidad del algoritmo \ref{alg:mcd} podemos usar el hecho de que si $a\geq b > 0$, entonces $a \!\! \mod b<\frac{a}{2}$. Las demostraciones de estas dos propiedades quedan propuestas para el lector. De esta forma, 
%si suponemos sin perdida de generalidad que $a \geq b$,
tenemos que el algoritmo \ref{alg:mcd} calcula $\MCD(a,b)$ 
%concluimos que el algoritmo  (sin pérdida de generalidad) que $a\geq b$, la complejidad del algoritmo \ref{alg:mcd}  es 
% $\MCD$ queda 
y funciona en orden~$O(\max\{\log a,\log b\})$.
%\comentario{no se si explicar mas aca}
%\comentario{Marcelo: basta considerar números naturales para este algoritmo?}


\subsection{Algoritmo para calcular el orden multiplicativo $O_r(n)$}
\label{app-orden_multiplicativo}
\begin{algorithm}[H]
\caption{\quad\textbf{OrdenMultiplicativo}}
\label{alg:mult_ord}
\hspace*{\algorithmicindent} \textbf{Input:} un par $n,r$, donde $n$ y $r$ son números naturales tales que $n,r > 1$\\
\hspace*{\algorithmicindent} \textbf{Output:} el valor de $O_r(n)$ si $\MCD (n,r) = 1$, el valor -1 en caso contrario.
\begin{algorithmic}[1]
   	\IF {$\textbf{MCD}(n,r) \neq 1$}
    	\RETURN -1
   	\ENDIF
	\STATE{$\textit{val} :=1$}
   	\FOR {$k := 1$ \TO $r-1$}
%   		\IF {$k=1$}
%   			\STATE $val:= n\mod r$
%   			\IF {$val = 1$}
%   				\RETURN $k$
%   			\ENDIF
%   		\ELSE
   			\STATE $\textit{val} := (val\cdot n)\mod r$
   			\IF {$\textit{val} = 1$}
   				\RETURN $k$
   			\ENDIF
   		%\ENDIF
   	\ENDFOR
   			
\end{algorithmic}
\end{algorithm}
El algoritmo \ref{alg:mult_ord} comienza calculando el máximo común divisor entre $n$ y $r$, lo cual de acuerdo al análisis realizado en la sección \ref{app-mcd} toma tiempo $O(\max\{\log n, \log r\})$. Luego, entra al ciclo, y en el peor de los casos retorna cuando $k = r-1$, es decir, realiza $O(r)$ iteraciones. 
Además, las operaciones artiméticas, de comparación y de asignación realizadas en el ciclo toman tiempo $O(\poly_1(\log n, \log r))$. De esta forma, 
%De esta forma, bajo los supuestos realizados en esta sección, 
concluimos que el algoritmo es de orden $O(\max\{\log n, \log r\} + r \cdot \poly_1(\log n, \log r))$, vale decir, es de orden~$O(\poly_2(\log n, r))$. 
%\comentario{Marcelo: por favor revisar versión simplificada del algoritmo}

%\subsection{Algoritmo para hacer módulo polinomio (esto lo pensaba sacar)}
%\label{app-mod-pol}
%\begin{algorithm}[H]
%\caption{ModPol}
%\label{alg:mod_pol}
%\hspace*{\algorithmicindent} \textbf{Input:} una tupla $(q(X), n, r)$, donde $q(X)$ está en su forma canónica (en forma de lista?)\\
%\hspace*{\algorithmicindent} \textbf{Output:} 
%\begin{algorithmic}[1]
%	\IF {$\deg(q(X))\geq r$}
%		\FOR {$i = r$ \TO $\deg(q(X))$}
%			\STATE $q[i\mod r] = q[i\mod r] + q[i]$
%		\ENDFOR	
%	\ENDIF	
%	%aca hacer modulo n
%	\FOR {$coeficiente$ \textbf{in} $q(X)$}
%		\STATE $coeficiente = coeficiente \mod n$
%	\ENDFOR
%	\RETURN $q(X)$        
%\end{algorithmic}
%\end{algorithm}
%\comentario{explicar por que tiene complejidad polinomial, y arreglar input output, hacer modulo a los coeficientes}

%El algoritmo \ref{alg:mod_pol} retorna el polinomio $q(X)$ en $F_n[X]/(X^r-1)$. Este ocupa el hecho de que $X^r\equiv 1 \modulo$. De esta forma, $X^i\equiv X^{i\mod r}$ para $1\leq i\leq deg(q)$. 


\subsection{Algoritmo de exponenciación rápida para polinomios en módulo $(X^r-1,n)$}
\label{app-fast_exp_mod}
\begin{algorithm}[H]
\caption{\quad\textbf{ExpMod}}
\label{alg:fast_exp_mod}
\hspace*{\algorithmicindent} \textbf{Input:} una tupla $(q(X), k, r, n)$\\
\hspace*{\algorithmicindent} \textbf{Output:} el valor de $q(X)^k \!\! \modulo$
\begin{algorithmic}[1]
	\IF {$k = 0$}
		\RETURN $1$
    \ELSIF {$k = 1$}
    	\RETURN $q(X)  \!\! \modulo$
   	\ELSIF {$k$ es par}
    	\STATE $val :=\textbf{ExpMod}(q(X),\frac{k}{2}, r, n)$
      	\RETURN $(val\cdot val)  \!\! \modulo$
   	\ELSE
   		\STATE $val :=\textbf{ExpMod}(q(X),\frac{k-1}{2}, r, n)$
	  	\RETURN $(val\cdot val\cdot q(X))  \!\! \modulo$   
	  	\ENDIF         
\end{algorithmic}
\end{algorithm}
%\comentario{explicar por que tiene complejidad polinomial, y arreglar input output, hacer modulo a los coeficientes}
%\comentario{el algoritmo lo hice para $X^r -1$ porque facilita mucho}

Para analizar la complejidad de este algoritmo, primero es necesario mencionar cuál es la complejidad 
de las operaciones aritméticas para polinomios. Un polinomio 
\begin{eqnarray*}
q(X) & = & \sum_{i=0}^{k-1} a_i X^i
\end{eqnarray*}
es representado como una tupla $(a_{k-1}, \ldots, a_0)$ con $k$ elementos. En particular, si consideramos polinomios en módulo $n$, entonces cada coeficiente $a_i \in \{0, \ldots, n-1\}$, y el tamaño de la tupla $(a_{k-1}, \ldots, a_0)$ es $O(k \cdot \log n)$. De manera general, un polinomio $q(X)$ en módulo $n$ es representado por una tupla de tamaño $O(\grado(q(X)) \cdot \log n)$, donde $\grado(q(X))$ es el grado del polinomio $q(X)$. Las operaciones aritméticas suma, resta, multiplicación, división y resto para polinomios en módulo $n$ pueden ser realizados en tiempo polinomial en el largo de la entrada. Por ejemplo, existe un algoritmo tal que, dados dos polinomios $q_1(X)$ y $q_2(X)$, calcula $q_1(X) \cdot q_2(X)$ en módulo $n$ en tiempo $O(\poly(\grado(q_1(X)) \cdot \log n,\, \grado(q_2(X)) \cdot \log n))$, vale decir, en tiempo polonomial en el largo de las entradas $q_1(X)$ y $q_2(X)$. Nótese que $O(\poly(\grado(q_1(X)) \cdot \log n,\, \grado(q_2(X)) \cdot \log n))$ es equivalente a $O(\poly_1(\grado(q_1(X)), \grado(q_2(X)), \log n))$, por lo que esta última notación es usada cuando consideramos la complejidad de las operaciones aritméticas para polinomios en módulo $n$. 


El algoritmo \ref{alg:fast_exp_mod} realiza $O(\log k)$ llamadas recursivas.
% en las que debe hacer multiplicación de polinomios, y obtener la representación del polinomio en el anillo $F_n[X]/(X^r-1)$. Para realizar la operación $p(X)\modulo$ para un polinomio $p(X)$ cualquiera se necesitan $O(deg(p(X)))$.
En cada una de estas llamadas debe realizar operaciones aritmeticas y de comparación para números naturales, por ejemplo verificar si $k$ es par, y operaciones aritmeticas y de asignación ($:=$) para polinomios en módulo $n$, por ejemplo calcular $q(X) \!\! \modulo$. Dado que $\grado(X^r - 1) = r$ y los polinomios almacenados en la variable $val$ son de grado menor a $r$, tenemos que cada llamada recursiva puede ser realizada en tiempo $O(\poly_1(\log k, \grado(q(X)), r, \log n))$. Concluimos entonces que el algoritmo \ref{alg:fast_exp_mod} funciona en tiempo $O(\log k \cdot \poly_1(\log k, \grado(q(X)), r, \log n))$, vale decir, en tiempo $O(\poly_2(\log k, \grado(q(X)), r, \log n))$.



\end{document}
