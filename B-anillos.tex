\subsection{Definiciones y propiedades básicas}

Un anillo es una estructura algebraica que cuenta con dos operaciones internas, a diferencia de los grupos, que solo cuentan con una. En el contexto de un anillo, a la primera de estas operaciones se le llama \emph{suma}, y a la segunda, \emph{multiplicación}. En ese sentido, usaremos los símbolos «$+$» y «$\cdot$» para referirnos a ellas, a pesar de que la naturaleza de estas operaciones puede ser muy distinta dependiendo del anillo concreto en el que estemos trabajando.

\begin{definition}[Anillo, anillo conmutativo]
    Un \emph{anillo} es una terna $(R, +, \cdot)$, donde $R$ es un conjunto y $+ \colon R \times R \to R$ y $\cdot \colon R \times R \to R$ son funciones (totales) que satisfacen los siguientes axiomas\footnote{Algunos autores 
    no incluyen el axioma $3$ en la definición de anillo, y usan el término \emph{anillo con identidad} para referirse a la estructura algebraica que nosotros llamamos simplemente \emph{anillo}. No hay una convención universal sobre esto, por lo que sugerimos siempre revisar la definición precisa de anillo en el material que se esté utilizando.}
    \begin{enumerate}
        \item $(R, +)$ es un grupo abeliano.
        \item $\cdot$ es asociativa: para todos $a,b,c \in R$ se cumple que $(a \cdot b) \cdot c = a \cdot (b
\cdot c)$.

	\item Existe un elemento $\1 \in G$ (llamado \emph{identidad}) tal que para todo $a \in R$ se cumple que $a \cdot \1 = \1 \cdot a = a$

   \item $\cdot$ distribuye sobre $+$, es decir, para todos $a, b, c \in R$ se cumple que $a \cdot (b + c) = (a \cdot b) + (a \cdot c)$ y $(a+b) \cdot c = (a \cdot c ) + (b \cdot c).$
    \end{enumerate}
    Decimos que $(R, +, \cdot)$ es un \emph{anillo conmutativo} si es un anillo en el que la operación $\cdot$ es conmutativa.\hfill$\blacksquare$
    \end{definition}

Notemos que el axioma $4$ es el único que conecta ambas operaciones, y lo hace de la misma manera en que interactúan la suma y multiplicación usuales.

En el apéndice de teoría de grupos discutimos que el símbolo $e$ estaba siempre reservado para el neutro. Sin embargo, en el contexto de un anillo, usaremos el símbolo $\0$ para referirnos al neutro de la operación $+$. La letra negrita sirve para diferenciar dicho elemento del número $0$. Siempre usaremos la notación aditiva para el grupo $(R, +$), de modo que $-a \in R$ denotará al inverso de $a \in R$ bajo $+$. En la misma línea, dados $a \in R$ y $n \in \mathbb{Z}$, escribiremos $na$ para denotar a la $n$-ésima potencia de $a$ bajo $+$.

También usaremos la negrita para diferenciar el elemento $\1 \in R$ del número $1$. Para que esta notación tenga sentido, primero debemos probar que la identidad de un anillo es única. La demostración es idéntica a la que hicimos para grupos.

\begin{proposition}
    La identidad de un anillo es única.
\end{proposition}
    
    \begin{proof}
    Sea $(R, +, \cdot)$ un anillo, y supongamos que $\1, \1' \in G$ son ambos identidades. Como $\1$ es identidad, tenemos que $\1 \circ \1' = \1'$. Pero $\1'$ también es identidad, así que $\1 \circ \1' = \1$. Por transitividad, concluimos que $\1 = \1'$.
    \end{proof}

Dado un $a \in R$ y un entero positivo $n$, escribiremos $a^n$ para denotar al elemento $a$ operado bajo $\cdot$ consigo mismo $n$ veces. También definimos $a^0 = \1$. Notemos que, \textit{a priori}, no tiene sentido escribir $a^n$ si $n$ es un entero negativo, pues podría no existir un inverso para $a$ bajo $\cdot$.

Con el objetivo de no sobrecargar la notación, es usual omitir el símbolo «$\cdot$» para escribir multiplicaciones. Además, a menos que un par de paréntesis indique lo contrario, la convención es que las multiplicaciones se realizan antes que las sumas. Así, los axiomas de distributividad se suelen escribir como $a(b+c) = ab+ac$ y $(a+b)c = ac + bc$. 

Nos centraremos en el estudio de los anillos conmutativos, pues no necesitaremos trabajar con ningún anillo no conmutativo en este documento.

\begin{example} 
Sabemos que $(\mathbb{Z}, +)$ es un grupo abeliano (ejemplo \ref{ejemplo_Z}). Si consideramos además la multiplicación usual de enteros, obtenemos un anillo conmutativo: $(\mathbb{Z}, +, \cdot)$. \hfill$\blacksquare$
\end{example}

\begin{example}
Dado un entero $n \geq 1$, sabemos que $(\mathbb{Z}, +)$ es un grupo abeliano (ejemplo \ref{grupo_ciclico}). Si consideramos además la multiplicación usual en módulo $n$, obtenemos un anillo conmutativo: $(\mathbb{Z}_n, +, \cdot)$. \hfill$\blacksquare$
\end{example}

A continuación presentamos algunas propiedades básicas de los anillos conmutativos.

\begin{proposition}\label{propiedades_anillos}
    Sea $(R, +, \cdot)$ un anillo conmutativo. Se cumplen las siguientes propiedades:
    \begin{enumerate}
    \item Para todo $a\in R$ se cumple que $a\cdot \0 =\0\cdot a= \0$
    
    \item Para todo $a,b\in R$ se cumple que $(-a)\cdot b=-(a\cdot b) = a\cdot (-b)$
    
    \item Para todo $a \in R$ y $n \in \mathbb{Z}$ se cumple que $-(na) = (-n)a = n(-a)$.
    
    \item Para todo $a \in F$ y $p,q \in \mathbb{Z}$ se cumple que $(pq)a = p(qa)$.
    
    \item Para todo $a,b\in R$ y $n \in \mathbb{Z}$ se cumple que $n(a\cdot b) = (na)\cdot b = a \cdot (nb)$. De esto y la propiedad 4 se concluye que para todo $a,b\in R$ y $n,m \in \mathbb{Z}$, se cumple que $(nm)(a\cdot b) = (na)\cdot (mb)$.   
    \end{enumerate}
    \end{proposition}
    
    \begin{proof} \hfill
        \begin{enumerate}
            \item 
            Sabemos que $\0+\0 = \0$, pues $\0$ es el neutro aditivo. Luego:
            \begin{eqnarray*}
    \0+\0 = \0		&\Rightarrow & a\cdot(\0+\0) = a\cdot \0\\
                &\Rightarrow &a\cdot \0+a\cdot \0 = a\cdot \0
            \end{eqnarray*}
        Como $(R, +)$ es un grupo abeliano, podemos cancelar $a \cdot \0$ (proposición \ref{cancelacion_grupos}). Así, obtenemos que $a \cdot \0 = \0$. Por la conmutatividad de la multiplicación, también tenemos que $\0 \cdot a=\0 $.
    
        \item Tenemos que:
            \begin{eqnarray*} (-a)\cdot b + a \cdot b & = & ((-a) +
            a) \cdot b\\ & = & \0 \cdot b\\ 
            & = & \0 \quad\quad\quad \text{por la propiedad 1}
            \end{eqnarray*}
    Así, tenemos que $-(a \cdot b) = (-a)\cdot b$. Además, por esta propiedad y la 
            conmutatividad de la multiplicación obtenemos $a\cdot (-b) = (-b) \cdot a = -(b\cdot a) = -(a\cdot b)$.
    
            \item Si $p = 0$, concluimos que $-(pa) = (-p)a = p(-a)$ por
            la definición de la notación $nb$ (para $n \in \mathbb{Z}$ y $b \in
            F$). Si $p > 0$, tenemos que:
    \begin{eqnarray*}
    (-p)a + pa & = & p(-a) + pa\\
    & = & \underbrace{(-a) +\cdots + (-a)}_\text{$p$ veces} \ + \ \underbrace{a +\cdots + a}_\text{$p$ veces}\\ 
    & = & \underbrace{((-a) + a) + \cdots + ((-a) + a)}_\text{$p$ veces}\\
    & = & \underbrace{\0 + \cdots + \0}_\text{$p$ veces}\\
    & = & \0
    \end{eqnarray*}
    Por lo tanto, $-(pa) = (-p)a$. Además, por definición tenemos que $p(-a) = (-p)a = -(pa)$. El caso $p < 0$ es dejado como ejercicio para el lector (para resolverlo es útil considerar el caso $p > 0$ y la propiedad $p = -|p|$). 
    
    \item Si $p = 0$ o $q = 0$, concluimos que $(pq)a = p(qa)$ por
            la definición de la notación $nb$ (para $n \in \mathbb{Z}$ y $b \in
            F$). Si $p > 0$ y $q > 0$, tenemos que:
    \begin{eqnarray*}
    (pq)a &=& \underbrace{a +\cdots + a}_\text{$pq$ veces}\\
     &=& \underbrace{c +\cdots + c}_\text{$p$ veces} \quad\quad\quad \text{para }
     c = \underbrace{a +\cdots + a}_\text{$q$ veces}\\
     & = & \underbrace{qa +\cdots + qa}_\text{$p$ veces}\\
     & = & p(qa)
    \end{eqnarray*}
    Si $p < 0$ y $q < 0$, tenemos por la propiedad anterior que:
    \begin{eqnarray*}
    (pq)a &=& ((-p)(-q))a\\
     &=& (-p)((-q)a)\\
     &=& (-p)(-(qa)) \quad\quad\quad \text{por la propiedad 3}\\
     &=& p(-(-(qa))) \quad\quad\quad \text{por la propiedad 3}\\
     &=& p(qa)
    \end{eqnarray*}
    Los casos faltantes pueden ser demostrados utilizando las mismas ideas, y son dejados como ejercicios para el lector.
    
        \item Si $p = 0$, concluimos que $p(a\cdot b) = (pa)\cdot b = a \cdot (pb)$ por la propiedad 1 y la definición de la notación $nb$ (para $n \in \mathbb{Z}$ y $b \in F$). Si $p > 0$, tenemos que:
        \begin{eqnarray*}
            p(a\cdot b)&=&\underbrace{a\cdot b +\cdots + a\cdot b}_\text{$p$ veces}\\
            &=&(\underbrace{a +\cdots + a}_\text{$p$ veces}) \cdot b\\
            &=&(pa) \cdot b
        \end{eqnarray*}		
    Además, utilizando esta propiedad y la conmutatividad de la multiplicación obtenemos $p(a\cdot b) = p(b \cdot a) = (pb) \cdot a = a \cdot (pb)$. Si $p < 0$, utilizando la propiedad anterior concluimos que:
        \begin{eqnarray*}
            p(a\cdot b)&=& (-|p|)(a \cdot b)\\
            &=& |p|(-(a \cdot b))\\
            &=& |p|((-a) \cdot b) \quad\quad\quad \text{por la propiedad 2}\\
            &=& (|p|(-a)) \cdot b\\
            &=& ((-|p|)(a)) \cdot b\\
            &=& (pa) \cdot b
        \end{eqnarray*}
        Finalmente, utilizando esta propiedad y la conmutatividad de la multiplicación obtenemos $p(a\cdot b) = a \cdot (pb)$. 
        \qedhere
        \end{enumerate}
    \end{proof}