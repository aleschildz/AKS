        Como es mostrado en la siguiente sección, el test de
        primalidad AKS está basado en el siguiente lema que
        caracteriza a los números primos.
	\begin{lemma}\label{lem-2.1}
		Sea $a\in \mathbb{Z}$ y $n \in \mathbb{N}$, tales que $n\geq 2$ y $\MCD(a,n)=1$. Entonces
		\begin{eqnarray*}
			n \text{ es primo} & \Leftrightarrow & (X+a)^n \equiv X^n+a \modl n
		\end{eqnarray*}
	\end{lemma}
	\begin{proof}
		Antes de demostrar ambas direcciones, vamos a desarrollar el polinomio $P(X)=(X+a)^n - (X^n+a)$. 
		%desarrollaremos un polinomio que será fundamental para. Sea $P(X)=(X+a)^n - (X^n+a)$. 
Por el teorema del binomio, tenemos que $(X+a)^n=\sum_{i=0}^{n}{n\choose i}X^i a^{n-i}$. Luego,
		\begin{eqnarray}
			P(X) &=&  \bigg(\sum_{i=0}^{n}{n\choose i}X^i a^{n-i}\bigg)-X^n -a \nonumber\\ 
			&=&  \bigg(\sum_{i=1}^{n-1}{n\choose i}X^i a^{n-i}\bigg) + X^n + a^n -X^n -a\nonumber\\
			&=&  \sum_{i=1}^{n-1}{n\choose i}X^i a^{n-i} + (a^n -a) \label{eq:one}
		\end{eqnarray}		 
Con esta propiedad podemos demostrar ambas direcciones del lema. 
\begin{itemize}
\item[($\Rightarrow$)]
%		\subsubsection*{($\Rightarrow$)}
			Supongamos que $n$ es primo. Ya que $\MCD(a,n)=1$, el pequeño teorema de Fermat nos dice que $a^{n-1}\equiv 1  \modl n$. Luego, $a^n\equiv a \modl n$. Por lo tanto, de $\eqref{eq:one}$ nos queda que
			\begin{eqnarray*}
				P(X) &\equiv  & \sum_{i=1}^{n-1}{n\choose i}X^i a^{n-i} \modl n. %\label{eq:one}
			\end{eqnarray*} 
			Además, 
			%como $n$ es primo, 
			para cada $i \in \{1, \ldots, n-1\}$ tenemos que
			\begin{eqnarray*}
				{n\choose i} \ = \ \frac{n!}{(n-i)!i!} \ = \ n\cdot \frac{(n-1)!}{(n-i)!i!} \ = \ n\cdot c_1, 
				%\equiv 0 \modl n
			\end{eqnarray*}
			donde $c_1 \in \mathbb{N}$. 
			Nótese que esto se cumple ya que $c_1=\frac{(n-1)!}{(n-i)!i!}$, $i \in \{1, \ldots, n-1\}$ y $n$ es primo, por lo que para cada divisor $d$ de $(n-i)! i!$ tal que $d \neq 1$, se tiene que $d \nodiv n$ ya que $d < n$. 
			 %los factores de $(n-i)!$ y $i!$ son menores que $n$, entonces $(n-i)!i! \nodiv n $, por lo tanto $c_1 \in \mathbb{N}$. Así­ nos queda
			 Por lo tanto, tenemos que ${n\choose i} \equiv 0 \modl n$ para cada $i \in \{1, \ldots, n-1\}$, y así nos queda
			\begin{eqnarray*}
				P(X) \ \equiv \ \sum_{i=1}^{n-1}{n\choose i}X^i a^{n-i} \ \equiv \ 0 \modl n
			\end{eqnarray*}
			Por lo tanto, se cumple que $(X+a)^n \equiv X^n+a \modl n$.
		
%		\subsubsection*{($\Leftarrow$)}
\item[($\Leftarrow$)]
			Esta dirección se demostrará considerando el contrapositivo. Es decir, mostraremos que si $n$ no es primo, entonces $ (X+a)^n \not\equiv X^n+a$ en módulo $n$.
			
			Como $n$ es compuesto, debe existir un primo $q$ %tal 
			que %$q$ 
			divide a $n$.
			Sea $q^k$ la potencia máxima de $q$ %tal 
			que %$q^k $ 
			divide a $n$, vale decir, existe un entero $ c $ tal que $n=cq^k$ y $q$ no divide a $c$. A continuación, vamos a demostrar que $q^k\nodiv{n \choose q}$. Para %ver que
			esto %se cumple, 
			considere que
			\begin{eqnarray}
				\nonumber {n \choose q} &=& \frac{n!}{(n-q)!q!}\\
				&=& \nonumber \frac{n(n-1)\cdot\cdot\cdot(n-q+1)(n-q+1)}{q!}\\
				&=&  \nonumber \frac{c q^k(n-1)\cdot\cdot\cdot(n-q+1)}{q(q-1)!}\\
				&=&\nonumber  \frac{c q^{k-1}(n-1)\cdot\cdot\cdot(n-q+1)}{(q-1)!}\\
				%\hspace{0.5cm}\text{y dado que $q$ es primo y ${n\choose q}$ es entero,}\\
				&=& q^{k-1}\cdot \frac{c (n-1)\cdot\cdot\cdot (n-q+1)}{(q-1)!} \label{eq:lem-left}
			\end{eqnarray}	
			%Dado lo anterior, como 
			Además, dado que $q$ divide a $n$, el 
			% siguiente número $r<n$ tal que $q$ divide a $r$ 
			mayor múltiplo de $q$ menor a $n$ es $n-q$. 
			%Esto se puede demostrar fácilmente por contradicción. Supongamos que existe un número $ \alpha $ en los naturales con $1\leq \alpha <q$ tal que $q$ divide a $(n-\alpha)$. Como $q$ divide a $n$, entonces necesariamente $q$ divide a la resta $n-(n-\alpha)$. De esto podemos afirmar que $q$ divide a $\alpha$,
			%\comentario{esta bien?}, 
			%lo cual lleva a una contradicción porque $1\leq \alpha <q$. 
			Entonces, $q$ no divide a ningún número $r\in\{n-q+1,n-q+2,...,n-1\}$, y como $q$ es primo, tampoco divide a la multiplicación de todos ellos, es decir, $q\nodiv (n-1)\cdot\cdot\cdot(n-q+1)$ (si $q$ no fuera un número primo, podría pasar que dividiera la multiplicación). Además, $q\nodiv c$ por como definimos $c$. De esto, de la igualdad \eqref{eq:lem-left} y el hecho que $q$ es primo, podemos concluir que $q^k\nodiv {n\choose q}$.
			
			Con lo anterior, y sabiendo que $a^{n-q}$ y $q^k$ son coprimos (ya que $\MCD(a,n) = 1$ y $q^k\divi  n$), sabemos que $q^k\nodiv {n\choose q} a^{n-q}$. Además, tenemos que
			\begin{eqnarray*}
				q^k\nodiv {n\choose q} a^{n-q}
				& \Rightarrow & c\cdot q^k\nodiv {n\choose q} a^{n-q}\\
				&  \Leftrightarrow &n\nodiv {n\choose q} a^{n-q}\\
				&  \Leftrightarrow& {n \choose q}a^{n-q} \not \equiv 0 \modl n
			\end{eqnarray*} 
			Hemos demostrado entonces que el coeficiente de $X^q$ en $P(X)$ no es idénticamente 0 en módulo $n$.
%en el puesto número $q$ de $\sum_{i=1}^{n-1}{n\choose q}X^q a^{n-q}\not \equiv 0 \modl n$. 
Luego, $P(X)$ no es idénticamente 0 en módulo $n$, y llegamos a que $(X+a)^n \not \equiv X^n+a \modl n$. Esto concluye la demostración del lema.		%Con esto terminamos ambas direcciones de la demostración del lema.
\end{itemize}
%\comentario{cambiar los nmid por not |}
	\end{proof}
	

	
