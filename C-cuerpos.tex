\subsection{Definiciones y propiedades básicas}
%\comentarioin{Habíamos hablado de poner $\oplus,\otimes$ para denotar la suma y multiplicacion en un cuerpo, y así dejar $q\cdot$ para denotar la suma de $q$ veces un cuerpo. Sin embargo, empecé a cambiar las cosas y la notación se volvió muy fea y además engorrosa (no podíamos denotar $-a$ por ejemplo. Creo que lo mejor es escribir la suma de $q$ términos de $a$ como $qa$ o $\textbf{q}a$. En cambio, cuando en el cuerpo estámos hablando del elemento neutro, preferí poner la notación $\mathbb{0},\mathbb{1}$ para denotarlos}
\begin{definition}[Cuerpo] \label{definicion_cuerpo}
Dado un conjunto $F$ y dos funciones (totales) $+ \colon R \times R \to R$ y $\cdot \colon R \times R \to R$, decimos que la terna $(F, +, \cdot)$ es un \emph{cuerpo} si $(F, +, \cdot)$ es un anillo conmutativo en el que $\0 \neq \1$ y $F^\ast = F \setminus \{\0\}$, es decir, todos los elementos distintos a $\0$ tienen inverso multiplicativo. 
\hfill$\blacksquare$
\end{definition}

La notación utilizada para trabajar con cuerpos es idéntica a la de anillos. Notemos que ahora también tienen sentido las expresiones $a^{-1}$ y $a^{-n}$ con $n$ un entero positivo, siempre que $a \neq \0$.

%\begin{definition}[Cuerpo]
%Un conjunto $F$ y dos funciones (totales) $\circ,+ : F\times
%F \to F$ forman un cuerpo si ambas estructuras algebraicas $(F,\circ)$ y $(F\setminus \{e_+\})$ son un grupo, donde $e_+$ el elemento neutro de $(F,\circ)$, y se cumplen las siguientes propiedades:
%\begin{enumerate}
%	\item $F$ es cerrado bajo las operaciones $\circ$ y $+$
%	\item Las operaciones $\circ, +$ son conmutativas \comentarioin{esto no iria si es que los grupos los definimos con conmutatividad}
%	\item La operación $\circ$ distribuye sobre $+$, es decir, 
%	$$(a+b)\circ(c + d) \ = \ (a \circ c ) + (a\circ d)+(b \circ c ) + (b\circ d)$$
%\end{enumerate}
%\end{definition}

\begin{proposition} \label{cuerpos_son_dominios_integrales}
Todo cuerpo es un dominio integral.
\end{proposition}

\begin{proof} 
Sea $(F, +, \cdot)$ un cuerpo, y sean $a, b \in F$ tales que $a \cdot b = \0$. Queremos probar que $a = \0$ o bien $b = \0$.

Supongamos que $b \neq \0$ (si no, no hay nada que demostrar). Entonces $b$ tiene un inverso multiplicativo, $b^{-1}$. Si multiplicamos ambos lados de la ecuación $a \cdot b = \0$ por $b^{-1}$, obtendremos $a = \0 \cdot b^{-1} = \0$, donde la última igualdad es por la propiedad 1 de la proposición \ref{propiedades_anillos}.
\end{proof}

Hacemos notar que no todo dominio integral es un cuerpo. Veamos el siguiente ejemplo.

\begin{example}
Sabemos que $(\mathbb{Z}, +, \cdot)$ es un dominio integral. Sin embargo, este no es un cuerpo, ya que, por ejemplo, $2 \neq \0$ y $2$ no tiene un inverso multiplicativo en $\mathbb{Z}$.
\end{example}

\ale{hasta aquí es nuevo}

Por último, vamos a definir lo que es un subcuerpo. Además, introducimos el concepto de cuerpo de extensión.
\begin{definition}
Sea $(F,+,\cdot)$ un cuerpo y $K\subseteq F$. Decimos que $(K,
+, \cdot)$ es un subcuerpo de $(F,+,\cdot)$ si $(K, +, \cdot)$ es un
cuerpo. En la misma línea, bajo las condiciones anteriores decimos que
$(F,+,\cdot)$ es un cuerpo de extensión de $(K,+,\cdot)$.
\end{definition}

Nótese que en la definición anterior usamos las mismas operaciones en
$(F,+,\cdot)$ y $(K, +, \cdot)$. En particular, se puede demostrar que
$(K, +, \cdot)$ y $(F,+,\cdot)$ tienen los mismos neutros bajo $+$ y
$\cdot$, lo cual es dejado como un ejercicio para el lector. Así, por
ejemplo, $(\mathbb{Q}, +, \cdot)$ es un subcuerpo de $(\mathbb{R},
+, \cdot)$, donde $+$ y $\cdot$ son la suma y multiplicación
usuales. Finalmente, nótese que si $(K, +, \cdot)$ es un subcuerpo de
$(F,+,\cdot)$, entonces $(K, +)$ y $(K \setminus \{\0\}, \cdot)$ son
subgrupos de $(F, +)$ y $(F \setminus \{\0\}, \cdot)$,
respectivamente.

%\comentarioin{aqui falta concluir con algo del estilo: los subcuerpos son subgrupos, etc}

%\comentarioin{Marcelo: hasta aquí revisé}

\subsection{Cuerpos finitos}
Los cuerpos finitos son de gran importancia para el desarrollo de este
documento.

Un cuerpo $(F,+,\cdot)$ se dice finito si $F$ es un conjunto
finito. Por ejemplo $(\mathbb{Z}_p, +, \cdot)$ es un cuerpo finito si
$p$ es un número primo y las operaciones $+, \cdot$ son realizadas en
módulo $p$.
%A continuación, vamos a definir formalmente el caso particular de los
%cuerpos finitos, los cuales 
%\begin{definition}[Cuerpo finito]
%	Una cuerpo finito
%        %(cuerpo de Galois)
%        $(F,+,\cdot)$ es un cuerpo con una cantidad finita de elementos.
%\end{definition}
Si $(F,+,\cdot)$ es un cuerpo finito, entonces $(F,+)$ es un grupo
finito, de lo cual concluimos por la proposición \ref{prop-orden} que
existe $n \in \mathbb{N}$ tal que $n \geq 1$ y $n \1 = \0$. Esta propiedad es
utilizada para definir la característica de un cuerpo finito, una
noción fundamental asociada a estas estructuras.
\begin{definition}
La característica de un cuerpo finito $(F, +, \cdot)$ se define como el menor $p\in \mathbb{N}$ tal que $p \geq 1$ y $p \1 = \0$, donde $\0$ y $\1$ son los neutros de las operaciones $+$ y $\cdot$, respectivamente.
%$\overbrace{\1 +\1 +\cdots +\1}^\text{$p$ veces}  = p\1 = \0 $ (suma de $p$ términos).
\end{definition}
%Si bien la característica de un cuerpo se puede definir de forma general, nosotros solo nos restringimos a los cuerpos finitos.\comentarioin{este comentario aportará? yo creo que no omitimos si lo ponemos}
Notemos que la característica de un cuerpo finito no puede ser 1 , ya
que $\1 \neq \0 $. El siguiente teorema nos provee más información
sobre la característica de un cuerpo finito.
\begin{theorem}\label{caracteristica primo}
Si $(F,+,\cdot)$ es un cuerpo finito de característica $p$, entonces
$p$ es un número primo.
\end{theorem}
\begin{proof}
Supongamos, por contradicción, que la característica $p$ de un cuerpo
finito $(F, +, \cdot)$ satisface que $p = ab$, con $1<a, b<p$. Luego,
\begin{eqnarray*}
	p\1  = \0 &\Rightarrow & (a b)\1 = \0 \\
        &\Rightarrow & (a b) (\1 \cdot \1) = \0 \\
	&\Rightarrow & (a\1) \cdot (b\1) =  \0 \quad\quad\quad \text{por la propiedad 5 de la proposición \ref{propiedades_anillos}}\\
	&\Rightarrow& a \1  = \0 \ \text{ o } \ b\1  = \0 \quad\quad\quad \text{por la proposición \ref{cuerpos_son_dominios_integrales}}
\end{eqnarray*}
En cualquiera de los dos casos, concluimos que existe un número $q <
p$ tal que $q > 1$ y $q \1 = \0$, lo cual contradice el hecho de que
$p$ es la característica de $(F, +, \cdot)$.
%Esto contradice nuestra elección de $p$, ya que cualquiera de los dos casos al sumar $\1$ una cantidad de veces menor que $p$ obtenemos que $\underbrace{\1 +\1 +\cdots +\1}_\text{$a$ o $b$ veces}  = \0 $. 
\end{proof}
El siguiente teorema relaciona la característica de un cuerpo con el
orden de este. Esto nos permite apreciar como es la estructura de un
cuerpo finito.
\begin{theorem}\label{orden cuerpo pot carac}
	Sea $(F,+,\cdot)$ un cuerpo finito y sea $p$ su
	característica. Entonces, el orden de $F$ es una potencia de
	$p$, es decir, $|F|=p^n$ para algún $n\in \mathbb{N}$.
\end{theorem}

\begin{proof}
Sea $(F,+,\cdot)$ un cuerpo finito de característica $p$, y
supongamos, por contradicción, que existe un primo $q\neq p$ tal que
$q$ divide a $|F|$. Como $(F,+)$ es un grupo finito, entonces por el
teorema de Cauchy (teorema \ref{teo cauchy}), existe un elemento $a\in
F$ que tiene orden $q$. Así, bajo la operación $+$ tenemos que
$\langle a\rangle = \{ 0 a, 1 a,...,(q-1) a\}$, y además tenemos que
$q a = \0$.
%y podemos concluir que
Adicionalmente, como $(F,+,\cdot)$ tiene característica $p$, se tiene
que $p x = \0 $ para todo $x \in F$, por la
proposición \ref{propiedades_anillos} y el hecho de que $p \1 =\0 $
(en particular, sabemos que $\1 \cdot x = x$ y $\0 \cdot x
= \0$). Como $p$ y $q$ son primos relativos, por la identidad de
Bézout sabemos que existen $s,t\in \mathbb{Z}$ tales que $p s + q t
= \MCD(p,q) = 1$.\footnote{Si el lector no está familiarizado con la
identidad de Bézout puede
visitar \url{https://es.wikipedia.org/wiki/Identidad_de_B\%C3\%A9zout}.}
Nótese que podemos ocupar la identidad de Bézout ya que
$p,q\in\mathbb{N}$.
%(no necesariamente son elementos de $F$).
Luego,
\begin{eqnarray*}
	a &=& 1a\\
        &=& (p s + q t) a\\
        &=& (p s) a + (q t) a \quad\quad\quad \text{por la definición de $m b$ para $m \in \mathbb{Z}$ y $b \in F$}\\
        &=& (p s) (a \cdot \1) + (q t) (a \cdot \1)\\
	&=& (pa) \cdot (s\1) + (qa)\cdot (t\1) \quad\quad\quad \text{por la propiedad 5 de la proposición \ref{propiedades_anillos}}\\
	&=& \0 \cdot (s\1) + \0 \cdot (t\1)\\
	&=& \0 + \0 \quad\quad\quad \text{por la propiedad 1 de la proposición \ref{propiedades_anillos}}\\
        &=& \0
\end{eqnarray*}
Pero esto es una contradicción ya que el orden de $a$ en $(F,+)$ es
$q\geq 2$, por lo que $a \neq \0$ (puesto que $a = 1a$ y $1a \neq \0$).
\end{proof}
%\comentarioin{Aca hay que hacer un lazo que de mas continuidad.}

%\comentarioin{¿El siguiente ejemplo es necesario?}
%Un ejemplo particular de cuerpos finitos es el cuerpo $$F_p := (\mathbb{Z}_p,+,\cdot),$$ donde $p$ es un número primo. \comentarioin{desde aqui hacia abajo revisar}
%Este cuerpo finito $F_q$ cobra importancia debido a que podemos demostrar que cualquier cuerpo $K$ de orden $q$ es isomorfo a $F_q$, es decir, podemos $K$ y $F_q$ tienen la misma forma. Esto implica que al trabajar con $K$ podemos asumir que estamos trabajando en $F_q$.
%\comentarioin{revisar esta parte. Encontré demostraciones pero implicaría mucha más teoría. Quizás se podría hablar de qué es un isomorfismo y por qué podríamos trabajar con $F_q$}

%\comentarioin{Marcelo: hasta aquí llegué}

A continuación vamos a mostrar la forma estándar para construir un
cuerpo finito con característica $p$ y orden $p^n$, para $n \geq 1$.
Sea $h(X)$ un polinomio en $\mathbb{Z}_p[X]$ de grado $n$. Definimos
el conjunto
\begin{eqnarray*}
\mathbb{Z}_p[X]/h(X) & := & \{p(X) \mods h(X) \ \mid \ p(X)\in \mathbb{Z}_p[X]\}\\
&=& \{p(X) \mods (h(X),p) \ \mid \ p(X)\in \mathbb{Z}[X]\}
\end{eqnarray*}
En el caso de que $h(X)$ es irreducible en $\mathbb{Z}_p[X]$, se tiene
que $(\mathbb{Z}_p[X]/h(X), +, \cdot)$ es un cuerpo finito de
característica $p$ y orden $p^n$, donde las operaciones $+$ y $\cdot$
sobre polinomios son definidos en módulo $(h(X),p)$.
%\comentarioin{aca tengo que explicar como saber que un polinomio de grado mayor que 3 es irreducible en $Z_p[X]$, ya que al parecer, que no tenga raíces en $Z_p$ no implica que no sea irreducible.}
\begin{example}
Para construir un cuerpo de 9 elementos, debemos construir el cuerpo
$\mathbb{Z}_3[X]/h(X)$, donde $h(X)$ es un polinomio irreducible en
$\mathbb{Z}_3[X]$ y tiene grado 2. Para esto, notamos que $h(X) =
X^2+2X+2$ es irreducible en $\mathbb{Z}_3[X]$ puesto que $h(0) = 2$,
$h(1) = 2$ y $h(2) = 1$, y un polinomio de grado 2 es irreducible en
$\mathbb{Z}_3[X]$ si y sólo si no tiene raíces en este cuerpo. Tenemos
que:
\begin{eqnarray*}
	\mathbb{Z}_3[X]/(X^2+2X+2) &=& \{a_0 +a_1X \ \mid \
	a_0,a_1 \in \mathbb{Z}_p \}\\
	&=& \{0,\,1,\,2,\,X,1+X,\,2+X,\,2X,1+2X,2+2X\}
\end{eqnarray*}
Por lo tanto, tenemos que $(\mathbb{Z}_3[X]/(X^2+2X+2), +, \cdot)$ es
un cuerpo, donde $0, 1$ son los elementos neutros de la suma y la
multiplicación, respectivamente. Por ejemplo, $(1+2X) + (1+X) = 2$ en
este cuerpo, puesto que $(1+2X) + (1+X) \equiv 2 \modl 3$, de lo cual
se deduce que $(1+2X) + (1+X) \equiv 2 \modl (X^2+2X+2, 3)$. Siguiendo
esta línea de razonamiento, concluimos que $2+X$ es el inverso aditivo
de $1+2X$ puesto que $(1+2X) + (2+X) \equiv 0 \modl (X^2+2X+2, 3)$. De
la misma forma, $2X+2$ es el inverso multiplicativo de $X+1$ puesto
que
\begin{eqnarray*}
(X+1)(2X+2) & = & 2X^2 + 4X + 2\\
& = & 2X^2 + 4X + 4 - 2\\
& = & 2(X^2 + 2X + 2) - 2\\
& \equiv & 2(X^2 + 2X + 2) + 1 \modl 3,
\end{eqnarray*}
y de esto se deduce que $(X+1)(2X+2) \equiv 1 \modl (X^2+2X+2,
3)$. Finalmente, nótese que $(\mathbb{Z}_3, +, \cdot)$
es un subcuerpo de $(\mathbb{Z}_3[X]/(X^2+2X+2), +, \cdot)$. \qed
\end{example}

%\comentarioin{2 nuevos teoremas para no usar isomorfismos}
El siguiente teorema nos reafirma sobre las buenas propiedades de los polinomios sobre cuerpos.
\begin{theorem}\label{raices en polinomios sobre cuerpos}
Si $F$ es un cuerpo y $p(X)$ un polinomio no nulo de grado $k$ sobre $F[X]$, entonces $p(X)$ tiene a lo más $k$ raíces en $F$.
\end{theorem}
\begin{proof}
Vamos a demostrar el teorema por inducción sobre el grado del
polinomio. Para el caso base consideramos $k=0$. Como $p(X)$ es no
nulo, concluimos que $p(X)$ no tiene raíces y se cumple lo
requerido. Para el paso inductivo supongamos que todo polinomio no
nulo sobre $F[X]$ de grado $k$ tiene a los más $k$ raíces en $F$, y
sea $p(X)$ un polinomio sobre $F[X]$ de grado $k+1$. Si $p(X)$ no
tiene raíces en $F$ entonces el resultado se cumple trivialmente, por
lo que vamos a asumir que existe una raíz $c \in F$ de
$p(X)$. Luego, podemos escribir $p(X)$ como
\begin{eqnarray*}
p(X) & = & (X-c)\cdot q(X),
\end{eqnarray*}
donde $q(X)$ es un polinomio de grado $k$ (dejamos como ejercicio para
el lector la demostración de este propiedad).
%Es claro que $q(X)$ es de grado $k$, y
Por la hipótesis de inducción, sabemos que $q(X)$ tiene a lo más $k$
soluciones en $F$. Es importante observar que todo elemento que $d\in
F$ que no sea $c$ o una raiz de $q(X)$ no puede ser una raiz de
$p(X)$, ya que tendríamos que $(d-c)\neq \0$ y $q(d)\neq \0$, y por la proposición \ref{cuerpos_son_dominios_integrales}
sabemos inmediatamente que $p(d)\neq\0$ (si $F$ no fuera un cuerpo
esta propiedad no necesariamente se cumpliría).  Luego, $p(X)$ tiene a
lo más $k+1$ soluciones en $F$.
\end{proof}
Notemos que el resultado del teorema \ref{raices en polinomios sobre
cuerpos} no es cierto para todas las estructuras algebraicas. Por
ejemplo, si tomamos la estructura $(\mathbb{Z}_6,+,\cdot)$, que no es
un cuerpo, y el polinomio $p(X) = 2X+4$ sobre $\mathbb{Z}_6[X]$,
entonces tenemos que $p(X)$ es de grado 1 pero tiene 2 raíces: 1 y 4.

Para finalizar esta sección mostraremos un resultado fundamental sobre
teoría de cuerpos, que utilizaremos en el desarrollo del documento.
\begin{theorem}\label{subgrupo de cuerpo  es ciclico}
Sea $(F,+,\cdot)$ un cuerpo y $(G,\cdot)$ un subgrupo finito del grupo
de $(F\setminus\{\0\},\cdot)$.
%multiplicativo de este. 
Luego, este subgrupo es un grupo cíclico, es decir, existe un elemento
$a\in G$ tal que $\langle a\rangle = G$.
\end{theorem}
\begin{proof}
%Sean $a,b\in G$, con $O_G(a) = m$ y $O_G(b)=n$. Por el
%corolario \ref{corolario orden}, sabemos que existe un elemento en $G$
%con orden multiplicativo igual a $\MCM(m,n)$.
Sea $c\in G$ el elemento de mayor orden multiplicativo. Esto quiere
decir que para todo $d\in G$:
\begin{eqnarray*}
O_G(d) & \leq & O_G(c).
\end{eqnarray*}
Nótese que este elemento $c$ existe ya que $(G,\cdot)$ es un grupo
finito.  Mostraremos a continuación que el orden multiplicativo de
cualquier elemento en $G$ divide a $O_G(c)$. Sea $d\in G$. Por el
corolario \ref{corolario orden}, sabemos que existe un elemento $c'$
con $O_G(c') = \MCM(O_G(d),O_G(c))$. Por definición del mínimo común
múltiplo sabemos que $O_G(c)$ divide a $\MCM(O_G(d),O_G(c))$, de lo
que concluimos que $O_G(c)\leq O_G(c')$. Por otro lado, como $c'\in G$
y $c$ es el elemento de mayor orden, tenemos que $O_G(c')\leq
O_G(c)$. De esto concluimos que $O_G(c)=O_G(c')$. Así, sabemos que
$O_G(d)$ divide a $\MCM(O_G(d),O_G(c)) = O_G(c') = O_G(c)$, que es
exactamente lo que afirmamos.
	
	Por último, notemos que cada elemento $d\in G$ cumple con:
	\begin{eqnarray*}
		d^{O_G(c)} &=& d^{\alpha O_G(d)}\\
		&=& (d^{O_G(d)})^{\alpha}\\
		&=& \1^{\alpha}\\
		&=& \1
	\end{eqnarray*}
%
De lo anterior podemos concluir que cada elemento de $G$ es una raiz
del polinomio $p(X)=X^{O_G(c)} - 1$ sobre $F[X]$. Por el
teorema \ref{raices en polinomios sobre cuerpos}, sabemos que $p(X)$
tiene a lo más $O_G(c)$ raíces. Así, podemos afirmar que $|G|\leq
O_G(c)$. Finalmente, sabemos que el orden de cualquier elemento de un
grupo es menor o igual al orden del grupo, por lo que también podemos
afirmar que $O_G(c)\leq |G|$. De esta forma se deduce que
$O_G(c)= |G|$. Por lo tanto, se tiene que $\langle c\rangle = G$, puesto
que $\langle c\rangle \subseteq G$ (ya que $c \in G$), y queda
demostrado el~resultado.
\end{proof}


