%\subsection{Definiciones y propiedades básicas}
%\begin{definition}[Grupo]
%Un conjunto $G$ y una función (total) $\circ : G \times
%G \to G$ forman un grupo si:
%\begin{enumerate}
%	\item Para cada $a,b,c \in G$: $(a \circ b) \circ c = a \circ (b
%\circ c)$

%	\item Existe $e \in G$ tal que para cada $a \in G$: $a \circ e = e \circ a = a$

%   \item Para cada $a \in G$, existe $b \in G$: $a \circ b 	= b \circ a = e$
%\end{enumerate}
%\end{definition}
%\comentarioin{Al parecer los nombres derivados de extranjeros se escriben sin mayuscula: http://aplica.rae.es/orweb/cgi-bin/v.cgi?i=huRPfQssalCJLqSk}
\begin{definition}[Grupo, neutro, inverso, grupo abeliano, orden]
Un \emph{grupo} es un par $(G,\circ)$, donde $G$ es un conjunto y $\circ \colon G\times G\to G$ es una función (total) que satisface los siguientes axiomas:
\begin{enumerate}
	\item $\circ$ es asociativa: para todos $a,b,c \in G$ se cumple que $(a \circ b) \circ c = a \circ (b
\circ c)$.

	\item Existe un $e \in G$ (llamado \emph{neutro}) tal que para todo $a \in G$ se cumple que $a \circ e = e \circ a = a$

   \item Para cada $a \in G$ existe un $b \in G$ (llamado \emph{inverso} de $a$) tal que $a \circ b 	= b \circ a = e$
\end{enumerate}
Decimos que $(G,\circ)$ es un \emph{grupo abeliano} si es un grupo que además cumple la siguiente propiedad:
\begin{enumerate}
	\item[4.] $\circ$ es conmutativa: para todos $a,b\in G$ se cumple que $a\circ b = b\circ a$. 
\end{enumerate}
Al número $|G|$ se le llama el \emph{orden} del grupo $(G, \circ)$. \hfill$\blacksquare$
\end{definition}

Cuando la operación $\circ$ está clara por contexto, es usual referirse al grupo $(G, \circ)$ escribiendo simplemente $G$. 

\begin{proposition}
El elemento neutro de un grupo es único.
\end{proposition}

\begin{proof}
Sea $(G,\circ)$ un grupo, y supongamos que $e, e' \in G$ son ambos neutros. Como $e$ es neutro, tenemos que $e \circ e' = e'$. Pero $e'$ también es neutro, así que $e \circ e' = e$. Luego, por transitividad, $e = e'$.
\end{proof}

En el contexto de la teoría de grupos, usualmente la letra $e$ está reservada para el neutro. Sin embargo, lo ideal es siempre especificar la notación. 

\begin{proposition} \label{inverso_unico}
	Cada elemento de un grupo tiene un único inverso.
	\end{proposition}
	
	\begin{proof}
	Sea $(G,\circ)$ un grupo, $a \in G$ y supongamos que $b, c \in G$ son ambos inversos de $a$. Luego
	\begin{align*}
	b &= b \circ e && e \text{ es neutro} \\
	b &= b \circ \left( a \circ c \right) && c \text{ es inverso de } a \\
	b &= \left( b \circ a \right) \circ c && \text{$\circ$} \text{ es asociativa} \\
	b &= e \circ c && b \text{ es inverso de } A \\
	b &= c && e \text{ es neutro}.
	\end{align*}
	\end{proof}

En virtud de la proposición \ref{inverso_unico}, podemos usar la notación $a^{-1}$ para denotar al (único) inverso del elemento $a$. Para que dicha notación tenga sentido es necesario que el grupo subyacente esté claro por contexto. 
Notemos que $e^{-1} = e$, es decir, el elemento neutro es su propio inverso. En general, puede ocurrir que existan elementos en el grupo distintos al neutro que igualmente tengan esa propiedad.

Como expresiones de la forma $a \circ b \circ c$ no presentan ambigüedad (ya que $\circ$ es asociativa), los paréntesis son opcionales y usualmente solo se incluyen cuando hacen más claro un cálculo o demostración.

\begin{example} \label{ejemplo_Z}
El conjunto $\mathbb{Z}$ de los números enteros forma un grupo con la operación de suma. En este caso, el neutro es el número $0$, y el inverso de un $a \in \mathbb{Z}$ será $-a \in \mathbb{Z}$. \hfill$\blacksquare$
\end{example}

\begin{proposition}[Leyes de cancelación] \label{cancelacion_grupos}
	Sea $(G, \circ)$ un grupo, y sean $a, b, c \in G$. Se cumple que:
	\begin{enumerate}
		\item Si $a \circ c = b \circ c$, entonces $a = b$.
		\item Si $c \circ a = c \circ b$, entonces $a = b$.
	\end{enumerate}
\end{proposition}

\begin{proof}
Probaremos solo la primera afirmación: la segunda se demuestra análogamente. Tenemos que
$$a = a \circ e = a \circ \left( c \circ c^{-1} \right) = \left(a \circ c \right) \circ c^{-1} = \left(b \circ c \right) \circ c^{-1} = b \circ \left( c \circ c^{-1} \right) = b \circ e = b.$$
\end{proof}

\begin{definition}[Potenciación en un grupo]
Sea $(G, \circ)$ un grupo con neutro $e$ y $a \in G$. Definimos $a^0 \coloneqq e$ y, si $n$ es un entero positivo, definimos $a^n \coloneqq \underbrace{a \circ a \circ \cdots \circ a}_{n \text{ veces}}$ y $a^{-n} \coloneqq \underbrace{a^{-1} \circ a^{-1} \circ \cdots \circ a^{-1}}_{n \text{ veces}}$. \hfill$\blacksquare$
\end{definition}

Dejamos de ejercicio al lector verificar que la potenciación tiene las siguientes propiedades.

\begin{prop} Sea $(G, \circ)$ un grupo. Se cumple que:
	\begin{enumerate}
		\item $a^m \circ a^n = a^{m+n}$ para todo $a \in G$ y todos $m, n \in \mathbb{Z}$.
		\item $\left( a^m \right)^n = a^{mn}$ para todo $a \in G$ y todos $m, n \in \mathbb{Z}$.
		\item Si $G$ es abeliano, entonces $\left( a \circ b \right)^{n} = a^n \circ b^n$ para todos $a, b \in G$ y todo $n \in \mathbb{Z}$.
	\end{enumerate}
\end{prop}


\begin{example} \label{grupo_ciclico}
Sea $n$ un entero positivo. Consideremos el conjunto de $n$ elementos $\mathbb{Z}_n \coloneq \left\{ 0, 1, \dots, n-1 \right\}$, y sea $\circ$ la operación de suma módulo $n$, es decir, $a \circ b$ es el resto de dividir $a+b$ en $n$. Entonces $(G, \circ)$ es un grupo abeliano. En efecto, se puede verificar que $\circ$ es asociativa y conmutativa, que $0$ es el elemento neutro, y que el inverso de $a \in \mathbb{Z}_n$ es $n-a \in \mathbb{Z}_n$ (excepto si $a = 0$, que es su propio inverso).
Este grupo tiene una propiedad particular: hay un elemento (el $1$) tal que sus potencias generan todos los elementos del grupo. En otras palabras, todo elemento de $\mathbb{Z}_n$ es una potencia del elemento $1$. Por ejemplo, en $\mathbb{Z}_5$, tenemos que $2 = 1 + 1$, $3 = 1 + 1 + 1$, $4 = 1 + 1 + 1 + 1$ y $0 = 1 + 1 + 1 + 1 + 1$ (considerando la suma en módulo $5$).
\hfill$\blacksquare$
\end{example}

\begin{remark} Notemos que, en el contexto del ejemplo \ref{grupo_ciclico}, sería confuso utilizar la notación $1^4$ para referirnos a la cuarta potencia de $1$, pues lo que queremos representar es el número que se obtiene \textit{sumando} $1$ consigo mismo $4$ veces. En estos casos en los que la operación del grupo se asemeja más a la suma que a la multiplicación, se suele usar la notación aditiva para denotar a las potencias. De esta forma, si $n \in \mathbb{Z}$ y $a$ es un elemento del grupo, escribimos $n a$ en lugar de $a^n$. \hfill$\blacksquare$
\end{remark}


La siguiente definición se refiere a los grupos que viven naturalmente dentro de otros grupos.

%--------------------------------------
\begin{definition}[Subgrupo]
Sea $(G,\circ)$ un grupo. Se dice que un subconjunto $H \subseteq G$ es un \emph{subgrupo} de $(G,\circ)$ si $(H, \circ)$ es un grupo (con la misma operación). \hfill$\blacksquare$
\end{definition}

\begin{remark}
	Todos los subgrupos de un grupo abeliano son abelianos. \hfill$\blacksquare$
	\end{remark}

Para probar que $(H, \circ)$ es un subgrupo de $(G, \circ)$ debemos, incluso antes de verificar los axiomas, comprobar que $H$ sea cerrado bajo la operación binaria $\circ$. En otras palabras, necesitamos que $\circ$ restringida a $H \times H$ tenga recorrido contenido en $H$. 


Notemos que, \textit{a priori}, no sabemos si el neutro y los inversos de un subgrupo coinciden con los del grupo más grande. Demostraremos a continuación que eso efectivamente ocurre.


\begin{proposition}
Sea $(H, \circ)$ un subgrupo del grupo $(G, \circ)$. Las siguientes afirmaciones se cumplen:
	\begin{enumerate}
		\item Si $e_1$ es el neutro en $(G, \circ)$ y $e_2$ es el neutro de $(H, \circ)$, entonces $e_1 = e_2$

		\item Para cada $a \in H$, si $b$ es el inverso de $a$ en $(H, \circ)$, entonces $b$ es el inverso de $a$ en $(G, \circ)$.
	\end{enumerate}
\end{proposition}


\begin{proof} \text{ }
\begin{enumerate}
\item Como $e_1$ es el neutro en $(G, \circ)$, $e_2\in H$ y $H \subseteq G$, se cumple que $e_2 \circ e_1 = e_2$.
Por otro lado, sabemos que $e_2$ es el neutro en $(H, \circ)$, por lo que $e_2 \circ e_2 = e_2$
(notemos que se usa la misma operación en $H$ y en $G$).
De esto concluimos que $e_2 \circ e_1 = e_2 \circ e_2$. Usando la proposición \ref{cancelacion_grupos} concluimos que $e_1 = e_2$.
\item Sea $a^{-1}$ el inverso de $a$ en $(G, \circ)$. Queremos probar que $b = a^{-1}$. Por la primera propiedad de esta proposición, sabemos que $(G, \circ)$ y $(H,
\circ)$ tienen el mismo neutro, que llamaremos $e$. 
Dado que $a^{-1}$ es el inverso de $a$ en $(G, \circ)$, tenemos que $a \circ a^{-1} = e$. Además, dado que $b$ es el inverso de $a$ en $(H, \circ)$, se cumple que $a \circ b = e$. Luego $a \circ a^{-1} = a \circ b$. Usando la proposición \ref{cancelacion_grupos} concluimos que $a^{-1} = b$.
\end{enumerate}
\end{proof}

\begin{example} \label{ejemplo_subgrupo}
Sea $m$ un entero positivo. Definimos el conjunto
$m\mathbb{Z} \coloneq \left\{mk \mid k \in \mathbb{Z}\right\}$ de los múltiplos de $m$. Notemos que $m\mathbb{Z}$ es cerrado bajo la operación de suma: si $mk_1$ y $mk_2$ son elementos de $m\mathbb{Z}$, entonces $mk_1 + mk_2 = m(k_1 + k_2) \in m\mathbb{Z}$. Dejamos al lector verificar que $m\mathbb{Z}$ es un subgrupo de $(\mathbb{Z}, +)$. \hfill$\blacksquare$
\end{example}


A continuación hablaremos sobre clases laterales y grupos cocientes. Puesto que todos los grupos que utilizaremos en el documento son abelianos, vamos a agregar esa hipótesis a los teoremas para simplificar las demostraciones. Sin embargo, hacemos notar que toda esa teoría también puede aplicarse a grupos no abelianos si se consideran algunas sutilezas adicionales. 

\begin{definition}[Clases laterales] \label{definicion_clase_lateral}
Sea $(G, \circ)$ un grupo abeliano, y sea $H$ un subgrupo de $G$. Dado un elemento $g \in G$, definimos la \emph{clase lateral} de $g$ bajo $H$ como el conjunto
$$g \circ H \coloneq \left\{ g \circ h \mid h \in H \right\}.$$ Al conjunto de las clases laterales de $G$ bajo $H$ se le denota $\faktor{G}{H}$.
\hfill$\blacksquare$
\end{definition}

Notemos que, según la definición \ref{definicion_clase_lateral}, $H$ en sí mismo es la clase lateral de $e \in G$. En general, ocurrirá que a varios elementos distintos de $G$ les corresponderá la misma clase lateral. Dejamos al lector verificar que la clase lateral de cualquier elemento de $H$ es justamente $H$. También hacemos notar que las clases laterales no necesariamente son subgrupos de $G$.

\begin{prop} \label{lema:rel bin}
Sea $(G, \circ)$ un grupo abeliano, y sea $H$ un subgrupo de $G$. Las clases laterales bajo $H$ forman una partición de $G$.
\end{prop}

\begin{proof}
Definiremos la siguiente relación binaria en $G$: $a \sim b$ si y solo si $b \circ a^{-1} \in H$. Afirmamos que $\sim$ es una relación de equivalencia. En efecto:
\begin{itemize}
\item Sea $a \in G$ arbitrario. Tenemos que $a \circ a^{-1} = e \in H$, ya que $H$ es un subgrupo. Luego $a \sim a$, y $\sim$ es refleja.
\item Sean $a, b \in G$ tales que $a \sim b$. Eso último significa que $b \circ a^{-1} \in H$. Como $H$ es un subgrupo, tenemos que $\left( b \circ a^{-1} \right)^{-1} \in H$ (ya que los subgrupos son cerrados bajo tomar inversos). Notemos que $\left( b \circ a^{-1} \right)^{-1} = a \circ b^{-1}$, pues
$$\left( b \circ a^{-1} \right) \circ \left(a \circ b^{-1} \right) = b \circ \left( a^{-1} \circ a \right) \circ b^{-1} = b \circ e \circ b^{-1} = b \circ b^{-1} = e.$$
Entonces $a \circ b^{-1} \in H$, lo que significa que $b \sim a$. Esto prueba que $\sim$ es simétrica.
\item Sean $a, b, c \in G$ tales que $a \sim b$ y $b \sim c$. Esto significa que $b \circ a^{-1} \in H$ y $c \circ b^{-1} \in H$. Pero $(c \circ b^{-1}) \circ (b \circ a^{-1}) = c \circ a^{-1}$, y sabemos que la operación
$\circ$ es cerrada en $H$. Por lo tanto, $c \circ a^{-1} \in H$. Luego
$a\sim c$, y $\sim$ es transitiva.
\end{itemize}
Ahora veremos que las clases de equivalencia que la relación $\sim$ induce en $G$ son justamente las clases laterales. Para ello, tomemos un $g \in G$ arbitrario, y denotemos por $\left[ g \right]_{\sim}$ a su clase de equivalencia bajo $\sim$. Queremos probar que $g \circ H = \left[ g \right]_\sim$. Veamos que
\begin{align*}
a \in g \circ H \qquad & \Leftrightarrow \qquad \exists\, h \in H \quad a = g \circ h \\
& \Leftrightarrow \qquad \exists\, h \in H \quad a \circ g^{-1} = h \\
& \Leftrightarrow \qquad a \circ g^{-1} \in H \\
& \Leftrightarrow \qquad g \sim a \\
& \Leftrightarrow \qquad a \in  \left[ g \right]_\sim.
\end{align*}
\end{proof}


\begin{prop} \label{cardinalidad_clases_laterales}
Sea $(G, \circ)$ un grupo abeliano finito, y sea $H$ un subgrupo de $G$. Todas las clases laterales bajo $H$ tienen la misma cardinalidad, es decir, $|a \circ H| = |b \circ H|$ para cualesquiera $a, b \in G$.
\end{prop}

\begin{proof}
Vamos a demostrar que todas las clases laterales bajo $H$ tienen cardinalidad $|H|$.

Sea $g \in G$ arbitrario. Queremos probar que $|g \circ H| = |H|$. Definimos la función $f\colon H \rightarrow (g \circ H)$ tal que $f(h) = g \circ h$. Afirmamos que $f$ es una biyección. En efecto: 
\begin{itemize}
\item Sean $h_1, h_2 \in H$ tales que $f(h_1) = f(h_2)$. Entonces $g \circ h_1 = g \circ h_2$. Usando cancelación, obtenemos que $h_1 = h_2$. Luego $f$ es inyectiva.
\item Sea $b \in g \circ H$. Entonces existe un $h \in H$ tal que $b = g \circ h$. Luego $f(h) = b$, y $f$ es sobreyectiva.
\end{itemize}
\end{proof}


El siguiente es un resultado importante en la teoría de grupos finitos, pues establece una relación entre el orden de un grupo y el orden de sus subgrupos. Hacemos notar que el teorema es válido incluso para grupos no abelianos. 

\begin{theorem}[Lagrange]
\label{teo:lagrange}
Si $(G, \circ)$ es un grupo abeliano finito y $H$ es un subgrupo de $G$, entonces $|H|$ divide a $|G|$.
\end{theorem}

\begin{proof}
Por la proposición \ref{lema:rel bin} sabemos que $|G|$ es la suma de las cardinalidades de todas las clases laterales de $G$ bajo $H$. Por otro lado, la proposición \ref{cardinalidad_clases_laterales} nos dice que cada una de dichas clases laterales tiene cardinalidad $|H|$. Luego $$|G| = \left| \faktor{G}{H}\right| \cdot |H|,$$ de lo que se deduce el resultado. 
\end{proof}

\begin{example} \label{ejemplo_lagrange}
Supongamos que $G$ es un grupo abeliano de orden $6$, y queremos encontrar todos sus subgrupos. En primer lugar, tenemos los subgrupos triviales $\{e\} \subseteq G$ y $G \subseteq G$. Lo que nos dice el teorema de Lagrange es que cualquier otro subgrupo necesariamente debe tener orden $2$ o $3$, pues estos son los divisores no triviales de $6$. Así, sabemos de antemano que, por ejemplo, no puede existir un subgrupo de orden $4$.
\hfill$\blacksquare$
\end{example}

Notemos que el teorema \ref{teo:lagrange} no garantiza que para cada divisor de $|G|$ habrá un subgrupo con ese orden. Más adelante veremos un recíproco parcial de este resultado para el caso de los divisores primos de $|G|$ (teorema \ref{teo cauchy}).


A continuación estudiaremos cómo se comporta la operación del grupo con las clases laterales. Veamos el siguiente ejemplo.

\begin{example} \label{ejemplo_clases_laterales}
Continuamos con el ejemplo \ref{ejemplo_subgrupo}. Consideremos el grupo $(\mathbb{Z}, +)$ junto con el subgrupo $4\mathbb{Z}$. Dado $a \in \mathbb{Z}$, tenemos que $a + 4\mathbb{Z}$ es el conjunto de enteros de la forma $a + 4k$ para algún $k \in \mathbb{Z}$. Notemos que, si $r$ es el resto de dividir $a$ en $4$, entonces $a + 4\mathbb{Z} = r + 4\mathbb{Z}$. Por lo tanto, tenemos que
$$\faktor{\mathbb{Z}}{4\mathbb{Z}} = \left\{ 4\mathbb{Z},\, 1+4\mathbb{Z},\, 2+4\mathbb{Z},\, 3+4\mathbb{Z}\right\}.$$
Notemos que, si $a$ es cualquier elemento de $1+4\mathbb{Z}$ y $b$ es cualquier elemento de $2+4\mathbb{Z}$, entonces $a+b$ tendrá resto $3$ al dividirse en $4$, por lo que $a+b \in 3+4\mathbb{Z}$. El mismo razonamiento sirve para cualquier otro par de clases laterales. Esto indica que, ultimadamente, no importa los representantes concretos que saquemos de las clases laterales, pues la operación del grupo se comporta bien entre ellas.  
\hfill$\blacksquare$
\end{example}

La siguiente proposición se refiere al fenómeno que observamos en el ejemplo \ref{ejemplo_clases_laterales}.

\begin{prop}[Grupo cociente] \label{grupo_cociente}
Sea $(G, \circ)$ un grupo abeliano, y sea $H$ un subgrupo de $G$. La siguiente operación binaria\footnote{El símbolo $\ast$ no es estándar para denotar la operación en el grupo cociente. Generalmente se utiliza el mismo símbolo de la operación de $G$. En este caso, evitamos hacer eso por claridad.} $\ast$ en $\faktor{G}{H}$ está bien definida, y el conjunto $\faktor{G}{H}$ forma un grupo abeliano con ella:
$$\left(a \circ H \right) \ast \left( b \circ H \right) \coloneqq  (a \circ b) \circ H .$$
\end{prop}

\begin{proof}
Sea $e$ el neutro de $G$.

Lo primero que debemos mostrar es que $\ast$ está bien definida, es decir, que $\left(a \circ H \right) \ast \left( b \circ H \right)$ no depende de los representantes concretos $a$ y $b$. Para ello, consideremos $a', b' \in G$ tales que $a \circ H = a' \circ H$ y $b \circ H = b' \circ H$. Queremos probar que $(a \circ b) \circ H = (a' \circ b') \circ H$. Como $e \in H$ y $a' = a' \circ e$, tenemos que $a' \in a' \circ H$. Como $a' \circ H = a \circ H$, se sigue que $a' \in a \circ H$. Análogamente, tenemos que $b' \in b \circ H$. Luego existen $h_1, h_2 \in H$ tales que $a' = a \circ h_1$ y $b' = b \circ h_2$. Por lo tanto, $a' \circ b' = (a \circ b) \circ (h_1 \circ h_2)$. Como $h_1 \circ h_2 \in H$, esto último muestra que $a' \circ b' \in (a \circ b) \circ H$, y luego $(a \circ b) \circ H = (a' \circ b') \circ H$. Concluimos que $\ast$ está bien definida.

Ahora debemos probar que $\left(\faktor{G}{H},\, \ast\right)$ es un grupo abeliano. La asociatividad y la conmutatividad se heredan directamente de $\circ$ hacia $\ast$. Dejamos al lector verificar que el elemento neutro es $H = e \circ H$, y que el inverso de $a \circ H$ es $a^{-1} \circ H$.
\end{proof}

Hacemos notar que, en la proposición \ref{grupo_cociente}, $\faktor{G}{H}$ no es un subgrupo de $G$. De hecho, ni siquiera es un subconjunto de $G$. Veamos otra vez el ejemplo \ref{ejemplo_clases_laterales}. Las cuatro clases laterales no son enteros, sino subconjuntos de $\mathbb{Z}$. Ahora sabemos que esas cuatro clases laterales forman un grupo, y, si miramos con atención, notaremos que ese grupo es esencialmente idéntico al grupo de $\mathbb{Z}_4 = \left\{0, 1, 2, 3\right\}$ con la operación de suma en módulo $4$. Resulta incómodo no poder afirmar que ambos grupos son idénticos solo porque sus elementos tengan naturaleza distinta. Esto motiva la siguiente definición.

\begin{definition}[Isomorfismo] \label{isomorfismo_de_grupo}
Sean $(G_1, \circ_1)$ y $(G_2, \circ_2)$ dos grupos. Diremos que una función $\phi\colon G_1 \rightarrow G_2$ es un \emph{isomorfismo} si $\phi$ es una biyección y, además,
$$\forall\, a, b \in G_1 \qquad \phi(a \circ_1 b) = \phi(a) \circ_2 \phi(b).$$
Si existe un isomorfismo, se dice que los grupos $G_1$ y $G_2$ son \emph{isomorfos}, y se escribe $G_1 \simeq G_2$. \hfill$\blacksquare$
\end{definition}

En otras palabras, un isomorfismo es una biyección entre los grupos que también induce una biyección entre las relaciones definidas por las operaciones binarias. Uno puede pensar que son traductores. Por lo tanto, según la definición \ref{isomorfismo_de_grupo}, operar dos elementos $a, b \in G_1$ y luego aplicar el traductor $\phi$ tiene el mismo efecto que aplicar el traductor a $a$ y a $b$ por separado, y luego operarlos según la operación de $G_2$.

No es difícil convencerse de que dos grupos isomorfos tienen exactamente las misma propiedades. Por ejemplo, si $G_1 \simeq G_2$, entonces $G_1$ es abeliano si y solo si $G_2$ es abeliano, o $G_1$ tiene un subgrupo de $n$ elementos si y solo si $G_2$ tiene un subgrupo de $n$ elementos.


\begin{example} \label{ejemplo_exponencial}
Se puede verificar que los números reales, $\mathbb{R}$, forman un grupo con la suma. Por otro lado, los reales positivos, $\mathbb{R}^{+}$, forman un grupo con la multiplicación. La función $\varphi\colon \mathbb{R} \rightarrow \mathbb{R}^{+}$ dada por $\varphi(x) = e^x$ es biyectiva, y además $e^{x+y} = e^x \cdot e^y$. Luego $(\mathbb{R}, +) \simeq (\mathbb{R}^{+}, \cdot)$. \hfill$\blacksquare$
\end{example}

Resumiremos nuestra discusión del ejemplo \ref{ejemplo_clases_laterales} en la siguiente proposición.

\begin{prop}
Sea $m$ un entero positivo, y considere los grupos abelianos $(\mathbb{Z}, +)$ y $(\mathbb{Z}_m, +)$. Entonces $$\faktor{\mathbb{Z}}{m \mathbb{Z}} \; \simeq \; \mathbb{Z}_m.$$
\end{prop}


La siguiente es otra definición clave en la teoría de grupos.
 
\begin{definition}[Conjunto generado]\label{def_gen}
	Sea $(G,\circ)$ un grupo, y sea $a \in G$.
	El \emph{conjunto generado} por $a$ es
	$$\langle a \rangle \; \coloneq \; \{a^k\; \mid \; k\in\mathbb{Z}\}.$$
    \hfill$\blacksquare$
\end{definition}

A pesar de que la notación no lo indica, en la definición \ref{def_gen} el conjunto generado $\langle a \rangle$ depende del grupo $G$ al cual pertenece $a$. Usualmente esto está claro por contexto, pero también podemos usar la notación $\langle a \rangle_{G}$ para evitar ambigüedades.


\begin{proposition}\label{prop-generado}
	Sea $(G,\circ)$ un grupo, y sea $a \in G$. Entonces $(\langle a\rangle , \circ)$ es un subgrupo abeliano de $(G,\circ)$. 
\end{proposition} 


\begin{proof}
Como $\langle a\rangle\subseteq G$, solo hace falta demostrar
que $(\langle a\rangle,\circ)$ es un grupo. Tenemos que verificar la clausura de la operación y los axiomas:
\begin{itemize}
\item Para la clausura, sean $b,c \in \langle
a\rangle$. Sabemos que existen $n, m \in \mathbb{Z}$ tales que $b = a^m$ y $c = a^n$. Luego $b \circ c = a^{m+n} \in \langle a\rangle$, y $\circ$ es cerrada en $\langle a \rangle$.
\item La asociatividad se hereda directamente.
\item Por definición tenemos que $a^0$ es el neutro del grupo, y $a^0 \in \langle a \rangle$. 
\item Si $b \in \langle a \rangle$, entonces $b = a^n$ para algún $n \in \mathbb{Z}$. Tenemos que $a^{-n} \in \langle a \rangle$ es el inverso de $b$, por lo que $\langle a \rangle$ es cerrado bajo tomar inversos.
\item Para la conmutatividad, sean $b,c \in \langle
	a\rangle$. Sabemos que existen $n, m \in \mathbb{Z}$ tales que $b = a^m$ y $c = a^n$. Luego $b \circ c = a^{m + n} = a^{n + m} = c \circ b$.
\end{itemize}
\end{proof}

\begin{definition}[Generador, grupo cíclico] 
Sea $(G, \circ)$ un grupo. Decimos que un elemento $a \in G$ es un \emph{generador} de $G$ si $G = \langle a \rangle$. Decimos que $(G, \circ)$ es \emph{cíclico} si tiene un generador.
\hfill$\blacksquare$
\end{definition}

\begin{remark}
Un grupo cíclico puede tener más de un generador. Por ejemplo, vimos anteriormente vimos que $1$ es un generador de $(\mathbb{Z}_5, +)$ (ver ejemplo \ref{grupo_ciclico}). Sin embargo, también $4$ es un generador, pues $4 + 4 = 3$, $4 + 4 + 4 = 2$, $4 + 4 + 4 + 4 = 1$ y $4 + 4 + 4 + 4 + 4 = 0$ (considerando todas las sumas en módulo $5$).
\hfill$\blacksquare$
\end{remark}

\begin{definition}[Orden de un elemento]\label{def_orden}
	Sea $(G,\circ)$ un grupo con neutro $e$, y sea $a \in G$. El \emph{orden} de $a$, denotado $\ord{G}(a)$, es el menor entero positivo $m$ tal que $a^m = e$. Si dicho $m$ no existe, decimos que $a$ tiene \emph{orden infinito}. \hfill$\blacksquare$
\end{definition} 

\begin{remark}
En cualquier grupo, el neutro es el único elemento con orden $1$. \hfill$\blacksquare$
\end{remark}

Estamos usando el término «orden» en dos sentidos distintos, pues ya habíamos definido el orden de un grupo como su cardinalidad. Esto no suele generar confusión, ya que es claro cuando se habla del grupo y cuando se habla de un elemento de él. 

\begin{proposition} \label{prop-orden}
Sea $(G, \circ)$ un grupo, y sea $a \in G$ un elemento con orden finito, digamos $\ord{G}(a) = m$. Entonces
$$\langle a \rangle = \{ a^0, a^1, \dots, a^{m-1}\}.$$
Más aún, $(\langle a \rangle, \circ) \simeq (\mathbb{Z}_m, +)$.
\end{proposition}

\begin{proof} Sea $e$ el neutro de $(G, \circ)$. Por definición, $m$ es el menor entero positivo tal que $a^m = e$.

Veamos primero que $\langle a \rangle \subseteq \{ a^0, a^1, \dots, a^{m-1}\}$. Sea $b \in \langle a \rangle$ arbitrario, de modo que $b = a^k$ para algún $k \in \mathbb{Z}$. Por el algoritmo de la división de enteros, existe un $q \in \mathbb{Z}$ y un $r \in \{0, 1, \dots, m-1\}$ de modo que $k = mq+r$. Entonces $b = a^{mq+r} = \left(a^m\right)^q \circ a^r = e^q \circ a^r = a^r$. Como $a^r \in \{ a^0, a^1, \dots, a^{m-1}\}$, esto prueba que $\langle a \rangle \subseteq \{ a^0, a^1, \dots, a^{m-1}\}$. Por otro lado, es claro que $\{ a^0, a^1, \dots, a^{m-1}\} \subseteq \langle a \rangle$, así que tenemos la igualdad.

Ahora estableceremos el isomorfismo. Esto, en particular, implicará que $\abs{\langle a \rangle} = m$. Consideremos la función $\phi\colon \mathbb{Z}_m \rightarrow \{ a^0, a^1, \dots, a^{m-1}\}$ dada por $\phi(k) = a^k$. Claramente $\phi$ es sobreyectiva.

Para la inyectividad, sean $i, j \in \mathbb{Z}_m$ tales que $a^i = \phi(i) = \phi(j) = a^j$. Sin pérdida de generalidad, podemos asumir que $i \geq j$. Entonces $$a^{i - j} = a^i \cdot a^{-j} = a^i \cdot \left(a^j \right)^{-1} = a^i \cdot \left(a^i \right)^{-1} = e.$$
Como $0 \leq i - j < m$, no puede ocurrir que $i-j$ sea un entero positivo, pues $m$ es el menor entero positivo $k$ tal que $a^k = e$. Esto obliga a que $i-j = 0$, es decir, $\phi$ es inyectiva.

Por último, veamos que $\phi$ es un isomorfismo. Sean $i, j \in \mathbb{Z}_m$. Tenemos que
$$\phi(i + j) = a^{i+j} = a^i \cdot a^j = \phi(i) \cdot \phi(j).$$ Esto completa la demostración.
\end{proof}

Lo que dice el isomorfismo de la proposición \ref{prop-orden} es que todos los grupos cíclicos del mismo orden son esencialmente iguales.

\begin{corollary} \label{orden de elemento divide a orden de grupo}
Si $(G, \circ)$ es un grupo abeliano finito y $a \in G$, entonces $\ord{G}(a)$ divide a $\abs{G}$.
\end{corollary}

\begin{proof}
Como $\langle a \rangle$ es un subgrupo de $G$, el teorema de Lagrange implica que $\abs{\langle a \rangle} = \ord{G}(a)$ divide a $\abs{G}$.
\end{proof}

\begin{remark}
Dado un grupo $G$, un elemento $a \in G$ es un generador si y solo si $\ord{G}(a) = |G|$.
\hfill$\blacksquare$
\end{remark}

Notemos que el orden del subgrupo generado por un elemento coincide con el orden de dicho elemento. Esta es la razón por la que el término «orden» se utiliza en ambos sentidos.

%https://resources.saylor.org/wwwresources/archived/site/wp-content/uploads/2011/05/Order-group-theory.pdf


De la demostración de la proposición \ref{prop-orden} también se desprende el siguiente resultado:

\begin{corollary} \label{observacion orden}
Sea $(G, \circ)$ un grupo con neutro $e$, y sea $a \in G$ un elemento con orden finito, digamos $\ord{G}(a) = m$. Dado un $k \in \mathbb{Z}$, tenemos que $a^k = e$ si y solo si $m$ divide a $k$.
\end{corollary}

El siguiente es un resultado fundamental en la teoría de grupos. Se desprende directamente de los corolarios que acabamos de discutir. Nuevamente, la hipótesis de abelianidad no es necesaria, y solo la agregamos por simplicidad.

\begin{prop} \label{PTF para grupos finitos}
Sea $(G, \circ)$ un grupo abeliano finito con neutro $e$. Entonces $a^{|G|} = e$ para todo $a \in G$.
\end{prop}

\begin{proof}
Sea $m = \ord{G}(a)$. Por el corolario \ref{orden de elemento divide a orden de grupo} tenemos que $m$ divide a $|G|$. Luego, por el corolario \ref{observacion orden}, $a^{|G|} = e$.
\end{proof}


Los siguientes resultados exponen las primeras conexiones entre la teoría de grupos y la teoría de números.

%\comentarioin{nuevo teorema y corolario para no usar isomorfismos}
\begin{prop}\label{orden del producto coprimos}
	Sea $(G,\circ)$ un grupo abeliano, y sean $a,b\in G$ con $\ord{G}(a)=m$ y $\ord{G}(b)=n$. Si $m$ y $n$ son coprimos, entonces $\ord{G}(a\circ b)=mn$.
\end{prop}

\begin{proof}
	Sea $e$ el neutro. Sea $r=\ord{G}(a\circ b)$. Debemos demostrar que $r = mn$. Tenemos que 
	\begin{eqnarray*}
		(a\circ b)^{mn} &=& a^{mn}\circ b^{mn}\\
                &=& (a^m)^n\circ (b^n)^m\\
		&=&e^n\circ e^m \\
		 &=&  e.
	\end{eqnarray*}
Luego, por el corolario \ref{observacion orden}, tenemos que $r$ divide a $mn$, y por lo tanto $r\leq mn$. Nos falta probar que $mn\leq r$. Para ello, notemos que, como $e = (a\circ b)^r$, entonces
\begin{eqnarray*}
	 e \ = \ e^n & =&  (a\circ b)^{rn} \\
	 & = & a^{rn} \circ b^{rn}\\
	 & = & a^{rn} \circ (b^{n})^{r}\\
	 & = & a^{rn} \circ e^{r}\\
	 &=& a^{rn}.
\end{eqnarray*}
 De la misma forma, tenemos que
\begin{eqnarray*}
	 e \ = \ e^m & =&  (a\circ b)^{rm} \\
	 & = & a^{rm} \circ b^{rm}\\
	 & = & (a^{m})^r \circ b^{rm}\\
	 & = & e^{r} \circ b^{rm}\\
	 &=& b^{rm}.
\end{eqnarray*}
Como $a^{rn}=e$ y $b^{rm} = e$, usando el corolario \ref{observacion orden} deducimos que $m \divi rn$ y $n\divi rm$. Como $m$ y $n$ son coprimos, necesariamente $m \divi r$ y $n\divi r$. Así, tenemos que $mn\divi r$, ya que $\MCD(m,n)=1$. De eso se sigue que $mn \leq r$, y esto concluye la demostración. 
\end{proof}

Podemos generalizar la proposición \ref{orden del producto coprimos} de la siguiente manera. 

\begin{corollary}\label{corolario orden}
Sea $(G,\circ)$ un grupo abeliano, y sean $a,b\in G$ con $\ord{G}(a)=m$ y $\ord{G}(b)=n$. Entonces existe un elemento $c\in G$ tal que $\ord{G}(c)=\MCM(m,n)$.
\end{corollary}
\begin{proof}
Sea $e$ el neutro. Sean $p_1,\dots,p_k$ todos los primos que dividen a $m$ o a $n$. Podemos escribir $m$ y $n$ como 
$$m \ = \ \prod\limits_{i=1}^k p_i^{\alpha_i} \ \qquad\text{y}\qquad \ n \ = \ \prod\limits_{i=1}^k p_i^{\beta_i},$$ donde $\alpha_i,\beta_i \geq 0$ (son iguales a 0 cuando el primo $p_i$ no es divisor del número). Además, sabemos que
$$\MCM(m, n) \ = \ \prod\limits_{i=1}^k p_i^{\max(\alpha_i,\, \beta_i)}.$$

Ahora definimos
$$m' \ = \ \prod\limits_{i\,:\,\alpha_i\geq \beta_i} p_i^{\alpha_i} \ \qquad\text{y}\qquad \ n' \ = \ \prod\limits_{i\,:\,\beta_i> \alpha_i} p_i^{\beta_i}.$$
Por ejemplo, si $m=2^3\cdot 3^2\cdot 5^1$ y $n=2^1\cdot 3^2\cdot 7^1$, entonces $m' = 2^3\cdot 3^2\cdot 5^1$ y $n' = 7^1$. Notemos que las siguientes afirmaciones sobre $m'$ y $n'$ se cumplen:
\begin{itemize}
	\item $m'$ divide a $m$
	\item $n'$ divide a $n$
	\item $m'$ y $n'$ son coprimos
	\item $\MCM(m,n) = m'n'$
\end{itemize}
Sean $q_1 = \frac{m}{m'}$ y $q_2 = \frac{n}{n'}$, y consideremos los elementos $a' = a^{q_1}$ y $b' = b^{q_2}$. Vamos a mostrar que $\ord{G}(a')=m'$ y $\ord{G}(b')=n'$. Primero, sea $k=\ord{G}(a')$. Entonces $$e\ = \ (a')^k\ = \ (a^{q_1})^k \ = \ a^{q_1k}$$
Por el corolario \ref{observacion orden}, tenemos que $\ord{G}(a) = m$ divide a $q_1k$. Es decir, existe un $q_3 \in \mathbb{Z}$ tal que $q_3 m = q_1k$, de lo que se deduce que $q_3 m' = k$. Como $k$ y $m'$ son enteros positivos, eso implica que $m' \leq k$. Por otro lado, $$(a')^{m'} \ = \ (a^{q_1})^{m'} \ = \ a^m \ = \ e$$ 
%Nuevamente, por la observación \ref{observacion orden} necesariamente $k$ divide a $m'$, y
Como $k=\ord{G}(a')$, deducimos que $k\leq m'$. Luego, $\ord{G}(a')=k=m'$. De forma análoga, se puede mostrar que $\ord{G}(b')=n'$.

%Una propiedad importante que se sigue del teorema \ref{orden del producto coprimos} y el corolario \ref{corolario orden} es la siguiente.

Por último, definimos $c=a'\circ b'$. Como $(G,\circ)$ es un grupo abeliano, y $a',b'\in G$ son tales que $\ord{G}(a') = m'$, $\ord{G}(b')= n'$, y $\MCD(m', n') = 1$, la proposición \ref{orden del producto coprimos} nos permite concluir que $\ord{G}(c) = m'n' = \MCM(m,n)$. Esto concluye la demostración.
\end{proof}

El último resultado que demostraremos en esta sección es el teorema de Cauchy. Nuevamente, por simplicidad solo demostraremos el caso abeliano (que es el que necesitaremos más adelante), pero hacemos notar que el teorema es válido para cualquier grupo finito. En cierto sentido, es un recíproco parcial del teorema de Lagrange.

\begin{theorem}[{Teorema de Cauchy}]\label{teo cauchy}
Sea $p$ un número primo que divide al orden de un grupo abeliano finito
$(G,\circ)$. Entonces existe un elemento de $G$ que tiene orden $p$.
\end{theorem}

\begin{proof}
Dejaremos fijo $p$ y haremos inducción en $n \coloneqq |G|$. 

Notemos que el resultado es vacuamente cierto si $p$ no divide a $n$, por lo que solo debemos revisar los casos en que $n$ sea efectivamente un múltiplo de $p$. 

El caso base es $n = p$. Como $p \geq 2$, existe un elemento $a \in G$ con $\ord{G}(a) > 1$ (ya que el único elemento con orden $1$ es el neutro). Por el teorema de Lagrange, tenemos que $\ord{G}(a)$ divide a $p$, y, como $p$ es primo, necesariamente $\ord{G}(a) = p$.

Para el paso inductivo, nuevamente consideramos un elemento $a \in G$ con $\ord{G}(a) = m > 1$. Hay dos casos:
\begin{itemize}
\item El primer caso es que $p$ divide a $m$. Entonces $a^{\frac{m}{p}}$ es un elemento de orden $p$.
\item El segundo caso es que $p$ no divide a $m$. Sabemos que $\langle a \rangle$ es un subgrupo, así que podemos considerar el cociente $H = \faktor{G}{\langle a \rangle}$, que también será un grupo abeliano. Notemos que
$$|G| = \left|\faktor{G}{\langle a \rangle}\right| \cdot \left|\langle a \rangle\right|.$$
Como $p$ divide a $|G|$ y no divide a $\left|\langle a \rangle\right| = m$, tenemos que $p$ divide a $|H|$. Como $H$ es un grupo abeliano con $|H| < |G| = n$, podemos aplicar la hipótesis de inducción: existe un $B \in H$ con $\ord{H}(B) = p$. Dicho $B$ es una clase lateral de $G$ bajo $\langle a \rangle$, es decir $B = c \circ \langle a \rangle$ para algún $c \in G$. Sea $k = \ord{G}(c)$. Notemos que, por definición de la operación en el grupo cociente $H$ (proposición \ref{grupo_cociente}), se cumple que
$$B^k = (c \circ \langle a \rangle)^k = c^k \circ \langle a \rangle = e \circ \langle a \rangle,$$
por lo que $\ord{H}(B)$ divide a $k$ por el corolario \ref{observacion orden}. Como $\ord{H}(B) = p$ y $k = \ord{G}(c)$, tenemos que $c^{\frac{k}{p}}$ es un elemento de orden $p$.
\end{itemize}
Como en ambos casos encontramos un elemento que tiene orden $p$, esto completa la inducción.
\end{proof}


% Ale: no estoy segura de si los lemas que estaban demostrando a continuación son necesarios para el documento, o si solo eran un camino para demostrar la versión no abeliana del teorema de Cauchy. Por el momento, escribiré una demostración simplificada para el caso abeliano, y dejaré la original comentada.

\begin{comment}
Para demostrar el teorema \ref{teo cauchy}, vamos a demostrar el lema \ref{tcf}, que es una versión más fuerte del teorema de Cauchy.
\begin{lemma}\label{tcf}
Sea $p$ un número primo que divide el orden de un grupo finito
$(G, \circ)$. Entonces $G$ tiene $p\cdot k$ elementos que son solución
de la ecuación $$x^p=e,$$ donde $e$ es el neutro de $(G, \circ)$ y $k$
es un número entero positivo.
%\comentarioin{al parecer este es el teorema de cauchy original (?)}
\end{lemma}
%\comentarioin{hay que poner en la demostracion que estamos demostrando el lema A13? o se entiende?}
\begin{proof}
Sea $|G|= n $, y suponga que $p\divi n$, es decir, $n= p\cdot c$ con
$c\in \mathbb{N}$. Definimos el conjunto
\begin{eqnarray*}
S & = & \{(g_0,g_1,\ldots,g_{p-1})\mid g_i\in G \text{ para cada }
i \in [0,p-1] \text{ y } g_0 \cdot g_1 \cdot \ldots \cdot g_{p-1} =
e\}.
\end{eqnarray*}
Notemos que si $(g_0,g_1,\ldots,g_{p-1})\in S$, entonces al escoger
los elementos $g_0,\ldots,g_{p-2}$ el elemento $g_{p-1}$ queda
únicamente determinado: como $g_0 \cdot \ldots \cdot g_{p-2}\cdot
g_{p-1}=e$, entonces necesariamente $g_{p-1} = (g_0 \cdot \ldots \cdot
g_{p-2})^{-1}$. De esto podemos concluir que $|S|=n^{p-1}$, usando el
argumento que para cada $i\in [0,p-2]$, $g_i\in G$, y tenemos $n$
posibles elementos que podemos escoger. Definamos la relación binaria
$\sim$ sobre las tuplas de $S$ de la siguiente forma:
\begin{multline*}
(g_0,g_1,\ldots,g_{p-1})\sim (h_0,h_1,\ldots,h_{p-1}) \ \ \text{si y solo si} \\
(h_0,h_1,\ldots,h_{p-1})\text{ es una permutación cíclica de } (g_0,g_1\ldots,g_{p-1}), 
\end{multline*}
es decir, si $(h_0,h_1,\ldots,h_{p-1}) = (g_{(0+k) \mods p},
g_{(1+k) \mods p}, \ldots g_{(p-1+k) \mods p})$ para algún $k\in
[0,p-1]$. Demostrar que $\sim$ es una relación de equivalencia es un
ejercicio sencillo y queda propuesto para el lector.  Ya que $\sim$
satisface esta propiedad, podemos analizar las clases de equivalencia
bajo esta relación.  Podemos ver que existen dos tipos de tuplas en
$S$.  El tipo I ocurre cuando todos los elementos de la tupla son el
mismo, en cuyo caso tenemos que $|[(g_0,\ldots,g_{p-1})]_{\sim}|=1$
(al hacer permutaciones cíclicas obtenemos la misma tupla).  El tipo
II occure cuando en la tupla existen elementos distintos. Nótese que, en general, si permutamos una tupla del tipo II podríamos eventualmente obtener la misma tupla. Por ejemplo, si nuestra tupla es $(a,b,a,b)$, entonces podemos hacer una permutación cíclica de dos hacia la izquierda (vale decir, $k = 2$) y obtenemos la misma tupla $(a,b,a,b)$. En la afirmación \ref{afirmacion-perm-cicl} demostramos que
%si el tamaño de la tupla es un número primo (como en nuestro caso)
en nuestro caso no sucede lo anterior porque $p$ es un número primo.
%entonces no sucede lo anterior, en cuyo caso 
De esta forma concluimos que $|[(g_0,\ldots,g_{p-1})]_{\sim}|=p$.


\begin{afirmacion}\label{afirmacion-perm-cicl}
%Sea $p$ un número primo.
Si $(g_0, g_1, \ldots, g_{p-1}) = (g_{(0+k) \mods p}, g_{(1+k) \mods
p}, \ldots, g_{(p-1+k) \mods p})$ para algún $k \in [1,p-1]$, entonces
$g_0 = g_1 = \cdots = g_{p-1}$.
\end{afirmacion}

\begin{proof}
Suponga que $(g_0, g_1, \ldots, g_{p-1}) = (g_{(0+k) \mods p},
g_{(1+k) \mods p}, \ldots, g_{(p-1+k) \mods p})$ para algún $k \in
[1,p-1]$. Así, tenemos que $g_i = g_{(i+k) \mods p}$ para cada
$i \in [0,p-1]$. Nótese que de esto se deduce que $g_{((j\mods p) +k)\mods p} = g_{(j+k)\mods p}$, de lo cual se concluye que
\begin{align}\label{eq-lem-perm-cicl}
g_0 = g_k = g_{(2\cdot k) \mods p} = g_{(3\cdot k) \mods p} = \cdots = g_{((k-1) \cdot k) \mods p}.
\end{align}
Por lo tanto, si demostramos que
\begin{eqnarray*}
(i \cdot k) \mods p & \neq & (j \cdot k) \mods p
\end{eqnarray*}
para cada $i,j \in [0,p-1]$ tales que $i < j$, entonces concluimos que
$g_0 = g_1 = \cdots = g_{p-1}$ ya que todos los elementos de la tupla
$(g_0, g_1, \ldots, g_{p-1})$ son mencionados
en \eqref{eq-lem-perm-cicl}. Por el contrario, supongamos que $(i \cdot
k) \mods p \ = \ (j \cdot k) \mods p$ para algún par $i,j \in [0,p-1]$ tal
que $i < j$. Tenemos entonces que $i \cdot k \equiv j \cdot k \modl
p$, vale decir, $(j-i) \cdot k \equiv 0 \modl p$. Así, dado que $k \in
[1,p-1]$ y $p$ es un número primo, concluimos que $(j-i) \equiv
0 \modl p$. Pero $0 < j-i < p$, por lo que obtenemos una
contradicción.
\end{proof}

Continuando entonces con la demostración del lema \ref{tcf},
supongamos que hay $r$ clases de equivalencia de tuplas del tipo I y
$q$ clases de equivalencia de tuplas del tipo II, y observemos que
$r\geq 1$ ya que $(e,e,\ldots,e)$ es de tipo I. Luego, como todas las
tuplas de $S$ caen en una clase de equivalencia de $\sim$, se cumple
que $r\cdot 1 \ + \ q\cdot p = |S|$. Además, tenemos que:
\begin{eqnarray*}
r\cdot 1 \ + \ q\cdot p \ = \ |S| &\Rightarrow&  r \ + \ q\cdot p \ = \ n^{p-1}\\ 
	&\Rightarrow& r \ = \ n^{p-1} \ - \ q\cdot p\\
	&\Rightarrow& r \ = \ (p \cdot c)^{p-1} - q \cdot p \quad\quad\quad \text{dado que } n= p\cdot c\\
	&\Rightarrow& r \ = \ p\cdot (c\cdot (p \cdot c)^{p-2}-q).
\end{eqnarray*}
Sea $k = c\cdot (p \cdot c)^{p-2}-q$. Dado que $p\geq 2$, tenemos que
$k \in \mathbb{Z}$. Así, dado que $r = p \cdot k$ y $r \geq 1$,
concluimos que $k$ es un número entero positivo.
%entonces
%necesariamente $k\geq 1$ (un número natural).
Luego, existen $r = p\cdot k$ tuplas de tipo I,
%que sus clases de equivalencia son del primer
%tipo,
es decir, $r=p\cdot k$ elementos distintos $g\in G$ tales que $g \cdot
g \cdot \ldots \cdot g = g^p = e$, los cuales corresponden a las
soluciones de la ecuación $x^p=e$.
\end{proof}
Habiendo demostrado el lema \ref{tcf}, podemos hacer la demostración del teorema de Cauchy.
\begin{proof}[Demostración del teorema \ref{teo cauchy}]
Supongamos que $(G,\circ)$ es un grupo finito, y $p$ es un número
primo que divide el orden de $G$. Por el lema \ref{tcf}, sabemos que
$G$ tiene $p\cdot k$ soluciones de la ecuación $x^p=e$, donde $e$ es
el neutro de $(G,\circ)$ y $k$ es un número entero positivo. Como
$p\cdot k \geq 2$, entonces existe un elemento $a\in G$ tal que $a\neq
e$ y $a^p=e$. Supongamos que $O_G(a)=m$ con $m<p$. Nótese que $m > 1$
puesto que $a \neq e$, y $e = a^m = a^p$. Usando el hecho de que
podemos escribir $p=\alpha \cdot m+\beta$, con $\alpha \in \mathbb{N}$
y $\beta = p \mods m$, concluimos que
\begin{eqnarray*}
	e & = & a^p \\
	&=& a^{\alpha \cdot m + \beta}\\
	&=& (a^{m})^{\alpha} \circ a^{\beta}\\
	&=& e^{\alpha}\circ a^{\beta}\\
	&=& a^{\beta}
\end{eqnarray*} 
Como $p$ es primo, $1<m<p$ y $\beta = p \mods m$, tenemos que
$\beta \in [1,m-1]$. Pero esto contradice el hecho que $O_G(a) = m$,
puesto que $a^{\beta}=e$. De esta forma concluimos que $O_G(a) = p$,
lo que demuestra que existe un elemento de $G$ que tiene orden $p$.
\end{proof}
\end{comment}
%\comentarioin{Bernardo: hay que cerrar de alguna forma este preliminar? o se corta aqui nomas?}
